\chapter{State Proof Generator}

This introduces a description for Solana's 'Light-Client' state proof generator.

This part's crucial components are defined by Solana's replication protocol
design and consist of:
\begin{enumerate}
    \item Input data format ('Light-Client' state data structure).
    \item Transaction auxiliary proof.
    \item Proof system used for the proof generation.
    \item Circuit definition used for the proof system.
\end{enumerate}

\section{'Light-Client' State}

Block Information $\bar B_k$ is defined as follows:
\begin{itemize}
    \item $k$ - the number of the block
    \item $B_k = H(B_{k - 1} || \texttt{account\_hash} || \texttt{signature\_count\_buf} || b_k || \texttt{validators\_state})$ - bank hash of the block\footnote{See \url{https://docs.solana.com/proposals/simple-payment-and-state-verification\#block-headers}}
    \item $b_k$ Merkle Block
    \item $B_{k - 1}$ - the previous block's bank hash
    \item \texttt{validators\_state} is not implemented for now. 
\end{itemize}

Proof algorithm input is defined as follows:
\begin{itemize}
    \item $n_1$ - current confirmed block number
    \item $n_2$ - new confirmed block number
    \item $\{\bar B_{n_1}, \dots, \bar B_{n_2}, \dots, \bar B_{n_2 + 32}$\} - block information for blocks from $n_1$ to $n_2 + 32$.
    \item $\sigma_0, \dots, \sigma_N$ - signatures for $B_{n_2 + 32}$
\end{itemize}

Approximate code representation of such a state data structure is as follows:
\lstset{language=C++,
    basicstyle=\ttfamily,
    keywordstyle=\color{blue}\ttfamily,
    stringstyle=\color{red}\ttfamily,
    commentstyle=\color{green}\ttfamily,
    morecomment=[l][\color{magenta}]{\#}
}
\begin{lstlisting}[frame=single]
template<typename Hash>
struct block_data {
    typedef typename Hash::digest_type digest_type;

    std::size_t block_number;
    digest_type bank_hash;
    digest_type merkle_hash;
    digest_type previous_bank_hash;
    //    std::vector<vote_state> votes;
};

template<typename Hash, typename SignatureSchemeType>
struct state_type {
    typedef Hash hash_type;
    typedef SignatureSchemeType signature_scheme_type;
    typedef typename signature_scheme_type::signature_type signature_type;

    std::size_t n_1 confirmed;
    std::size_t n_2 new_confirmed;
    std::vector<block_data<hash_type>> repl_data;
    std::vector<signature_type> signatures;
};
\end{lstlisting}

Validator state-representing data structure (\texttt{vote\_state}) supposes such
a state to begin being handled by Solana replication protocol (or its
implementation) for handling the tracking of votes state being unchanged 'till
the end of epoch.

\section{Transaction Proof}

Since state proof sequence is represented as a short Merkle-tree fingerprint and 
does not allow to check transactions without additional mechanisms (as described
in \ref{subsec:stateseq}), there is a need to introduce additional transaction 
proof as simple Merkle-tree inclusion proofs with low verification costs. 

Recall the current state representation on the Ethereum side. 
$H(T_{n_1, n_2})$ is a Merkle tree root of blocks $\{n_1, ..., n_2\}$.
Each block $n_i$ contains transactions tree $Tx_{n_i}$ with a root $H(Tx_{n_i})$.

The prover wants to show that transaction $tx$ was included in the block $n_i$.
Let $D_{n_1, n_2}$ be a hash table.
\begin{enumerate}
    \item If transaction can be spent only once (i.e. it is not a "state-check" transaction):
        \begin{enumerate}
            \item Check that $D_{n_1, n_2}$ does not contain $H(tx)$. 
        \end{enumerate}
    \item The prover provides path of $tx$ to the $H(Tx_{n_i})$ and shows that $H(Tx_{n_i})$ is included in $n_i$.
    \item The prover provides the path of $H(n_i)$ to the $H(T_{n_1, n_2})$.
    \item If transaction can be spent only once:
        \begin{enumerate}
            \item Add $H(tx)$ to $D_{n_1, n_2}$. 
        \end{enumerate}
\end{enumerate} 
This is enough to prove that the transaction was included in the confirmed state. 

\begin{remark}
    Transaction proof can be included in the original state-proof circuit.
    In this case, a state prover can include as many transaction proofs as they desire.
    It (almost) does not influence the resulting verification costs and proof size.
\end{remark}

The transaction proof contains $\log(n_2 - n_1) + \log(k)$ hashes where $k$ is number of transactions in the block. 
Note that only the transaction's hash value will be written to Ethereum's storage.
All required information is available to the light client.

\section{Proof System}
\label{section:proof_system}
\textbf{WIP}

The proof system used for proving Solana's 'Light-Client' state on EVM is Redshift SNARK\cite{cryptoeprint:2019:1400}.
RedShift is a transparent SNARK that uses PLONK\cite{cryptoeprint:2019:953} proof system but replaces the commitment scheme.
Initial paper proposal is to employ FRI\cite{ben2018fast} protocol to obtain transparency for the PLONK system.

However, FRI cannot be straightforwardly used with the PLONK system.
To achieve the required security level without huge overheads, the authors introduce \textit{list polynomial commitment} scheme as a part of the protocol.
For more details, the reader gets referred to \cite{cryptoeprint:2019:1400}.

The original RedShift protocol utilizes the classic PLONK\cite{cryptoeprint:2019:953} system.
To provide better performance, the original protocol is generalized to be used
with PLONK with custom gates \cite{turbo}, \cite{plonkhalo2} and lookup arguments \cite{cryptoeprint:2020:315}, \cite{lookuphalo2}.

\section{Optimizations}
\label{section:optimizations}
\textbf{WIP}

\subsection{Batched FRI}

Instead of check each commitment individualy, we can aggregate them for FRI.
For polynomials $f_0, \dots, f_k$:
\begin{enumerate}
    \item Get $\theta$ from transcript
    \item $f = f_0 \cdot \theta^{k - 1} + \dots + f_k$
    \item Run FRI over $f$, using oracles to $f_0, \dots, f_k$
\end{enumerate}

Thus, we can run only one FRI instance for all commited polynomials.

See \cite{cryptoeprint:2019:1400} for details.

\subsection{Hash By Column}

Instead of committing each of the polynomials, we can use the same Merkle tree for several polynomials. 
It decreases the number of Merkle tree paths that need to be provided by the prover. 

See \cite{cryptoeprint:2019:1076}, \cite{cryptoeprint:2019:1400} for details.

\subsection{Hash By Subset}

On the each $i + 1$ FRI round, the prover should send all elements from a coset $H \in D^{(i)}$.
Each Merkle leaf is able to contain the whole coset instead of separate values. 

See \cite{cryptoeprint:2019:1076} for details.
Similar approach is described in \cite{cryptoeprint:2019:1400}.
However, the authors of \cite{cryptoeprint:2019:1400} use more values per leaf, that leads to better performance. 

\section{RedShift Protocol}
\label{section:protocol}
\textbf{WIP}

Notations:

\begin{center}
\begin{table}[H]
\begin{tabular}{| l | l |}
 	\hline
	$N_{\texttt{wires}}$ & Number of wires (`advice columns`) \\
	\hline
	$N_{\texttt{perm}}$ & Number of wires that are included in the permutation argument \\
	\hline
	$N_{\texttt{sel}}$ & Number of selectors used in the circuit \\
	\hline
	$N_{\texttt{const}}$ & Number of constant columns  \\
	\hline 
	$N_{\texttt{lookups}}$ & Number of lookups \\
	\hline
	$\textbf{f}_i$ & Witness polynomials, $0 \leq i < N_{\texttt{wires}}$  \\
	\hline
	$\textbf{f}_{c_i}$ & Constant-related polynomials, $0 \leq i < N_{\texttt{const}}$  \\
	\hline
	$\textbf{gate}_i$ & Gate polynomials, $0 \leq i < N_{\texttt{sel}}$  \\
	\hline
	$\sigma(\text{col : } i, \text{row : } j) = (\text{col : } i', \text{row : } j')$ & Permutation over the table \\
	\hline
\end{tabular}
\end{table}
\end{center}

For details on polynomial commitment scheme and polynomial evaluation scheme, we refer the reader to \cite{cryptoeprint:2019:1400}.

\paragraph{Preprocessing:}


\begin{algorithm}[h]
\begin{enumerate}
	\item $\mathcal{L}' = (\textbf{q}_{0}, ..., \textbf{q}_{N_{\texttt{sel}}})$
	\item Let $\omega$ be a $2^k$ root of unity
	\item Let $\delta$ be a $T$ root of unity, where $T \cdot 2^S + 1 = p$ with $T$ odd and $k \leq S$
	\item Compute $N_{\texttt{perm}}$ permutation polynomials $S_{\sigma_i}(X)$ such that $S_{\sigma_i}(\omega^j) = \delta^{i'} \cdot \omega^{j'}$
	\item Compute $N_{\texttt{perm}}$ identity permutation polynomials: $S_{id_i}(X)$ such that $S_{id_i}(\omega^j) = \delta^i \cdot \omega^j$
	\item Let $H = \{\omega^0, ..., \omega^n\}$ be a cyclic subgroup of $\mathbb{F}^*$
	\item Let $Z(X) = \prod\limits{a \in H^*}(X - a)$
	\item Let $A_i$ be a witness lookup columns and $S_i$ be a table columns, $i = 0,.., m$.
\end{enumerate}
\end{algorithm}

\paragraph{Protocol (Prover):}
\begin{enumerate}
	\item Choose masking polynomials: 
	\begin{center}
		$h_i(X) \leftarrow \mathbb{F}_{<k}[X]$ for $0 \leq i < N_{\texttt{wires}}$
	\end{center}
	\textbf{Remark}: For details on choice of $k$, we refer the reader to \cite{cryptoeprint:2019:1400}.
	\item Define new witness polynomials:
	\begin{center}
		$f_i(X) = \textbf{f}_{i}(X) + h_i(X)Z(X)$ for $0 \leq i < N_{\texttt{wires}}$
	\end{center}
	\item Send commitments to $f_i$ to $\textbf{V}$
	\item Get $\theta\leftarrow \mathbb{F}$ from \textbf{V}
	\item Construct the witness lookup compression and table compression $S(\theta)$ and $A(\theta)$:
	\begin{center}
		$A(\theta) = \theta^{m - 1} A_{0} + \theta^{m - 2} A_{1} + ... + \theta A_{ m - 2} + A_{m - 1}$ //
		$S(\theta) = \theta^{m-1} S_{0} + \theta^{m-2} S_{1} + ... + \theta S_{m-2} + S_{m-1}$
	\end{center}
	\item Produce the permutation polynomials $S'(X)$ and $A'(X)$ such that:
	\begin{enumerate}
		\item All the cells of column $A'$ are arranged so that like-valued cells are vertically adjacent to each other. 
		\item The first row in a sequence of values in $A'$ is the row that has the corresponding value in $S'$.
	\end{enumerate}
	\item Compute and send commitments to $A'$ and $S'$ to \textbf{V}
	\item Get $\beta, \gamma \leftarrow \mathbb{F}$ from \textbf{V}
	\item For $0 \leq i < N_{\texttt{perm}}$
	\begin{center}
		$p_i = f_i + \beta \cdot S_{id_i} + \gamma$ \\
		$q_i = f_i + \beta \cdot S_{\sigma_i} + \gamma$
	\end{center}
	\item Define:
	\begin{center}
		$p'(X) = \prod\limits_{0 \leq i < N_{\texttt{perm}}} p_i(X) \in \mathbb{F}_{<N_{\texttt{perm}} \cdot n}[X]$ \\
		$q'(X) = \prod\limits_{0 \leq i < N_{\texttt{perm}}} q_i(X) \in \mathbb{F}_{<N_{\texttt{perm}} \cdot n}[X]$
	\end{center}
	\item Compute $P(X), Q(X) \in \mathbb{F}_{<n+1}[X]$, such that:
	\begin{center}
		$P(\omega) = Q(\omega) = 1$ \\
		$P(\omega^i) = \prod\limits_{1 \leq j < i}p'(\omega^i)$ for $i \in {2, \dots, n + 1}$ \\
		$Q(\omega^i) = \prod\limits_{1 \leq j < i}q'(\omega^i)$ for $i \in {2, \dots, n + 1}$ \\
	\end{center}
	\item Compute and send commitments to $P$ and $Q$ to \textbf{V}
	\item Compute permutation product column:
	\begin{center}
	$V (\omega^i) = \frac{(\theta^{m-1} A_0(\omega^i) + \theta^{m-2} A_1(\omega^i) + ... + \theta A_{m-2}(\omega^i) + A_{m-1}(\omega^i) + \beta) \cdot (\theta^{m-1} S_0(\omega^i) + \theta^{m-2} S_1(\omega^i) + ... + \theta S_{m-2}(\omega^i) + S_{m-1}(\omega^i) + \gamma)} {(A'(\omega^i) + \beta) (S'(\omega^i) + \gamma)}$ \\
	$V (1) = V (\omega^{N_{\texttt{lookups}}}) = 1$
	\end{center}
	\item Compute and send commitments to $V$ to \textbf{V}
	\item Get $\alpha_0, \dots, \alpha_5 \leftarrow \mathbb{F}$ from \textbf{V}
	\item Define polynomials ($F_0, \dots, F_4$ - copy-satisfability):
	\begin{center}
		$F_0(X) = L_1(X)(P(X) - 1)$\\
		$F_1(X) = L_1(X)(Q(X) - 1)$ \\
		$F_2(X) = P(X)p'(X) - P(X\omega)$ \\
		$F_3(X) = Q(X)q'(X) - Q(X\omega)$ \\
		$F_4(X) = L_n(X)(P(X\omega) - Q(X\omega))$ \\
		$F_5(X) = \sum\limits_{0 \leq i < N_{\texttt{sel}}} (\textbf{q}_{i}(X) \cdot \texttt{gate}_i(X))
			+ \sum\limits_{0 \leq i < N_{\texttt{const}}}(\textbf{f}_{c_i}(X)) + PI(X)$
	\end{center}
	\item For the lookup:
	\begin{enumerate}
		\item Two selectors $q_{last}$ and $q_{blind}$ are used, where $q_{last} = 1$ for $t$ last blinding rows and $q_{blind} = 1$ on the row in between the usable rows and the blinding rows.
		\item $F_6(X) = L_0(X) (1 - V(X))$
		\item $F_7(X) = q_{last} \cdot (V(X)^2 - V(X))$
		\item $F_8(X) = (1 - (q_{last} + q_{blind})) \cdot ( V(\omega X) (A'(X) + \beta) (S'(X) + \gamma) - V(X) (\theta^{m-1} A_0(X) + ... + A_{m-1}(X) + \beta) (\theta^{m-1} S_0(X) + ... + S_{m-1}(X) + \gamma) )$
		\item $F_9(X) = L_0(X) \cdot (A'(X) - S'(X))$
		\item $F_{10}(X) = (1 - (q_{last} + q_{blind})) \cdot (A'(X) - S'(X))\cdot(A'(X) - A'(\omega^{-1} X))$
	\end{enumerate}
	\item Compute:
	\begin{center}
		$F(X) = \sum\limits_{i = 0}^{10} \alpha_iF_i(X)$ \\
		$T(X) = \frac{F(X)}{Z(X)}$
	\end{center}
	\item Split $T(X)$ into separate polynomials $T_0(X), ..., T_{N_{\texttt{perm}}}(X)$
	\item Send commitments to $T_0(X), ..., T_{N_{\texttt{perm}}}(X)$ to \textbf{V}
	\item Get $y \leftarrow \mathbb{F}/H$ from \textbf{V}
	\item Run evaluation scheme with the committed polynomials and $y$ \\
	\textbf{Remark}: Depending on the circuit, evaluation can be done also on $y\omega, y\omega^{-1}$.
	\item Send proof $\pi$ to $\textbf{V}$
\end{enumerate}

\subsection{Non-Interactive Verification}

\begin{enumerate}
	\item Let $f_{0, \texttt{comm}}, \dots, f_{N_{\texttt{wires}}, \texttt{comm}}$ be commitments to $f_{0}(X), \dots, f_{N_{\texttt{wires}}}(X)$
	\item $\text{transcript} = \text{setup\_values} || f_{0, \texttt{comm}} || \dots || f_{N_{\texttt{wires}}, \texttt{comm}}$
	\item $\theta= H(\text{transcript})$
	\item Let $A'_{\texttt{comm}}, S'_{\texttt{comm}}$ be commitments to $A'(X), S'(X)$.
	\item $\text{transcript} = \text{transcript} || A'_{\texttt{comm}} || S'_{\texttt{comm}}$ 
	\item $\beta, \gamma = H(\text{transcript})$
	\item Let $P_{\texttt{comm}}, Q_{\texttt{comm}}, V_{i, \texttt{comm}}$ be commitments to $P(X), Q(X), V(X)$.
	\item $\text{transcript} = \text{transcript} || P_{\texttt{comm}} || Q_{\texttt{comm}} || V_{\texttt{comm}}$
	\item $\alpha_0, \dots, \alpha_5 = H(\text{transcript})$
	\item Let $T_{0, \texttt{comm}}, ..., T_{N_{\texttt{perm}, \texttt{comm}}}$ be commitments to $T_0(X), ..., T_{N_{\texttt{perm}}}(X)$ 
	\item $\text{transcript} = \text{transcript} || T_{0, \texttt{comm}} || ... || T_{N_{\texttt{perm}, \texttt{comm}}}$
	\item $y = H_{\mathbb{F}/H}(\text{transcript})$
	\item Run evaluation scheme verification with the committed polynomials and $y$ to get values 
		$f_i(y), P(y), P(y\omega), Q(y), Q(y\omega), T_j(y), A'(y), S'(y), V(y), A'(y\omega^{-1}, V(y\omega))$.  \\
		\textbf{Remark}: Depending on the circuit, evaluation can be done also on $f_i(y\omega), f_i(y\omega^{-1})$ for some $i$.
	\item Calculate:
	\begin{center}
		$F_0(y) = L_1(y)(P(y) - 1)$ \\
		$F_1(y) = L_1(y)(Q(y) - 1)$ \\
		$p'(y) = \prod p_i(y) = \prod f_i(y) + \beta \cdot S_{id_i}(y) + \gamma$ \\
		$F_2(y) = P(y)p'(y) - P(y\omega)$ \\
		$q'(y) = \prod q_i(y) = \prod f_i(y) + \beta \cdot S_{\sigma_i}(y) + \gamma$ \\
		$F_3(y) = Q(y)q'(y) - Q(y\omega)$ \\
		$F_4(y) = L_n(y)(P(y\omega) - Q(y\omega))$ \\
		$F_5(y) = \sum\limits_{0 \leq i < N_{\texttt{sel}}} (\textbf{q}_{i}(y) \cdot \texttt{gate}_i(y))
			+ \sum\limits_{0 \leq i < N_{\texttt{const}}}(\textbf{f}_{c_i}(y)) + PI(y)$ \\
		$T(y) = \sum\limits_{0 \leq j < N_{\texttt{perm} + 1}}y^{n \cdot j}T_j(y)$
		$F_6(y) = L_0(y) (1 - V(y))$ \\
		$F_7(y) = q_{last} \cdot (V(y)^2 - V(y))$ \\
		$F_8(y) = (1 - (q_{last} + q_{blind})) \cdot ( V(\omega y) (A'(y) + \beta) (S'(y) + \gamma) - V(y) (\theta^{m-1} A_{0}(y) + ... + A_{m-1}(y) + \beta) (\theta^{m-1} S_{i, 0}(y) + ... + S_{m-1}(y) + \gamma) )$ \\
		$F_9(y) = L_0(y) \cdot (A'(y) - S'(y))$ \\
		$F_{10}(y) = (1 - (q_{last} + q_{blind})) \cdot (A'(y) - S'(y))\cdot(A'(y) - A'(\omega^{-1} y))$ \\
	\end{center}
	\item Check the identity:
	\begin{center}
		$\sum\limits_{i = 0}^{10}\alpha_iF_i(y) = Z(y)T(y)$
	\end{center}
\end{enumerate}

\section{Proof System}
\label{section:proof_system}
\textbf{WIP}

The proof system used for proving Solana's 'Light-Client' state on EVM is Redshift SNARK\cite{cryptoeprint:2019:1400}.
RedShift is a transparent SNARK that uses PLONK\cite{cryptoeprint:2019:953} proof system but replaces the commitment scheme.
Initial paper proposal is to employ FRI\cite{ben2018fast} protocol to obtain transparency for the PLONK system.

However, FRI cannot be straightforwardly used with the PLONK system.
To achieve the required security level without huge overheads, the authors introduce \textit{list polynomial commitment} scheme as a part of the protocol.
For more details, the reader gets referred to \cite{cryptoeprint:2019:1400}.

The original RedShift protocol utilizes the classic PLONK\cite{cryptoeprint:2019:953} system.
To provide better performance, the original protocol is generalized to be used
with PLONK with custom gates \cite{turbo}, \cite{plonkhalo2} and lookup arguments \cite{cryptoeprint:2020:315}, \cite{lookuphalo2}.

\section{SHA256 Circuit}
\label{section:sha256}
Suppose that input data in the 32-bits form, which is already padded to the required size.
Checking that chunked input data corresponds to the original data out of this circuit.
However, we add the boolean check and range proof.
\paragraph{Range proof that $a < 2^{32}$}
Let $a = \{ a_0, ..., a_{15} \}$, where $a_i$ is two bits.
\begin{center}
\begin{tabular}{ c|c|c|c|c|c } 
  & $w_1$ & $w_2$ & $w_3$ & $w_4$ & $w_o$\\ 
 \hline
j + 0 & $a_{12}$ & $ a_{13}$ & $a_{14}$ & $a_{15}$ & acc\\ 
j + 1 & $a_8$ & $a_{9}$ & $a_{10}$ & $a_{11}$ & acc \\ 
j + 2 &$a_4$ & $a_5$ & $a_6$ & $a_7$ & acc \\
j + 3 & $a_0 $& $a_1$ & $a_2$ & $a_3$ & a \\ 
\end{tabular}
\end{center}
Range gate constraints:
\begin{center}
$w_{1,i}(w_{1,i}-1)(w_{1,i}-2)(w_{1,i} -3) + w_{2,i}(w_{2,i}-1)(w_{2,i}-2)(w_{2,i} -3) + w_{3,i}(w_{3,i}-1)(w_{3,i}-2)(w_{3,i} -3) + w_{4,i}(w_{4,i}-1)(w_{4,i}-2)(w_{4,i} -3) = 0$ \\
$w_{o,i} = w_{o, i - 1} \cdot 4^4 + w_{4,i} \cdot 4^3 + w_{3,i} \cdot 4^2 + w_{2,i} \cdot 4 + w_{1,i}$
\end{center}
The range proofs are included for each input data block. 

\paragraph{The function $\sigma_0$} contain sparse mapping subcircuit with base $4$.
Let $a$ be divided to 8 bits-chunks $a_0, a_1, a_2, a_3$.
The values $a'_0, a'_1, a'_2, a'_3$ are in sparse form, and $a'$ is a sparse $a$.
We need the following lookup tables:
\begin{enumerate}
\item \textbf{SHA-256 NORMALIZE4}: Read $a'_i$ to $a_i$
\item \textbf{SHA-256 8ROT3 32}: Read $a'_1$ to $r_1$
\item \textbf{SHA-256 8ROT2 32}: Read $a'_4$ to $r_2$
\item \textbf{SHA-256 8SHR3 32}: Read $a'_0$ to $r_3$
\end{enumerate}
\begin{center}
\begin{tabular}{ c|c|c|c|c|c } 
  & $w_1$ & $w_2$ & $w_3$ & $w_4$ & $w_o$\\ 
 \hline
j + 0 & $a_0$ & $ a_1$ & $a_2$ & $a_3$ & a\\ 
j + 1 & $a'_0$ & $a'_1$ & $a'_2$ & $a'_3$ & acc \\
j + 2 & $r1 $& $r_2$ & $r_3$ &  & $\sigma_0$ \\ 
\end{tabular}
\end{center}
Sparse map gate constraints:
\begin{center}
$w_{o,j} = w_{1,j} + w_{2,j} \cdot 2^8 + w_{3,j} \cdot 2^{8 \cdot 2} + w_{4,j} \cdot 2^{8 \cdot 3}$ \\
$w_{o,j + 1} =  w_{2,j + 1} \cdot 4^{8 - 7} + w_{3,j + 1} \cdot 4^{8 \cdot 2 - 7} + w_{4,j + 1} \cdot 4^{8 \cdot 3 - 7}
	 	+ w_{1,j + 1} \cdot 4^{8 \cdot 2 - 2} + w_{2,j + 1} \cdot 4^{8 \cdot 3 - 2}$ \\
	 $+ w_{4,j + 1} \cdot 4^{8 - 2} + w_{2,j + 1} \cdot 4^{8 - 3} + w_{3,j + 1} \cdot 4^{8 \cdot 2 - 3}
	 	+ w_{4,j + 1} \cdot 4^{8^3 - 3}$ \\
$w_{o, j+2} = w_{0, j+1} + w_{1,j+2} + w_{2, j+2} + w_{3, j+2}$ \\
7 plookup constraints \\
\end{center}

\paragraph{The function $\sigma_1$} contain sparse mapping subcircuit with base $4$.
Let $a$ be divided to 8 bits-chunks $a_0, a_1, a_2, a_3$.
The values $a'_0, a'_1, a'_2, a'_3$ are in sparse form and $a'$ is a sparse $a$.
We need the following lookup tables:
\begin{enumerate}
\item \textbf{SHA-256 NORMALIZE4}: Read $a_i$ to $a'_i$
\item \textbf{SHA-256 8ROT1 32}: Read $a'_2$ to $r_1$
\item \textbf{SHA-256 8ROT3 32}: Read $a'_2$ to $r_2$ 
\item \textbf{SHA-256 8ROT2 32}: Read $a'_1$ to $r_3$ 
\end{enumerate}
\begin{center}
\begin{tabular}{ c|c|c|c|c|c } 
  & $w_1$ & $w_2$ & $w_3$ & $w_4$ & $w_o$\\ 
 \hline
j + 0 & $a_0$ & $ a_1$ & $a_2$ & $a_3$ & a\\ 
j + 1 & $a'_0$ & $a'_1$ & $a'_2$ & $a'_3$ & acc \\ 
j + 2 & $r1 $& $r_2$ & $r_3$ &   & $\sigma_1$ \\ 
\end{tabular}
\end{center}
Sparse map gate constraints:
\begin{center}
$w_{o,j} = w_{1,j} + w_{2,j} \cdot 2^8 + w_{3,j} \cdot 2^{8 \cdot 2} + w_{4,j} \cdot 2^{8 \cdot 3}$ \\
$w_{o,j + 1} = w_{1,j + 1} \cdot 4^{8 \cdot 2 - 1} + w_{2,j + 1} \cdot 4^{8 \cdot 3 - 1} + w_{4,j + 1} \cdot 4^{8 - 1}
	 	+ w_{1,j + 1} \cdot 4^{8 \cdot 2 - 3} + w_{2,j + 1} \cdot 4^{8 \cdot 3 - 3}$\\
	 $+ w_{4,j + 1} \cdot 4^{8 - 3}
	 	+ w_{1,j + 1} \cdot 4^{8 \cdot 3 - 2} + w_{3,j + 1} \cdot 4^{8-2} + w_{4,j + 1} \cdot 4^{8^2 - 2}$ \\
$w_{o, j+2} = w_{0, j+1} + w_{1,j+2} + w_{2, j+2} + w_{3, j+2}$ \\
7 plookup constraints \\
\end{center}

The sparse values $\sigma_0$ and $\sigma_1$ have to be normalized.
The final addition requires one add gate.
We use \textbf{SHA256 NORMALIZE4}
\begin{center}
\begin{tabular}{ c|c|c|c|c|c } 
  & $w_1$ & $w_2$ & $w_3$ & $w_4$ & $w_o$\\ 
 \hline
j + 0 & $a'_0$ & $a'_1$ & $a'_2$ & $a'_3$ &\\ 
j + 1 & $a_0$ & $ a_1$ & $a_2$ & $a_3$ &  $\sigma_i$ \\ 
\end{tabular}
\end{center}

Normalize gate constraints:
\begin{center}
$w_{o,j-1} = w_{4,j} \cdot 4^{8 \cdot 3} + w_{3,j} \cdot 4^{8 \cdot 2} + w_{2,j} \cdot 4^8 + w_{1,j}$
$w_{o,j+1} = w_{4,j+1} \cdot 256^3 + w_{3,j+1} \cdot 256^2 + w_{2,j+1} \cdot 256 + w_{1,j+1}$ \\
4 plookup constraints \\
\end{center}

\paragraph{The $\Sigma_0$ function}
contain sparse mapping subcircuit with base $2$.
Let $a$ be divided to 8 bits-chunks $a_0, a_1, a_2, a_3$.
The values $a'_0, a'_1, a'_2, a'_3$ are in sparse form, and $a'$ is a sparse $a$.
We need the following lookup tables:
\begin{enumerate}
\item \textbf{SHA-256 NORMALIZE4}: Read $a_i$ to $a'_i$
\item \textbf{SHA-256 8ROT2 32}: Read $a'_0$ to $r_1$
\item \textbf{SHA-256 8ROT5 32}: Read $a'_1$ to $r_2$
\item \textbf{SHA-256 8ROT6 32}: Read $a'_2$ to $r_3$
\end{enumerate}
\begin{center}
\begin{tabular}{ c|c|c|c|c|c } 
  & $w_1$ & $w_2$ & $w_3$ & $w_4$ & $w_o$\\ 
 \hline
j + 0 & $a_0$ & $ a_1$ & $a_2$ & $a_3$ & a\\ 
j + 1 & $a'_0$ & $a'_1$ & $a'_2$ & $a'_3$ & a' \\ 
j + 2 & $r1 $& $r_2$ & $r_3$ &    & $\Sigma_0$ \\ 
\end{tabular}
\end{center}
Sparse map gate constraints:
\begin{center}
$w_{o,j} = w_{1,j} + w_{2,j} \cdot 2^8 + w_{3,j} \cdot 2^{8 \cdot 2} + w_{4,j} \cdot 2^{8 \cdot 3}$ \\
$w_{o,j+1} = w_{1,j+1} + w_{2,j+1} \cdot 4^8 + w_{3,j+1} \cdot 4^{8 \cdot 2} + w_{4,j+1} \cdot 4^{8 \cdot 3}$ \\
$w_{o,j+2} = w_{2,j+1} \cdot 4^{8-2} + w_{3,j+1} \cdot 4^{8 \cdot 2-2} + w_{4,j+1} \cdot 4^{8 \cdot 3 - 2}
	 + w_{1,j+1} \cdot 4^{8 \cdot 3 - 5} + w_{3,j+1} \cdot 4^{8-5} + w_{4,j+1} \cdot 4^{8 \cdot 2 - 5} 
	 + w_{1,j+1} \cdot 4^{8 \cdot 2-6} + w_{2,j+1} \cdot 4^{8 \cdot 3-6} + w_{4,j+1} \cdot 4^{8 - 6} + w_{1,j+2} + w_{2, j+2} + w_{3, j+2}$ \\
7 plookup constraints \\
\end{center}

\paragraph{The $\Sigma_1$ function}
contain sparse mapping subcircuit with base $2$.
Let $a$ be divided to 8 bits-chunks $a_0, a_1, a_2, a_3$.
The values $a'_0, a'_1, a'_2, a'_3$ are in sparse form, and $a'$ is a sparse $a$.
We need the following lookup tables:
\begin{enumerate}
\item \textbf{SHA-256 NORMALIZE7}: Read $a_i$ to $a'_i$
\item \textbf{SHA-256 8ROT6 32}: Read $a'_0$ to $r_1$
\item \textbf{SHA-256 8ROT3 32}: Read $a'_1$ to $r_2$
\item \textbf{SHA-256 8ROT1 32}: Read $a'_3$ to $r_3$
\end{enumerate}
\begin{center}
\begin{tabular}{ c|c|c|c|c|c } 
  & $w_1$ & $w_2$ & $w_3$ & $w_4$ & $w_o$\\ 
 \hline
j + 0 & $a_0$ & $ a_1$ & $a_2$ & $a_3$ & a\\ 
j + 1 & $a'_0$ & $a'_1$ & $a'_2$ & $a'_3$ & a' \\ 
j + 2 & $r1 $& $r_2$ & $r_3$ &   & $\Sigma_1$ \\ 
\end{tabular}
\end{center}
Sparse map gate constraints:
\begin{center}
$w_{o,j} = w_{1,j} + w_{2,j} \cdot 2^8 + w_{3,j} \cdot 2^{8 \cdot 2} + w_{4,j} \cdot 2^{8 \cdot 3}$ \\
$w_{o,j+1} = w_{1,j+1} + w_{2,j+1} \cdot 7^8 + w_{3,j+1} \cdot 7^{8 \cdot 2} + w_{4,j+1} \cdot 7^{8 \cdot 3}$ \\
$w_{o,j+2} =  w_{2,j+1} \cdot 7^{8-6} + w_{3,j+1} \cdot 7^{8 \cdot 2 - 6} + w_{7,j+1} \cdot 4^{8 \cdot 3 - 6}
	+ w_{1,j+1} \cdot 7^{8 \cdot 3 - 3} + w_{3,j+1} \cdot 7^{8-3} + w_{4,j+1} \cdot 7^{8 \cdot 2 - 3}+ w_{1,j+1} \cdot 7^{8-1}
	+ w_{2,j+1} \cdot 7^{8 \cdot 2-1} + w_{3,j+1} \cdot 7^{8 \cdot 3 - 1}+ w_{1,j+2} + w_{2, j+2} + w_{3, j+2}$ \\
7 plookup constraints \\
\end{center}

The sparse values $\Sigma_0$ and $\Sigma_1$ have to be normalized.
We use \textbf{SHA256 NORMALIZE4} and \textbf{SHA256 NORMALIZE7}.
\begin{center}
\begin{tabular}{ c|c|c|c|c|c } 
  & $w_1$ & $w_2$ & $w_3$ & $w_4$ & $w_o$\\ 
 \hline
j + 0 & $a'_0$ & $a'_1$ & $a'_2$ & $a'_3$ &\\ 
j + 1 & $a_0$ & $ a_1$ & $a_2$ & $a_3$ &  $\Sigma_i$ \\ 
\end{tabular}
\end{center}

Normalize gate constraints:
\begin{center}
$w_{o,j-1} = w_{4,j} \cdot 4^8 \cdot 3 + w_{3,j} \cdot 4^8 \cdot 2 + w_{2,j} \cdot 4^8 + w_{1,j}$ for $\Sigma_1$ replace 4 with 7\\
$w_{o,i} = w_{4,i} \cdot 256^3 + w_{3,i} \cdot 256^2 + w_{2,i} \cdot 256 + w_{1,i}$ \\
7 plookup constraints \\
\end{center}

\paragraph{The Maj function}
contain sparse mapping subcircuit with base $2$ for $a, b ,c$.
Let $a; b; c$ be divided to 8 bits-chunks $a_0, a_1, a_2, a_3; b_0, b_1, b_2, b_3; c_0, c_1, c_2, c_3$.
The values $a'_0, a'_1, a'_2, a'_3$ are in sparse form, and $a'$ is a sparse $a$.
Similarly for b and c. 
Note, that $a$ we already have in the sparse from $\Sigma_0$ in the circuit. 
The variables $b$ and $c$ were represented in sparse form in the previous rounds or it is public inputs.
\begin{center}
\begin{tabular}{ c|c|c|c|c|c } 
  & $w_1$ & $w_2$ & $w_3$ & $w_4$ & $w_o$\\ 
 \hline 
j - k & $a'_0$ & $a'_1$ & $a'_2$ & $a'_3$ & a' \\ 
... & & & & \\
j - l & $b'_0$ & $b'_1$ & $b'_2$ & $b'_3$ & b'\\ 
... & & & & \\
j - t & $c'_0$ & $c'_1$ & $c'_2$ & $c'_3$ & c' \\ 
... & & & & \\
j + 0 & a' & b' & c' & & maj \\
\end{tabular}
\end{center}
Sparse map gate constraints:
\begin{center}
$w_{o, j} = w_{1,j} + w_{2, j} + w_{3, j}$ \\
\end{center}

The sparse values $maj$ have to be normalized.
We use \textbf{SHA256 MAJ NORMALIZE4}
\begin{center}
\begin{tabular}{ c|c|c|c|c|c } 
  & $w_1$ & $w_2$ & $w_3$ & $w_4$ & $w_o$\\ 
 \hline
j + 0 & $a'_0$ & $a'_1$ & $a'_2$ & $a'_3$ &\\ 
j + 1 & $a_0$ & $ a_1$ & $a_2$ & $a_3$ &  $maj$ \\ 
\end{tabular}
\end{center}

Normalize gate constraints:
\begin{center}
$w_{o,i} = w_{4,i} \cdot 256^3 + w_{3,i} \cdot 256^2 + w_{2,i} \cdot 256 + w_{1,i}$ \\
\end{center}

The final addition requires one add gate.

\paragraph{The Ch function}
contain sparse mapping subcircuit with base $2$ for $e, f ,g$.
Let $e; f; g$ be divided to 8 bits-chunks $e_0, e_1, e_2, e_3; f_0, f_1, f_2, f_3; g_0, g_1, g_2, g_3$.
The values $e'_0, e'_1, e'_2, e'_3$ are in sparse form, and $e'$ is a sparse $e$.
Similarly for b and c. 
Note, that $e$ we already have in the sparse from $\Sigma_1$ in the circuit. 
The variables $f$ and $g$ were represented in sparse form in the previous rounds or it is public inputs.
\begin{center}
\begin{tabular}{ c|c|c|c|c|c } 
  & $w_1$ & $w_2$ & $w_3$ & $w_4$ & $w_o$\\ 
 \hline 
j - k & $a'_0$ & $a'_1$ & $a'_2$ & $a'_3$ & a' \\ 
... & & & & \\
j - l & $b'_0$ & $b'_1$ & $b'_2$ & $b'_3$ & b'\\ 
... & & & & \\
j - t & $c'_0$ & $c'_1$ & $c'_2$ & $c'_3$ & c' \\ 
... & & & & \\
j + 0 & a' & b' & c' & & ch\\
\end{tabular}
\end{center}
Sparse map gate constraints:
\begin{center}
$w_{o, j} = w_{1,j} + 2*w_{2, j} + 3*w_{3, j}$ \\
\end{center}

The sparse values $ch$ have to be normalized.
We use \textbf{SHA256 CH NORMALIZE7}
\begin{center}
\begin{tabular}{ c|c|c|c|c|c } 
  & $w_1$ & $w_2$ & $w_3$ & $w_4$ & $w_o$\\ 
 \hline
j + 0 & $a'_0$ & $a'_1$ & $a'_2$ & $a'_3$ &\\ 
j + 1 & $a_0$ & $ a_1$ & $a_2$ & $a_3$ &  $ch$ \\ 
\end{tabular}
\end{center}

Normalize gate constraints:
\begin{center}
$w_{o,i} = w_{4,i} \cdot 256^3 + w_{3,i} \cdot 256^2 + w_{2,i} \cdot 256 + w_{1,i}$ \\
\end{center}

The final addition requires one add gate.

The updating of variables for new rounds costs 10 add gates.

Producing the final hash value costs two add gates.

\subsection{SHA-512}\label{sha512}
SHA-512 uses the similar logical functions as in \ref{section:sha256} which operates on $64$-bits words.
Thus each input uses the same range proof which extended to 64-bits.

\paragraph{Range proof that $a < 2^{64}$}
Let $a = \{ a_0, ..., a_{32} \}$, where $a_i$ is two bits.
\begin{center}
\begin{tabular}{ c|c|c|c|c|c } 
  & $w_1$ & $w_2$ & $w_3$ & $w_4$ & $w_o$\\ 
 \hline
j + 0 & $a_{29}$ & $ a_{30}$ & $a_{31}$ & $a_{32}$ & acc\\ 
j + 1 & $a_{25}$ & $ a_{26}$ & $a_{27}$ & $a_{28}$ & acc\\ 
... & & & & & \\
j + 6 &$a_4$ & $a_5$ & $a_6$ & $a_7$ & acc \\
j + 7 & $a_0 $& $a_1$ & $a_2$ & $a_3$ & a \\ 
\end{tabular}
\end{center}
Range gate constraints:
\begin{center}
$w_{1,i}(w_{1,i}-1)(w_{1,i}-2)(w_{1,i} -3) + w_{2,i}(w_{2,i}-1)(w_{2,i}-2)(w_{2,i} -3)
	+ w_{3,i}(w_{3,i}-1)(w_{3,i}-2)(w_{3,i} -3) + w_{4,i}(w_{4,i}-1)(w_{4,i}-2)(w_{4,i} -3)$ \\
$w_{o,i} = w_{o, i - 1} \cdot 4^4 + w_{4,i} \cdot 4^3 + w_{3,i} \cdot 4^2 + w_{2,i} \cdot 4 + w_{1,i}$
\end{center}
The range proofs are included for each input data block. 

\paragraph{The function $\sigma_0$} contain sparse mapping subcircuit with base $4$.
Let $a$ be divided to 8 bits-chunks $a_0, a_1, a_2, ..., a_7$.
The values $a'_0, a'_1, a'_2,...,  a'_7$ are in sparse form, and $a'$ is a sparse $a$.
We need the following lookup tables:
\begin{enumerate}
\item \textbf{SHA-256 NORMALIZE4}: Read $a_i$ to $a'_i$
\item \textbf{SHA-512 8ROT1 64}: Read $a'_0$ to $r_1$
\item \textbf{SHA-512 8SHR7 64}: Read $a'_0$ to $r_3$
\end{enumerate}
\begin{center}
\begin{tabular}{ c|c|c|c|c|c } 
  & $w_1$ & $w_2$ & $w_3$ & $w_4$ & $w_o$\\ 
 \hline
j + 0 & $a_0$ & $ a_1$ & $a_2$ & $a_3$ & $a_4$\\ 
j + 1 & $a'_0$ & $a'_1$ & $a'_2$ & $a'_3$ & a \\
j + 2 & $a_5 $& $a_6$ & $a_7$ & $a_4'$ & $\sigma_0$ \\ 
j + 3 & $a'_5$ & $ a'_6$ & $a'_7$ & $r_1$ & $r_2$\\ 
\end{tabular}
\end{center}
Sparse map gate constraints:
\begin{center}
$w_{o,j+1} = w_{1,j} + w_{2,j} \cdot 2^8 + w_{3,j} \cdot 2^{8 \cdot 2} + w_{4,j} \cdot 2^{8 \cdot 3}
	+ w_{o,j} \cdot 2^{8 \cdot 4} + w_{1,j+2} \cdot 2^{8 \cdot 5} + w_{2,j+2} \cdot 2^{8 \cdot 6} + w_{3,j+2} \cdot 2^{8 \cdot 7}$ \\
$w_{o,j+2} =  w_{2,j+1} \cdot 4^{8-1} + w_{3,j+1} \cdot 4^{8 \cdot 2-1} + w_{4,j+1} \cdot 4^{8 \cdot 3 - 1}
	+ w_{4,j+2} \cdot 4^{8 \cdot 4 - 1} + w_{1,j+3} \cdot 4^{8 \cdot 5 - 1} + w_{2,j+3} \cdot 4^{8 \cdot 6 - 1}
	+ w_{3,j+3} \cdot 4^{8 \cdot 7 - 1} + w_{1,j+1} \cdot 4^{8 \cdot 7} + w_{2,j+1} + w_{3,j+1} \cdot 4^{8}
	+ w_{4,j+1} \cdot 4^{8 \cdot 2} +w_{4,j+2} \cdot 4^{8 \cdot 3} + w_{1,j+3} \cdot 4^{8 \cdot 4}
	+ w_{2,j+3} \cdot 4^{8 \cdot 5} + w_{3,j+3} \cdot 4^{8 \cdot 6} + w_{2,j+1} \cdot 4^{8-7}
	+ w_{3,j+1} \cdot 4^{8 \cdot 2-7} + w_{4,j+1} \cdot 4^{8 \cdot 3 - 7} + w_{4,j+2} \cdot 4^{8 \cdot 4 - 7}
	+ w_{1,j+3} \cdot 4^{8*5 - 7} + w_{2,j+3} \cdot 4^{8 \cdot 6 - 7} + w_{3,j+3} \cdot 4^{8 \cdot 7 - 7}
	+ w_{4, j+3} + w_{o, j+3}$ \\
10 plookup constraints \\
\end{center}

\paragraph{The function $\sigma_1$} contain sparse mapping subcircuit with base $4$.
Let $a$ be divided to 8 bits-chunks $a_0, a_1, a_2, ..., a_7$.
The values $a'_0, a'_1, a'_2,...,  a'_7$ are in sparse form, and $a'$ is a sparse $a$.
We need the following lookup tables:
\begin{enumerate}
\item \textbf{SHA-256 NORMALIZE4}: Read $a_i$ to $a'_i$
\item \textbf{SHA-512 8ROT3 64}: Read $a'_2$ to $r_1$
\item \textbf{SHA-512 8ROT5 SHR6 64}: Read $a'_7 + a'_0$ to $r_2$
\end{enumerate}
\begin{center}
\begin{tabular}{ c|c|c|c|c|c } 
  & $w_1$ & $w_2$ & $w_3$ & $w_4$ & $w_o$\\ 
 \hline
j + 0 & $a_0$ & $ a_1$ & $a_2$ & $a_3$ & $a_4$\\ 
j + 1 & $a'_0$ & $a'_1$ & $a'_2$ & $a'_3$ & a \\
j + 2 & $a_5 $& $a_6$ & $a_7$ & $a_4'$ & $\sigma_1$ \\ 
j + 3 & $a'_5$ & $ a'_6$ & $a'_7$ & $r_1$ & $r_2$\\
\end{tabular}
\end{center}
Sparse map gate constraints:
\begin{center}
$w_{o,j+1} = w_{1,j} + w_{2,j} \cdot 2^8 + w_{3,j} \cdot 2^{8 \cdot 2} + w_{4,j} \cdot 2^{8 \cdot 3}
	+ w_{o,j} \cdot 2^{8 \cdot 4} + w_{1,j+2} \cdot 2^{8 \cdot 5} + w_{2,j+2} \cdot 2^{8 \cdot 6} + w_{3,j+2} \cdot 2^{8 \cdot 7}$ \\
$w_{o,j+2} =  w_{1,j+1} \cdot 4^{64 -19} + w_{2,j+1} \cdot 4^{64 + (8-19)}
	+ w_{4,j+1} \cdot 4^{8 \cdot 3 - 19} + w_{4,j+2} \cdot 4^{8 \cdot 4 - 19}
	+ w_{1,j+3} \cdot 4^{8 \cdot 5 - 19} + w_{2,j+3} \cdot 4^{8 \cdot 6 - 19}
	+ w_{3,j+3} \cdot 4^{8 \cdot 7 -19} + w_{1,j+1} \cdot 4^{64  - 61)} + w_{2,j+1} \cdot 4^{64 + (8 - 61)}
	+ w_{3,j+1} \cdot 4^{64 + (8 \cdot 2 - 61)} + w_{4,j+1} \cdot 4^{64 + (8 \cdot 3- 61)}
	+ w_{4,j+2} \cdot 4^{64 + (8 \cdot 4 - 61)} + w_{1,j+3} \cdot 4^{64 + (8 \cdot 5 - 61)}
	+ w_{2,j+3} \cdot 4^{64 +(8 \cdot 6- 61)} + w_{2,j+1} \cdot 4^{8-6} + w_{3,j+1} \cdot 4^{8 \cdot 2-6}
	+ w_{4,j+1} \cdot 4^{8 \cdot 3 - 6} + w_{4,j+2} \cdot 4^{8 \cdot 4 - 6} + w_{1,j+3} \cdot 4^{8 \cdot 5 - 6}
	+ w_{2,j+3} \cdot 4^{8 \cdot 6 - 6} + w_{3,j+3} \cdot 4^{8 \cdot 7 - 6} + w_{4, j+3} + w_{o, j+3}$ \\
10 plookup constraints \\
\end{center}

The sparse values $\sigma_0$ and $\sigma_1$ have to be normalized.
The final addition requires one add gate.
Note, that $a'$ already initialized in the row $j - 2$.
We use \textbf{SHA256 NORMALIZE4}
\begin{center}
\begin{tabular}{ c|c|c|c|c|c } 
  & $w_1$ & $w_2$ & $w_3$ & $w_4$ & $w_o$\\ 
 \hline
j + 0 & $a'_0$ & $a'_1$ & $a'_2$ & $a'_3$ &  $acc$\\ 
j + 1 & $a_0$ & $ a_1$ & $a_2$ & $a_3$ & 0\\
j + 2 & $a'_4$ & $a'_5$ & $a'_6$ & $a'_7$ & $\sigma_i$\\ 
j + 3 & $a_4$ & $ a_5$ & $a_6$ & $a_7$ &  \\
\end{tabular}
\end{center}

Normalize gate constraints:
\begin{center}
$w_{o,j+2} = w_{4,j+1} \cdot 256^3 + w_{3,j+1} \cdot 256^2 + w_{2,j+1} \cdot 256 + w_{1,j+1} 
	+ w_{1,j+3} \cdot 256^4$ \\
	$+ w_{2,j+3} \cdot 256^5+ w_{3,j+3} \cdot 256^6 + w_{4,j+4} \cdot 256^7$ \\
$w_{o,j} = w_{o, j - 2} - (w_{4,j} \cdot 256^3 + w_{3,j} \cdot 256^2 + w_{2,j} \cdot 256 + w_{1,j})$ \\
$w_{o,j+1} = w_{o,j} - ( w_{1,j+3} \cdot 256^4 + w_{2,j+3} \cdot 256^5+ w_{3,j+3} \cdot 256^6 + w_{4,j+4} \cdot 256^7)$

8 plookup constraints \\
\end{center}

\paragraph{The $\Sigma_0$ function}
contain sparse mapping subcircuit with base $4$.
Let $a$ be divided to 7-bits chunks $a_0, a_1, a_2,a_3$ and 9 bits-chunks $ a_4, a_5, a_6, a_7$.
The values $a'_0, a'_1, a'_2,...,  a'_7$ are in sparse form, and $a'$ is a sparse $a$.
We need the following lookup tables:
\begin{enumerate}
\item \textbf{SHA-512 9NORMALIZE4}: Read $a_i$ to $a'_i$
\item \textbf{SHA-512 7NORMALIZE4}: Read $a_i$ to $a'_i$
\item \textbf{SHA-512 9ROT6 64}: Read $a'_4$ to $r_2$
\item \textbf{SHA-512 9ROT2 64}: Read $a'_5$ to $r_3$
\end{enumerate}
\begin{center}
\begin{tabular}{ c|c|c|c|c|c } 
  & $w_1$ & $w_2$ & $w_3$ & $w_4$ & $w_o$\\ 
 \hline
j + 0 & $a_0$ & $ a_1$ & $a_2$ & $a_3$ & $a_4$\\ 
j + 1 & $a'_0$ & $a'_1$ & $a'_2$ & $a'_3$ & a \\
j + 2 & $a_5 $& $a_6$ & $a_7$ & $a_4'$ & $\Sigma_0$ \\ 
j + 3 & $a'_5$ & $ a'_6$ & $a'_7$ & $r_1$ & $r_2$\\
\end{tabular}
\end{center}

Sparse map gate constraints:
\begin{center}
$w_{o,j+1} = w_{1,j} + w_{2,j} \cdot 2^7 + w_{3,j} \cdot 2^{7 \cdot 2} + w_{4,j} \cdot 2^{7 \cdot 3}
	+ w_{o,j} \cdot 2^{7 \cdot 4} + w_{1,j+2} \cdot 2^{7 \cdot 4 + 9}
	+ w_{2,j+2} \cdot 2^{7 \cdot 4 + 9 \cdot 2} + w_{3,j+2} \cdot 2^{7 \cdot 4 + 9 \cdot 3}$ \\
$w_{o,j+2} =  w_{4,j+2} + w_{1,j+3} \cdot 4^9 + w_{2,j+3} \cdot 4^{9 \cdot 2}
	+ w_{3,j+3} \cdot 4^{9 \cdot 3} + w_{1,j+1} \cdot 4^{9 \cdot 4} + w_{2,j+1} \cdot 4^{9 \cdot 4 +7}$ \\
	$+ w_{3,j+1} \cdot 4^{9 \cdot 4 +7 \cdot 2} + w_{4,j+1} \cdot 4^{9 \cdot 4 +7 \cdot 3}
	+	w_{1,j+1} \cdot 4^{64  - 34)} + w_{2,j+1} \cdot 4^{64 + (7 - 34)}
	+ w_{3,j+1} \cdot 4^{64 + (7 \cdot 2 - 34)} + w_{4,j+1} \cdot 4^{64 + (7 \cdot 3- 34)}
	+ w_{1,j+3} \cdot 4^{7 \cdot 4 + 9 - 34} + w_{2,j+3} \cdot 4^{7 \cdot 4 + 9  \cdot 2  -34}
	+ w_{3,j+3} \cdot 4^{7 \cdot 4 + 9 \cdot 3 - 34} + w_{1,j+1} \cdot 4^{64  - 39)}
	+ w_{2,j+1} \cdot 4^{64 + (7 - 39)} + w_{3,j+1} \cdot 4^{64 + (7 \cdot 2 - 39)}
	+ w_{4,j+1} \cdot 4^{64 + (7 \cdot 3- 39)} +w_{4,j+2} \cdot 4^{64 + (7 \cdot 4 - 39)}
	+ w_{2,j+3} \cdot 4^{7 \cdot 4 + 9 \cdot 2 - 39}
	+ w_{3,j+3} \cdot 4^{7 \cdot 4 + 9 \cdot 3- 39} + w_{4, j+3} + w_{o, j+3}$ \\
10 plookup constraints \\
\end{center}

\paragraph{The $\Sigma_1$ function}
contain sparse mapping subcircuit with base $7$.
Let $a$ be divided to 7-bits chunks $a_0, a_1, a_2,a_3$ and 9 bits-chunks $ a_4, a_5, a_6, a_7$.
The values $a'_0, a'_1, a'_2,...,  a'_7$ are in sparse form, and $a'$ is a sparse $a$.
We need the following lookup tables:
\begin{enumerate}
\item \textbf{SHA-512 9NORMALIZE7}: Read $a_i$ to $a'_i$
\item \textbf{SHA-512 7NORMALIZE7}: Read $a_i$ to $a'_i$
\item \textbf{SHA-512 7ROT4 32}: Read $a'_2$ to $r_2$
\item \textbf{SHA-512 9ROT4 32}: Read $a'_5$ to $r_3$
\end{enumerate}
\begin{center}
\begin{tabular}{ c|c|c|c|c|c } 
  & $w_1$ & $w_2$ & $w_3$ & $w_4$ & $w_o$\\ 
 \hline
j + 0 & $a_0$ & $ a_1$ & $a_2$ & $a_3$ & $a_4$\\ 
j + 1 & $a'_0$ & $a'_1$ & $a'_2$ & $a'_3$ & a \\
j + 2 & $a_5 $& $a_6$ & $a_7$ & $a_4'$& $\Sigma_1$ \\ 
j + 3 & $a'_5$ & $ a'_6$ & $a'_7$ & $r_1$ & $r_2$\\
\end{tabular}
\end{center}

Sparse map gate constraints:
\begin{center}
$w_{o,j+1} = w_{1,j} + w_{2,j} \cdot 2^7 + w_{3,j} \cdot 2^{7 \cdot 2} + w_{4,j} \cdot 2^{7 \cdot 3}
	+ w_{o,j} \cdot 2^{7 \cdot 4} + w_{1,j+2} \cdot 2^{7 \cdot 4 + 9}
	+ w_{2,j+2} \cdot 2^{7 \cdot 4 + 9 \cdot 2} + w_{3,j+2} \cdot 2^{7 \cdot 4 + 9 \cdot 3}$ \\
$w_{o,j+2} =  w_{3,j+1} + w_{4,j+1} \cdot 7^7 + w_{4,j+2} \cdot 7^{7 \cdot 2}
	+ w_{1,j+3} \cdot 7^{7 \cdot 2+9} + w_{2,j+3} \cdot 7^{7 \cdot 2+9 \cdot 2}
	+ w_{3,j+3} \cdot 7^{9 \cdot 3 +7 \cdot 2} + w_{1,j+1} \cdot 7^{9 \cdot 4 +7 \cdot 2}
	+ w_{2,j+1} \cdot 7^{9 \cdot 4 +7 \cdot 3} + w_{1,j+1} \cdot 7^{64  - 18)}
	+ w_{2,j+1} \cdot 7^{64 + (7 - 18)} + w_{4,j+1} \cdot 7^{7 \cdot 3 - 18}
	+ w_{4,j+2} \cdot 7^{7 \cdot 4- 18}
	+ w_{1,j+3} \cdot 7^{7 \cdot 4 + 9  - 18}
	+ w_{2,j+3} \cdot 7^{7 \cdot 4 + 9 \cdot 2 - 18}
	+ w_{3,j+3} \cdot 7^{7 \cdot 4 + 9 \cdot 3 - 18} + w_{1,j+1} \cdot 7^{64  - 41)}
	+ w_{2,j+1} \cdot 7^{64 + (7 - 41)} + w_{3,j+1} \cdot 7^{64 + (7 \cdot 2 - 41)}
	+ w_{4,j+1} \cdot 7^{64 + (7 \cdot 3- 41)} +w_{4,j+2} \cdot 7^{64 + (7 \cdot 3 + 9 - 41)}
	+ w_{2,j+3} \cdot 7^{64 + (7 \cdot 3 + 9 \cdot 2 -41)}
	+ w_{3,j+3} \cdot 7^{7 \cdot 3 + 9 \cdot 3- 41} + w_{4, j+3} + w_{o, j+3}$ \\
10 plookup constraints \\
\end{center}

The sparse values $\Sigma_0$ and $\Sigma_1$ have to be normalized.
We use \textbf{SHA256 NORMALIZE4} and \textbf{SHA256 NORMALIZE7}.
Note, that $a'$ already initialized in the row $j - 2$.
\begin{center}
\begin{tabular}{ c|c|c|c|c|c } 
  & $w_1$ & $w_2$ & $w_3$ & $w_4$ & $w_o$\\ 
 \hline
j + 0 & $a'_0$ & $a'_1$ & $a'_2$ & $a'_3$ &  $a''$\\ 
j + 1 & $a_0$ & $ a_1$ & $a_2$ & $a_3$ & $0$ \\
j + 2 & $a'_4$ & $a'_5$ & $a'_6$ & $a'_7$ & $\Sigma_i$\\ 
j + 3 & $a_4$ & $ a_5$ & $a_6$ & $a_7$ &  \\
\end{tabular}
\end{center}

Normalize gate constraints:
\begin{center}
$w_{o,j+2} = w_{4,j+1} \cdot 256^3 + w_{3,j+1} \cdot 256^2 + w_{2,j+1} \cdot 256
	+ w_{1,j+1} +  w_{1,j+3} \cdot 256^4$ \\ 
	$+ w_{2,j+3} \cdot 256^5+ w_{3,j+3} \cdot 256^6 + w_{4,j+4} \cdot 256^7$ \\
$w_{o,j} = w_{1,j-3} + w_{2,j-3} \cdot 4^7 + w_{3,j-3} \cdot 4^{7 \cdot 2}
	+ w_{4,j-3} \cdot 4^{7 \cdot 3} + w_{4,j-2} \cdot 4^{7 \cdot 4}
	+ w_{1,j-1} \cdot 7^{7 \cdot 4+9}$ \\
	$+ w_{2,j-1} \cdot 7^{7 \cdot 4 + 9 \cdot 2}
	+ w_{2,j-1} \cdot 7^{7 \cdot 4 + 9 \cdot 3}$ for maj or ch function. For $\Sigma_1$ replace 4 with 7\\
$w_{o,j+1} = w_{o, j - 2} - (w_{4,j} \cdot 256^3 + w_{3,j} \cdot 256^2 + w_{2,j} \cdot 256 + w_{1,j} + w_{1,j+3} \cdot 256^4 + w_{2,j+3} \cdot 256^5+ w_{3,j+3} \cdot 256^6 + w_{4,j+4} \cdot 256^7)$

8 plookup constraints \\
\end{center}

\paragraph{The Maj function}
contain sparse mapping subcircuit with base $4$ for $a, b ,c$.
Note, that the sparse chunks of $a$ we already have in $\Sigma_0$ in the circuit. 
The variables $b$ and $c$ were represented in sparse chunks in the previous rounds or it is public inputs.
\begin{center}
\begin{tabular}{ c|c|c|c|c|c } 
  & $w_1$ & $w_2$ & $w_3$ & $w_4$ & $w_o$\\ 
 \hline 
j  & $a'$ & $b'$ & $c'$ &  & maj\\ 
\end{tabular}
\end{center}
Sparse map gate constraints:
\begin{center}
$w_{o, j} = w_{1,j} + w_{2, j} + w_{3, j}$ \\
\end{center}

The sparse values $maj$ have to be normalized.
We use \textbf{SHA256 MAJ NORMALIZE4}
Note, that the sparse $maj$ already initialized in the row $j - 1$.
\begin{center}
\begin{tabular}{ c|c|c|c|c|c} 
  & $w_1$ & $w_2$ & $w_3$ & $w_4$ & $w_o$\\ 
 \hline
j + 0 & $a'_0$ & $a'_1$ & $a'_2$ & $a'_3$ &  $acc$\\ 
j + 1 & $a_0$ & $ a_1$ & $a_2$ & $a_3$ & 0\\
j + 2 & $a'_4$ & $a'_5$ & $a'_6$ & $a'_7$ & $maj$ \\ 
j + 3 & $a_4$ & $ a_5$ & $a_6$ & $a_7$ &  \\
\end{tabular}
\end{center}

Normalize gate constraints:
\begin{center}
$w_{o,j+2} = w_{4,j+1} \cdot 256^3 + w_{3,j+1} \cdot 256^2 + w_{2,j+1} \cdot 256 + w_{1,j+1}
	+  w_{1,j+3} \cdot 256^4$ \\ 
	$+ w_{2,j+3} \cdot 256^5+ w_{3,j+3} \cdot 256^6 + w_{4,j+4} \cdot 256^7$ \\
$w_{o,j} = w_{o, j - 1} - (w_{4,j} \cdot 256^3 + w_{3,j} \cdot 256^2 + w_{2,j} \cdot 256 + w_{1,j})$ \\
$w_{o,j+1} = w_{o,j} - ( w_{1,j+3} \cdot 256^4 + w_{2,j+3} \cdot 256^5+ w_{3,j+3} \cdot 256^6 + w_{4,j+4} \cdot 256^7)$

8 plookup constraints \\
\end{center}


The final addition requires one add gate.

\paragraph{The Ch function}
contain sparse mapping subcircuit with base $7$ for $e, f ,g$.
Note, that $e$ we already have in the sparse from $\Sigma_1$ in the circuit. 
The variables $f$ and $g$ were represented in sparse form in the previous rounds or it is public inputs.
\begin{center}
\begin{tabular}{ c|c|c|c|c|c } 
  & $w_1$ & $w_2$ & $w_3$ & $w_4$ & $w_o$\\ 
 \hline 
j + 0 & $e'$ & $f'$ & $g'$ & & ch\\
\end{tabular}
\end{center}
Sparse map gate constraints:
\begin{center}
$w_{o, j} = w_{1,j} + 2 \cdot w_{2, j} + 3 \cdot w_{3, j}$ \\
\end{center}

The sparse values $ch$ have to be normalized.
Note, that $ch$ already initialized in the row $j - 1$.
We use \textbf{SHA256 CH NORMALIZE7}
\begin{center}
\begin{tabular}{ c|c|c|c|c|c } 
  & $w_1$ & $w_2$ & $w_3$ & $w_4$ & $w_o$\\ 
 \hline
j + 0 & $a'_0$ & $a'_1$ & $a'_2$ & $a'_3$ &  $acc$\\ 
j + 1 & $a_0$ & $ a_1$ & $a_2$ & $a_3$ & 0\\
j + 2 & $a'_4$ & $a'_5$ & $a'_6$ & $a'_7$ & $ch$ \\ 
j + 3 & $a_4$ & $ a_5$ & $a_6$ & $a_7$ &  \\
\end{tabular}
\end{center}

Normalize gate constraints:
\begin{center}
$w_{o,j+2} = w_{4,j+1} \cdot 256^3 + w_{3,j+1} \cdot 256^2 + w_{2,j+1} \cdot 256 + w_{1,j+1}
	+  w_{1,j+3} \cdot 256^4 + w_{2,j+3} \cdot 256^5$ \\
	$+ w_{3,j+3} \cdot 256^6 + w_{4,j+4} \cdot 256^7$ \\
$w_{o,j} = w_{o, j - 1} - (w_{4,j} \cdot 256^3 + w_{3,j} \cdot 256^2 + w_{2,j} \cdot 256 + w_{1,j})$ \\
$w_{o,j+1} = w_{o,j} - ( w_{1,j+3} \cdot 256^4 + w_{2,j+3} \cdot 256^5+ w_{3,j+3} \cdot 256^6 + w_{4,j+4} \cdot 256^7)$ \\
8 plookup constraints \\
\end{center}

The final addition requires one add gate.

The updating of variables for new rounds costs 10 add gates.

Producing the final hash value costs two add gates.

\subsection{Poseidon Circuit}
\label{section:poseidon}
Consider a poseidon permutation $F: [0_{\mathbb{F}}, I[2], I[3]] \rightarrow [O[1], H, O[3]]$ of width $3$ and $\alpha = 5$.
The 1-call sponge function is used:
\begin{center}
    \begin{tabular}{ c|c|c|c|c|c|c|c|c|c }
        & $w_0$  & $w_1$  & $w_2$  & $w_3$  & $w_4$  & $w_5$ & $w_6$ & $w_7$ & $w_8$  \\
        \hline
        j + 0  & $0_{\mathbb{F}}$ & $I[2]$ & $I[3]$ & $T_{1,0}$ & $T_{1,1}$ & $T_{1,2}$ & $T_{2,0}$ & $T_{2,1}$ & $T_{2,2}$ \\
        j + 1   & $T_{3,0}$ & $T_{3,1}$ & $T_{3,2}$ & $T_{4,0}$ & $T_{4,1}$ & $T_{4,2}$ & $T_{5,0}$ & $T_{5,1}$ & $T_{5,2}$\\
        ...    &                  &           &           &           &           \\
        j + 21 & $T_{63,0}$ & $T_{63,1}$ & $T_{63,2}$ & $T_{64,0}$ & $T_{64,1}$ & $T_{64,2}$ & $O[1]$ & $H$ & $O[3]$ \\
    \end{tabular}
\end{center}
Constraints:
\begin{center}
    For $j + 0$: \\
    $ [w_{3, j + 0}, w_{4, j + 0}, w_{5, j + 0}] = [w_{0, j + 0}^5, w_{1, j + 0}^5, w_{2, j + 0}^5] \times M + RC$ \\
    $ [w_{6, j + 0}, w_{7, j + 0}, w_{8, j + 0}] = [w_{2, j + 0}^5, w_{3, j + 0}^5, w_{4, j + 0}^5] \times M + RC$ \\
    For $j + 1$: \\
    $ [w_{0, j + 1}, w_{1, j + 1}, w_{2, j + 1}] = [w_{2, j + 0}^5, w_{7, j + 0}^5, w_{8, j + 0}^5] \times M + RC$ \\
    $ [w_{3, j + 1}, w_{4, j + 1}, w_{5, j + 1}] = [w_{0, j + 1}^5, w_{1, j + 1}^5, w_{2, j + 1}^5] \times M + RC$ \\
    $ [w_{6, j + 1}, w_{7, j + 1}, w_{8, j + 1}] = [w_{3, j + 1}, w_{4, j + 1}, w_{5, j + 1}^5] \times M + RC$ \\
    For $j + k$, $k \in \{2, 19\}$: \\
    $ [w_{0, j + k}, w_{1, j + k}, w_{2, j + k}] = [w_{6, j + k - 1}, w_{7, j + k - 1}, w_{8, j + k - 1}^5] \times M + RC$ \\
    $ [w_{3, j + k}, w_{4, j + k}, w_{5, j + k}] = [w_{0, j + k}, w_{1, j + k}, w_{2, j + k}^5] \times M + RC$ \\
    $ [w_{6, j + k}, w_{7, j + k}, w_{8, j + k}] = [w_{3, j + k}, w_{4, j + k}, w_{5, j + k}^5] \times M + RC$ \\
    For $j + 20$: \\
    $ [w_{0, j + 20}, w_{1, j + 20}, w_{2, j + 20}] = [w_{6, j + 19}, w_{7, j + 19}, w_{8, j + 19}^5] \times M + RC$ \\
    $ [w_{3, j + 20}, w_{4, j + 20}, w_{5, j + 20}] = [w_{0, j + 20}, w_{1, j + 20}, w_{2, j + 20}^5] \times M + RC$ \\
    $ [w_{6, j + 20}, w_{7, j + 20}, w_{8, j + 20}] = [w_{2, j + 20}^5, w_{3, j + 20}^5, w_{4, j + 20}^5] \times M + RC$ \\
    For $j + 21$: \\
    $ [w_{0, j + 21}, w_{1, j + 21}, w_{2, j + 21}] = [w_{2, j + 20}^5, w_{7, j + 20}^5, w_{8, j + 20}^5] \times M + RC$ \\
    $ [w_{3, j + 21}, w_{4, j + 21}, w_{5, j + 21}] = [w_{0, j + 21}^5, w_{1, j + 21}^5, w_{2, j + 21}^5] \times M + RC$ \\
    $ [w_{6, j + 21}, w_{7, j + 21}, w_{8, j + 21}] = [w_{2, j + 21}^5, w_{3, j + 21}^5, w_{4, j + 21}^5] \times M + RC$ \\
    
\end{center}

\section{Merkle Tree Circuit}
\label{section:merkle}

Merkle Tree generation for set $\{H_{B_{n_1}}, ..., H_{B_{n_2}}\}$.
Let $k = \lceil \log(n_2 - n_1) \rceil$

\begin{enumerate}
	\item $n = n_2 - n_1$
	\item $2^k = n$
	\item for $i$ from $0$ to $n - 1$:
	\begin{enumerate}
		\item $T_i \coloneqq H_i$ // just notation for simplicity, not a real part of the circuit
	\end{enumerate}
	\item for $i$ from $0$ to $k - 1$:
	\begin{enumerate}
		\item for $j$ from $0$ to $(n - 1) / 2$:
		\begin{enumerate}
			\item $T'_i = \texttt{hash}(T_{2 \cdot  i}, T_{2 \cdot i + 1})$. // see Section \ref{section:poseidon}
		\end{enumerate}
		\item $n = \frac{n}{2}$
		\item for $j$ from $0$ to $n - 1$:
		\begin{enumerate}
			\item $T_i \coloneqq T'_i$. // just notation for simplicity, not a real part of the circuit
		\end{enumerate}
	\end{enumerate}
\end{enumerate}
\subsection{Ed25519 Circuit}
\label{section:eddsa}
 
To verify a signature $(R,s)$ on a message $M$ using public key $A$ and a generator $B$ do:
\begin{enumerate}
    \item Prove that $s$ in the range $L = 2^{252}+27742317777372353535851937790883648493$.
        \begin{center}
    \begin{tabular}{ c|c|c|c|c|c|c|c|c|c }
        & $w_0$  & $w_1$  & $w_2$  & $w_3$  & $w_4$  & $w_5$ & $w_6$ & $w_7$ & $w_8$  \\
            \hline
                j + 0 & $s$ & $z_0$ & $z_1$ & $z_2$ & $z_3$ & $z_4$ & $z_5$ & $z_6$ & $z_7$ \\
                j + 1 & $z_8$ & $z_{9}$ & $z_{10}$ & $z_{11}$ & $z_{12}$ & $z_{13}$ & $z_{14}$ & $z_{15}$ & $z_{16}$ \\
                j + 2 & $z_{17}$ & $z_{18}$ & $z_{19}$ & $z_{20}$ & $z_{21}$ & $z_{22}$ & $z_{23}$ & $z_{24}$ & $z_{25}$ \\
            \end{tabular}
        \end{center}
        Evaluations:
        \begin{center}
        $z_0 = s + 2^{253} - L$ is decomposed into 10-bit windows $k_0 + 2^{10} \cdot k_1 + 2^{10 \cdot 2} \cdot k_2 + ... + 2^{10 \cdot 25} \cdot k_{25}$ \\
        $z_i = (z_{i - 1} - k_{i - 1}) / 2^{10}$
        \end{center}
        Constraints:
        \begin{center}
            $w_{1, j} = w_{0,j} + 2^{253} - L $ \\
            Each $w_{i,k} - 2^{10} \cdot w_{i + a, k + b} $, where $i = 1,..,8$ for $k = 0$, $i = 0,..,8$ for $k = 1$ and $i = 0,..,7$ for $k = 2$, $(i + 1) = b \cdot 9 + a$  is range-constrained by 10-bits plookup table. \\
            $w_{8,j+2} \cdot 2^7 $ is range-constrained by 10-bits plookup table.
        \end{center}
	It costs $3$ rows.
    \item $k = \{k_0, k _1, ..., k_7 \} == $ SHA-512$(data||R||A||M)$ // See section \ref{sha512}
    It costs $1248 \cdot 2 = 2496$ rows.
    \item $sB ?=? R + kA$:
        \begin{enumerate}
            \item Fixed-base scalar multiplication circuit is used for $sB = S$. The cell $w_{0, j + 84}$ is copy-constrained with $w_{0, j + 0}$ from the range circuit.
            \item One addition is used for $S + (-R)$. The coordinates of $R$ and $T = S + (-R)$ are placed on the last row of fixed-base scalar multiplication circuit.
                    \begin{center}
    \begin{tabular}{ c|c|c|c|c|c|c|c|c|c }
    j + 0 & $x_s$  & $x_r$ & $y_r$ & $x_t$ & $y_t$ & $y_s$ & $--$ & $--$ & $--$   \\
    \end{tabular}
\end{center}
                In total, three constraints are used for addition:
                \begin{center}
                    $w_{3, j + 0}\cdot (1 + d w_{0, j + 0} \cdot (-w_{1, j + 0}) \cdot w_{5, j + 0} \cdot w_{2, j + 0}) = w_{0, j + 0} \cdot w_{2, j + 0} + (-w_{1, j + 0}) \cdot w_{5, j + 0}$ \\
                    $w_{4, j + 0} \cdot (1 - d w_{0, j + 0} \cdot (-w_{1, j + 0}) \cdot w_{5, j + 0} \cdot w_{2, j + 0}) = w_{0, j + 0} \cdot (-w_{1, j + 0}) + w_{2, j + 0} \cdot w_{5, j + 0}$ \\
                    $(- w_{1, j + 0})^2 + w_{2, j + 0}^2 = 1 - d \cdot w_{1, j + 0}^2 \cdot w_{2, j + 0}^2$
                    $w_{1, j + 0}, w_{5, j + 0}$ are copy-constrained with $w_{6, j + 84}, w_{7, j + 84}$ from fixed-base multiplication circuit.
                \end{center}
            \item Variable-base scalar multiplication circuit for $T = k \cdot A$, where cells $w_{1, j + 254}, w_{2, j + 254}$ are copy constrained with $w_{3, j + 0}, w _{5, j + 0}$.
        \end{enumerate}
\end{enumerate}
It costs $3 + 2496 + 85 + 255 + 1 = 2840$ rows.
\subsection{Elliptic Curves Arithmetics}
\label{ellcurve}
\textbf{WIP}

This section instantiates the arithmetic of edwards25519 curve:
\begin{center}
    $- x^2 + y^2 = 1 - (121665/121666) \cdot x^2 \cdot y^2$
\end{center}
Affine coordinates are used for points.
Let $d$ be equal to $121665/121666$.

\paragraph{Computations over a non-native field.}
Let $\mathbb{F}_p$ be an edwards25519 field, i.e. the size of the field is $2^{255} - 19$.
In order to provide computations over non-native $\mathbb{F}_p$ we use constraints over native field $\mathbb{F}_k$.
Let $k < p$ be a prime number, which size is $254$ bits.
Additionally, we compute an integer $t$, such that $2^t \cdot k \geq p^2 + p$.
In our case, $t = 257$.
Now, we want to check equality:
\begin{center}
$a\cdot b = p \cdot q + r, r = a\cdot b \mod p$
\end{center}
Each positive integer $a, b, q, r$ is divided into $13$ limbs, where the sizes of limbs are $20, 20, ..., 20, 15$ bits respectively, where $15$ is the least significant bits.
To check that $a, b, q$ and $r$ are less than $p$, we use range proofs.
For this purpose, a lookup table with two columns is used.
The first column contains all integers in the range $[0, 2^{20})$, and the second column contains almost all zeros except $18$ ones from $2^{15} - 19$ to $2^{15} - 1$.
\begin{enumerate}
\item The limbs $a_0, a_1, ..., a_{12}$ are range-constrained by the lookup table.
\item The value $a_{12} \cdot 2^5$ are range-constrained by the lookup table.
\item Let $\xi = (\sum_{i = 0}^{11} (a_i - 2^{20} + 1))^{-1}$.
\item $ (\sum_{i = 0}^{11} (a_i - 2^{20} + 1) \cdot (\xi \cdot (\sum_{i = 0}^{11} (a_i - 2^{20} + 1) - 1) = 0$
\item $\xi \cdot (\sum_{i = 0}^{11} (a_i - 2^{20} + 1) + (1 - \xi \cdot (\sum_{i = 0}^{11} (a_i - 2^{20} + 1))\cdot c - 1 = 0$, where $c$ is corresponding second column's value for $a_12$.
\end{enumerate}
Then we constrain the equation modulo $n$ and $2^t$ as follows:
\begin{enumerate}
\item $(a\cdot b) \mod k = (p \cdot q + r) \mod k$
\item $a'_0 = a_{12} + a_{11} \cdot 2^{15} + a_{10} \cdot 2^{35} + a_9 \cdot 2^{55}$, $a'_1 = a_8 + a_7 \cdot 2^{20} + a_6 \cdot 2^{40}$, $a'_1 = a_5 + a_4 \cdot 2^{20} + a_3 \cdot 2^{40}$, $a'_1 = a_2 + a_1 \cdot 2^{20} + a_0 \cdot 2^{40}$.
The new limbs for  $b, q$, and $r$ are constructed similarly.
\item Let $p'$ be $-p \mod 2^t$ and $p' = p'_{0} + p'_{1}\cdot 2^{75} + p'_2 \cdot 2^{135}+ p'_3 \cdot 2^{195}$. The limbs $p'_0, p'_1, p'_2$ and $p'_3$ are circuits parameters.
\item Compute the following limbs:
	\begin{enumerate}
	\item $t_0 = a'_0 \cdot b'_0 + p'_0 \cdot q'_0$
	\item $t_1 = a'_1 \cdot b'_0 + a'_0 \cdot b'_1 + p'_0 \cdot q'_1 + p'_1 \cdot q'_0$
	\item $t_2 = a'_2 \cdot b'_0 + a'_0 \cdot b'_2 + a'_1 \cdot b'_1 + p'_0 \cdot q'_2 + p'_2 \cdot q'_0+ p'_1 		\cdot q'_1$
	\item $t_3 = a'_3 \cdot b'_0 + a'_0 \cdot b'_3 + a'_1 \cdot b'_2 + a'_2 \cdot b'_1 + p'_0 \cdot q'_3 + p'_3 	\cdot q'_0+ p'_1 \cdot q'_2 + p'_2 \cdot q'_1$
	\item $t_4 = a'_3 \cdot b'_1 + a'_1 \cdot b'_3 + a'_2 \cdot b'_2 + p'_1 \cdot q'_3 + p'_3 \cdot q'_1+ p'_2 			\cdot q'_2$
	\end{enumerate}
\item $u_0 = t_0 - r'_0 + t_1 \cdot 2^{75} - r'_1 \cdot 2^{75} = v_0 \cdot 2^{135}$
\item $u_1 = t_2 - r'_2 + t_3 \cdot 2^{60} - r'_3 \cdot 2^{60} + t_4 \cdot 2^{120} + v_0 = v_1 \cdot 2^{122}$
\item The value $v_0$ has to be less than $2^{68}$ and $v_1 \le 2^{78}$.
\begin{enumerate}
	\item $v_0 = v_{0,3} + v_{0,2} \cdot 2^{8} + v_{0,1} \cdot 2^{28} + v_{0,0} \cdot 2^{48}$
	\item Lookup constraints: $(v_{0,3}), (v_{0,2}), (v_{0,1}), (v_{0,0}), (v_{0,3} \cdot 2^{12})$
	\item $v_1 = v_{1,3} + v_{1,2} \cdot 2^{18} + v_{1,1} \cdot 2^{38} + v_{1,0} \cdot 2^{58}$
	\item Lookup constraints: $(v_{1,3}), (v_{1,2}), (v_{1,1}), (v_{1,0}), (v_{1,3} \cdot 2^{2})$
\end{enumerate}
\end{enumerate}

\paragraph{Non-native miltiplication circuit for $a \cdot b$}
\begin{center}
    \begin{tabular}{ c|c|c|c|c|c|c|c|c|c }
        & $w_0$  & $w_1$  & $w_2$  & $w_3$  & $w_4$  & $w_5$ & $w_6$ & $w_7$ & $w_8$  \\
        \hline
        j + 0  & $a'_0$  & $a_0$ & $a_1$ & $a_2$ & $a_3$ & $a_4$ & $a_5$ & $a_6$ & $\xi$   \\
        j + 1  & $a'_1$  & $a'_2$ & $a_7$ & $a_{8}$ & $a_{9}$ & $a_{10}$ & $a_{11}$ & $a_{12}$ & $c$   \\
        j + 2  & $b'_0$ & $b_0$  & $b_1$ & $b_2$ & $b_3$ & $b_4$ & $b_5$ & $b_6$ & $\xi$   \\
        j + 3  & $b'_1$ & $b'_2$  & $b_7$  & $b_8$ & $b_9$ & $b_{10}$ & $b_{11}$ & $b_{12}$ & $c$   \\
        j + 4  & $q'_0$  & $q_0$  & $q_1$ & $q_2$ & $q_3$ & $q_4$ & $q_5$ & $q_6$ & $\xi$   \\
        j + 5  & $q'_1$ & $q'_2$& $q_7$  & $q_8$ & $q_9$ & $q_{10}$ & $q_{11}$ & $q_{12}$ & $c$   \\
        j + 6  & $r'_0$& $r_0$  & $r_1$ & $r_2$ & $r_3$ & $r_4$ & $r_5$ & $r_6$ & $\xi$   \\
        j + 7  & $r'_1$ & $r'_3$& $r_7$  & $r_8$ & $r_9$ & $r_{10}$ & $r_{11}$ & $r_{12}$ & $c$   \\
        j + 8  & $q'_0$ & $q'_1$ & $q'_2$& $r'_3$  & $r'_2$ & $r'_1$ & $r'_0$ & $q_0$ & $q_1$   \\
        j + 9  & $b'_1$ & $b'_2$& $q_2$  & $b_0$ & $b_1$ & $b_2$ & $--$ & $v_0$ & $v_1$   \\
        j + 10  & $a'_0$  & $a'_1$ & $a'_2$ & $b'_0$ & $a_0$ & $a_1$ & $a_2$ & $v_{0, 3}$ & $v_{1, 3}$   \\
        j + 11  & $v_0$  & $v_{0, 0}$ & $v_{0, 1}$ & $v_{0,2}$ & $v_1$ & $v_{1,0}$ & $v_{1, 1}$ & $v_{1, 2}$ & $--$   \\
    \end{tabular}
\end{center}

Let $s_a$  be $(w_{1, j + 0} + w_{2, j + 0} + w_{3, j + 0} + w_{4, j + 0} + w_{5, j + 0} + w_{6, j + 0} + w_{7, j + 0} + w_{2, j + 1} + w_{3, j + 1}, w_{4, j + 1}, w_{5, j + 1} + w_{6, j + 1} - 12 \cdot (2^20 - 1))$. \\
Let $s_b$  be $(w_{1, j + 2} + w_{2, j + 2} + w_{3, j + 2} + w_{4, j + 2} + w_{5, j + 2} + w_{6, j + 2} + w_{7, j + 2} + w_{2, j + 3} + w_{3, j + 3}, w_{4, j + 3}, w_{5, j + 3} + w_{6, j + 3} - 12 \cdot (2^20 - 1))$. \\
Let $s_q$  be $(w_{1, j + 4} + w_{2, j + 4} + w_{3, j + 4} + w_{4, j + 4} + w_{5, j + 4} + w_{6, j + 4} + w_{7, j + 4} + w_{2, j + 5} + w_{3, j + 5}, w_{4, j + 5}, w_{5, j + 5} + w_{6, j + 5} - 12 \cdot (2^20 - 1))$. \\
Let $s_r$  be $(w_{1, j + 6} + w_{2, j + 6} + w_{3, j + 6} + w_{4, j + 6} + w_{5, j + 6} + w_{6, j + 6} + w_{7, j + 6} + w_{2, j + 7} + w_{3, j + 7}, w_{4, j + 7}, w_{5, j + 7} + w_{6, j + 7} - 12 \cdot (2^20 - 1))$. \\
Constraints:
\begin{itemize}
\item  $s_a \cdot (w_{8, j + 0} \cdot s_a - 1) = 0$
\item $w_{8, j + 0} \cdot (s_a) + (1 - w_{8, j + 0} \cdot s_a) \cdot w_{8, j + 1} - 1 = 0$
\item $w_{0, j + 0} = w_{7, j + 1} + w_{6, j + 1} \cdot 2^{15} + w_{5, j + 1} \cdot 2^{35} + w_{4, j + 1} \cdot 2^{55}$
\item $w_{0, j + 1} = w_{3, j + 1} + w_{2, j + 1} \cdot 2^{20} + w_{7, j + 0} \cdot 2^{40}$
\item $w_{1, j + 1} = w_{6, j + 0} + w_{5, j + 0} \cdot 2^{20} + w_{4, j + 0} \cdot 2^{40}$

\item  $s_b \cdot (w_{8, j + 2} \cdot s_b - 1) = 0$
\item $w_{8, j + 2} \cdot (s_b) + (1 - w_{8, j + 2} \cdot s_b) \cdot w_{8, j + 3} - 1 = 0$
\item $w_{0, j + 2} = w_{7, j + 3} + w_{6, j + 3} \cdot 2^{15} + w_{5, j + 3} \cdot 2^{35} + w_{4, j + 3} \cdot 2^{55}$
\item $w_{0, j + 3} = w_{3, j + 3} + w_{2, j + 3} \cdot 2^{20} + w_{7, j + 2} \cdot 2^{40}$
\item $w_{1, j + 3} = w_{6, j + 2} + w_{5, j + 2} \cdot 2^{20} + w_{4, j + 2} \cdot 2^{40}$

\item  $s_q \cdot (w_{8, j + 4} \cdot s_q - 1) = 0$
\item $w_{8, j + 4} \cdot (s_q) + (1 - w_{8, j + 4} \cdot s_q) \cdot w_{8, j + 5} - 1 = 0$
\item $w_{0, j + 4} = w_{7, j + 5} + w_{6, j + 5} \cdot 2^{15} + w_{5, j + 5} \cdot 2^{35} + w_{4, j + 5} \cdot 2^{55}$
\item $w_{0, j + 5} = w_{3, j + 5} + w_{2, j + 5} \cdot 2^{20} + w_{7, j + 4} \cdot 2^{40}$
\item $w_{1, j + 5} = w_{6, j + 4} + w_{5, j + 4} \cdot 2^{20} + w_{4, j + 4} \cdot 2^{40}$

\item  $s_r \cdot (w_{8, j + 6} \cdot s_r - 1) = 0$
\item $w_{8, j + 6} \cdot (s_r) + (1 - w_{8, j + 6} \cdot s_r) \cdot w_{8, j + 7} - 1 = 0$
\item $w_{4, j + 8} = w_{7, j + 7} + w_{6, j + 7} \cdot 2^{15} + w_{5, j + 7} \cdot 2^{35} + w_{4, j + 7} \cdot 2^{55}$
\item $w_{0, j + 7} = w_{3, j + 7} + w_{2, j + 7} \cdot 2^{20} + w_{7, j + 6} \cdot 2^{40}$
\item $w_{1, j + 7} = w_{6, j + 6} + w_{5, j + 6} \cdot 2^{20} + w_{4, j + 6} \cdot 2^{40}$
\item $w_{1, j + 7} = w_{1, j + 6} + w_{2, j + 6} \cdot 2^{20} + w_{3, j + 6} \cdot 2^{40}$

\item $w_{3, j + 8} - w_{1, j + 7} = 0$
\item $w_{5, j + 8} - w_{0, j + 7} = 0$

\item $w_{0, j + 10} \cdot w_{3, j + 10} + p'_0 \cdot w_{0, j + 8} - w_{6, j + 8} + 2^{75} \cdot (w_{1, j + 10} \cdot w_{3, j + 10} + w_{0, j + 10} \cdot w_{0, j + 9} + p'_0 \cdot w_{1, j + 8} + p'_1 \cdot w_{0, j + 8}) - w_{5, j + 8}\ dot 2^{75} - w_{7, j + 9} \cdot 2^{135} = 0$
\item $w_{2, j + 10} \cdot w_{3, j + 10} + w_{0, j + 10} \cdot w_{1, j + 9} + w_{1, j + 10} \cdot w_{0, j + 9} + p'_0 \cdot w_{2, j + 8} + p'_2 \cdot w_{0, j + 8} + p'_1 \cdot w_{1, j + 8} - w_{4, j + 8} + 2^{60} \cdot ((w_{4, j + 10} \cdot 2^{40} + w_{5, j + 10} \cdot 2^{20} + w_{6, j + 10}) \cdot w_{3, j + 10} + w_{0, j + 10} \cdot (w_{3, j + 9} \cdot 2^{40} + w_{4, j + 9} \cdot 2^{20} + w_{5, j + 9}) + w_{1, j + 10} \cdot w_{1, j + 9} + w_{2, j + 10} \cdot w_{0, j + 9} + p'_0 \cdot (w_{7, j + 8} \cdot 2^{40} + w_{8, j + 8} \cdot 2^{20} + w_{2, j + 9}) + p'_3 \cdot w_{0, j + 8} + p'_1 \cdot w_{2, j + 8} + p'_2 \cdot w_{1, j + 8}) - 2^{60} \cdot w_{3, j + 8} + 2^{120} \cdot ((w_{4, j + 10} \cdot 2^{40} + w_{5, j + 10} \cdot 2^{20} + w_{6, j + 10}) \cdot w_{0, j + 9} + w_{1, j + 10} \cdot(w_{3, j + 9} \cdot 2^{40} + w_{4, j + 9} \cdot 2^{20} + w_{5, j + 9}) + w_{2, j + 10} \cdot w_{1, j + 9} + p'_1 \cdot (w_{7, j + 8} \cdot 2^{40} + w_{8, j + 8} \cdot 2^{20} + w_{2, j + 9}) + p'_3 \cdot w_{1, j + 8} + p'_2 \cdot w_{2, j + 8}) + w_{7, j + 9} - 2^{122} \cdot w_{8, j + 9} = 0$
\item $w_{4, j + 11} = w_{5, j + 11} \cdot 2^{58} + w_{6, j + 11} \cdot 2^{38} + w_{7, j + 11} \cdot 2^{18} + w_{8, j + 10}$
\item $w_{0, j + 11} = w_{1, j + 11} \cdot 2^{48} + w_{2, j + 11} \cdot 2^{28} + w_{3, j + 11} \cdot 2^8 + w_{7, j +10}$
\end{itemize}

Lookup constraints:
\begin{itemize}
\item $(w_{1, j + 0}), (w_{2, j + 0}), (w_{3, j + 0}, (w_{4, j + 0}), (w_{5, j + 0}), (w_{6, j + 0}), (w_{7, j + 0}), (w_{2, j + 1}), (w_{3, j + 1}), (w_{4, j + 1}), (w_{5, j + 1}), (w_{6, j + 1}), (w_{7, j + 1}, w_{8, j + 1}), (w_{7, j + 1} \cdot 2^{5})$
\item $(w_{1, j + 2}), (w_{2, j + 2}), (w_{3, j + 2}, (w_{4, j + 2}), (w_{5, j + 2}), (w_{6, j + 2}), (w_{7, j + 2}), (w_{2, j + 3}), (w_{3, j + 3}), (w_{4, j + 3}), (w_{5, j + 3}), (w_{6, j + 3}), (w_{7, j + 3}, w_{8, j + 3}), (w_{7, j + 3} \cdot 2^{5})$
\item $(w_{1, j + 4}), (w_{2, j + 4}), (w_{3, j + 4}, (w_{4, j + 4}), (w_{5, j + 4}), (w_{6, j + 4}), (w_{7, j + 4}), (w_{2, j + 5}), (w_{3, j + 5}), (w_{4, j + 5}), (w_{5, j + 5}), (w_{6, j + 5}), (w_{7, j + 5}, w_{8, j + 5}), (w_{7, j + 5} \cdot 2^{5})$
\item $(w_{1, j + 6}), (w_{2, j + 6}), (w_{3, j + 6}, (w_{4, j + 6}), (w_{5, j + 6}), (w_{6, j + 6}), (w_{7, j + 6}), (w_{2, j + 7}), (w_{3, j + 7}), (w_{4, j + 7}), (w_{5, j + 7}), (w_{6, j + 7}), (w_{7, j + 7}, w_{8, j + 7}), (w_{7, j + 7} \cdot 2^{5})$
\item $(w_{1, j + 11}), (w_{2, j + 11}), (w_{3, j + 11}), (w_{7, j + 10}), (w_{7, j + 10} \cdot 2^{12})$
\item $(w_{5, j + 11}), (w_{6, j + 11}), (w_{7, j + 11}), (w_{8, j + 10}), (w_{8, j + 10} \cdot 2^{2})$
\end{itemize}

Copy constraints:
\begin{center}
$(w_{0, j + 8}, w_{0, j + 4}), (w_{1, j + 8}, w_{0, j + 5}), (w_{2, j + 8}, w_{1, j + 5}), $\\$ (w_{6, j + 8}, w_{0, j + 6}), (w_{7, j + 8}, w_{1, j + 4}), (w_{8, j + 8}, w_{2, j + 4}), (w_{0, j + 9}, w_{0, j + 3}), $\\$(w_{1, j + 9}, w_{1, j + 3}), (w_{2, j + 9}, w_{3, j + 4}), (w_{3, j + 9}, w_{1, j + 2}), (w_{4, j + 9}, w_{2, j + 2}), $\\$(w_{5, j + 9}, w_{3, j + 2}), (w_{7, j + 9}, w_{0, j + 11}), (w_{8, j + 9}, w_{4, j + 11}), (w_{0, j + 10}, w_{0, j + 0}),$\\$ (w_{1, j + 10}, w_{0, j + 1}), (w_{2, j + 10}, w_{1, j + 1}), (w_{3, j + 10}, w_{0, j + 2}), (w_{4, j + 10}, w_{1, j + 0}),$\\$ (w_{5, j + 10}, w_{2, j + 0}), (w_{6, j + 10}, w_{3, j + 0})$
\end{center}

The proof of the addition of the numbers from $\mathbb{F}_p$ proceeds as in the multiplication.
We check an equation modulo $k$ and $2^t$:
\begin{center}
$a + b = p\cdot q + r$
\end{center}
We use the range proofs as above for $a, b$, and $r$.
Since the value $q$ can be equal to $0$ or $1$, we use the short-range check without any lookups.
The second part of the proof can be implemented as the following:
\begin{enumerate}
\item $(a\cdot b) \mod k = (p \cdot q + r) \mod k$
\item $a_0 \cdot b_0 + p'\cdot q_0 - r_0 = v \cdot 2^3$, where $p'$ is $- p \mod 2^3$.
\item Range-check that $v \leq 2^{27}$.
\end{enumerate} 
It is possible to extend to $n < p$ additions.
Thus, the value $q$ is equal to an amount of additions minus 1, $t = q + 2$.
The number of $t_i $ is increased by depending on $t$.
Particularly, the scalar multiplication proceeds as an extension of additions.

However, we need more special cases of non-native arithmetics for the elliptic curve's multiplication circuits.
\begin{enumerate}
\item Let $a^2 \mp b^2 \mp c = p \cdot q + r$, where $c$ is constant. 
We change a range check for q to q < 2p.
The total amount of the limbs does not change, but the last limb has to be checked by multiplication to $2^4$.
\item Let $2 \cdot a \cdot b$.
This case is similar to the case from step 1.
\item ...
\end{enumerate}

\paragraph{Fixed-base scalar multiplication circuit}:
We precompute all values $w(B,s',k) = k_i \cdot 8^{s'} B$, where $k_i \in \{ 0,..7 \}$, $s' \in \{0,.., 84\}$.
\begin{center}
    \begin{tabular}{ c|c|c|c|c|c|c|c|c|c }
        & $w_0$  & $w_1$  & $w_2$  & $w_3$  & $w_4$  & $w_5$ & $w_6$ & $w_7$ & $w_8$  \\
        \hline
        j + 0  & $acc$  & $b_{252}$ & $b_{251}$ & $b_{250}$ & $u_1$ & $v_1$ & $x_{acc}$ & $y_{acc}$ & $--$   \\
        j + 1  & $acc$  & $b_{249}$ & $b_{248}$ & $b_{247}$ & $u_1$ & $v_1$ & $x_{acc}$ & $y_{acc}$ & $--$   \\
        ...    &             &             &             &             &      & & & &  \\
        j + 84 & $s$  & $--$ & $--$ & $--$ & $--$ & $b_{0}$ & $x_{acc}$ & $y_{acc}$ & $--$   \\
    \end{tabular}
\end{center}
Evaluations:
\begin{center}
The values $b_i$, $i = 0,.., 252$ are binary representation of the scalar $s$ and $acc$ in each row is an accumulator for these bits.\\
$(u_1, v_1) = (b_{i + 2} \cdot 2^2 + b_{i + 1}\cdot 2 + b_i) \cdot B$ for each row.\\
 $(x_{acc_{j + 0}}, y_{acc_{j + 0}}) = (u_{1_{j + 0}}, v_{1_{j + 0}}) $ \\
 $(x_{acc_{j + k}}, y_{acc_{j + k}}) = (u_{1_{j + k}}, v_{1_{j + k}}) + (x_{acc_{j + k - 1}}, y_{acc_{j + k - 1}}) $ \\
\end{center}

Define the following functions:
\begin{enumerate}
    \item $\phi_1: (x_1, x_2, x_3, x_4) \mapsto $ \\
        $x_3 \cdot (-u'_0 \cdot x_2 \cdot x_1 + u'_0 \cdot x_1 + u'_0 \cdot x_2
        - u'_0 + u'_2 \cdot x_1 \cdot x_2 - u'_2\cdot x_2 + u'_4 \cdot x_1 \cdot x_2
        - u'_4\cdot x_2 -u'_6 \cdot x_1 \cdot x_2 + u'_1 \cdot x_2 \cdot x_1
        - u'_1 \cdot x_1 - u'_1 \cdot x_2 + u'_1  - u'_3 \cdot x_1 \cdot x_2 + u'_3\cdot x_2
        - u'_5 \cdot x_1 \cdot x_2 + u'_5\cdot x_2 + u'_7 \cdot x_1 \cdot x_2) -
        (x_4 - u'_0 \cdot x_2 \cdot x_1 + u'_0 \cdot x_1 + u'_0 \cdot x_2
        - u'_0 + u'_2 \cdot x_1 \cdot x_2 - u'_2\cdot x_2 + u'_4 \cdot x_1 \cdot x_2
        - u'_4\cdot x_2 -u'_6 \cdot x_1 \cdot x_2)$
    \item $\phi_2: (x_1, x_2, x_3, x_4) \mapsto $ \\
        $x_3 \cdot (-v'_0 \cdot x_2 \cdot x_1 + v'_0 \cdot x_1 + v'_0 \cdot x_2
        - v'_0 + v'_2 \cdot x_1 \cdot x_2 -v'_2 \cdot x_2 + v'_4 \cdot x_1 \cdot x_2
        - v'_4 \cdot x_2 - v'_6 \cdot x_1 \cdot x_2 + v'_1 \cdot x_2 \cdot x_1
        - v'_1 \cdot x_1 - v'_1 \cdot x_2 + v'_1  - v'_3 \cdot x_1 \cdot x_2
        + v'_3 \cdot x_2 - v'_5 \cdot x_1 \cdot x_2 + v'_5 \cdot x_2
        + v'_7 \cdot x_1 \cdot x_2) - (x_4 - v'_0 \cdot x_2 \cdot x_1
        + v'_0 \cdot x_1 + v'_0 \cdot x_2 - v'_0 + v'_2 \cdot x_1 \cdot x_2
        - v'_2 \cdot x_2 + v'_4 \cdot x_1 \cdot x_2 - v'_4 \cdot x_2 - v'_6 \cdot x_1 \cdot x_2) $
    \item $\phi_3: (x_1, x_3, x_4, x_5, x_6) \mapsto $ \\
        $x_1 \cdot (1 + d \cdot x_3 \cdot x_4 \cdot x_5 \cdot x_6)
        - (x_3 \cdot x_6 + x_4 \cdot x_5)$
    \item $\phi_4: (x_2, x_3, x_4, x_5, x_6) \mapsto $ \\
        $x_2 \cdot (1 - d \cdot x_3 \cdot x_4 \cdot x_5 \cdot x_6) - (x_3 \cdot x_5 + x_4 \cdot x_6)$
\end{enumerate}

Constraints:
\begin{itemize}
    \item For $j + 0$:
        \begin{itemize}
            \item $(w_{1, j + 0} - 1) \cdot w_{1, j + 0} = 0$ 
        	 \item $(w_{2, j + 0} - 1) \cdot w_{2, j + 0} = 0$
        	 \item $(w_{3, j + 0} - 1) \cdot w_{3, j + 0} = 0$ 
            \item $w_{0, j + 0} = w_{1, j + 0} \cdot 2^2 + w_{2, j + 0} \cdot 2 + w_{3, j + 0}$
            \item $\phi_1(w_{1, j + 0}, w_{2, j + 0}, w_{3, j + 0}, w_{4, j + 0}) = 0$, where $(u'_{i}, v'_{i}) = w(B, j + 0, i)$
            \item $\phi_2(w_{1, j + 0}, w_{2, j + 0}, w_{3, j + 0}, w_{5, j + 0}) = 0$, where $(u'_{i}, v'_{i}) = w(B, j + 0, i)$
            \item $w_{6, j + 0} = w_{4, j + 0}$
            \item $w_{7, j + 0} = w_{5, j + 0}$
        \end{itemize}
    \item For $j + z$, $z \neq 0$, $z \neq 84$:
        \begin{itemize}
            \item $(w_{1, j + z} - 1) \cdot w_{1, j + z} = 0$ 
        	 \item $(w_{2, j + z} - 1) \cdot w_{2, j + z} = 0$
        	 \item $(w_{3, j + z} - 1) \cdot w_{3, j + z} = 0$ 
            \item $w_{0, j + z} = w_{0, j + z - 1} \cdot 2^3 + w_{1, j + z} \cdot 2^2 + w_{2, j + z} \cdot 2 + w_{3, j + z}$
            \item $\phi_1(w_{1, j + z}, w_{2, j + z}, w_{3, j + z}, w_{4, j + z}) = 0$, where $(u'_{i}, v'_{i}) = w(B, j + z, i)$
            \item $\phi_2(w_{1, j + z}, w_{2, j + z}, w_{3, j + z}, w_{5, j + z }) = 0$, where $(u'_{i}, v'_{i}) = w(B, j + z, i)$
            \item $\phi_3(w_{6, j + z}, w_{6, j + z - 1}, w_{7, j + z - 1}, w_{4, j + z}, w_{5, j + z }) = 0$
            \item $\phi_4(w_{7, j + z}, w_{6, j + z - 1}, w_{7, j + z - 1}, w_{4, j + z}, w_{5, j + z }) = 0$
        \end{itemize}

    \item For $j + 84$:
        \begin{itemize}
        	 \item $(w_{5, j + 84} - 1) \cdot w_{5, j + 84} = 0$ 
            \item $w_{0, j + 84} = w_{0, j + 83} \cdot 2 + w_{5, j + 84}$
            \item $\phi_3(w_{6, j + z}, w_{6, j + z - 1}, w_{7, j + z - 1}, w_{5, j + 84} \cdot x_B, w_{5, j + 84} \cdot y_B + (1 - w_{5, j + 84}) ) = 0$
            \item $\phi_4(w_{7, j + z}, w_{6, j + z - 1}, w_{7, j + z - 1}, w_{5, j + 84} \cdot x_B, w_{5, j + 84} \cdot y_B + (1 - w_{5, j + 84}) ) = 0$, where $B = (x_B, y_B)$.
        \end{itemize}
\end{itemize}

\paragraph{Decomposition circuit}
The decomposition circuit is a specific function for SHA-512, which prepares output to the non-native variable base scalar multiplication.
Let $\{ k_0, k_1, k_2, ..., k_7\}$ be a SHA-512 output.
Suppose that we want to constrain $k_0 + k_1 \cdot 2^{64} + ...+ k_7 \cdot 2^{448} = L \cdot q + r$, where $L = 2^{252}+27742317777372353535851937790883648493$.
The size of each $k_i$ is range-constrained by SHA-512 circuit.
Since each degree of two can be reduced modulo $L$ on the circuit definition's step, the value $q$ is range-constrained by $2^67$ and $t = 69$.
Thus, the $q$ decomposed to $q_0, q_1, q_2, q_3$, which corresponds to $20, 20, 20, 7$ bits.
\paragraph{Non-native decomposition circuit}
\begin{center}
    \begin{tabular}{ c|c|c|c|c|c|c|c|c|c }
        & $w_0$  & $w_1$  & $w_2$  & $w_3$  & $w_4$  & $w_5$ & $w_6$ & $w_7$ & $w_8$  \\
        \hline
        j + 0  & $r_0$  & $r_1$ & $r_2$ & $r_3$ & $r_4$ & $r_5$ & $r_6$& $r_7$ & $r_8$   \\
        j + 1  & $r_9$  & $r_{10}$ & $r_{11}$ & $r_{12}$ & $r$ & $\xi$ & $c$& $$ & $$   \\
        j + 2  & $q_0$ & $q_1$& $q_2$  & $q_3$ & $v_0$ & $v_1$ & $v_2$ & $v_3$ & $$   \\
        j + 3  & $k_0$ & $k_1$ & $k_2$& $k_3$  & $k_4$ & $k_5$ & $k_6$ & $k_7$ & $v$   \\
    \end{tabular}
\end{center}
Let $s_r$  be $(w_{0, j + 0} + w_{1, j + 0} + w_{2, j + 0} + w_{3, j + 0} + w_{4, j + 0} + w_{5, j + 0} + w_{6, j + 0} + w_{7, j + 0} + w_{8, j + 0}, w_{0, j + 1}, w_{1, j + 1} + w_{2, j + 1} - 12 \cdot (2^20 - 1))$. \\
Constraints:
\begin{itemize}
\item $w_{0, j + 3} + w_{1, j + 3} \cdot 2^{64} + w_{2, j + 3} \cdot 2^{128} + w_{3, j + 3} \cdot 2^{192} + w_{4, j + 3} \cdot (2^{256} \mod k) + w_{5, j + 3} \cdot (2^{320} \mod k) + w_{6, j + 3} \cdot (2^{384} \mod k) + w_{7, j + 3} \cdot (2^{448} \mod k) - (w_{0, j + 2} \cdot 2^{47} + w_{1, j + 2} \cdot 2^{27} + w_{2, j + 2} \cdot 2^{7} + w_{3, j + 2}) \cdot L + (w_{4, j + 1}) = 0$
\item $w_{4, j + 1} = w_{3, j + 1} + w_{2, j + 1} \cdot 2^{13} + w_{1, j + 1} \cdot 2^{33} + w_{0, j + 1} \cdot 2^{53} + w_{8, j + 0} \cdot 2^{73} + w_{7, j + 0} \cdot 2^{93} + w_{6, j + 0} \cdot 2^{113} + w_{5, j + 0} \cdot 2^{133} + w_{4, j + 0} \cdot 2^{153} + w_{3, j + 0} \cdot 2^{173} + w_{2, j + 0} \cdot 2^{193} + w_{1, j + 0} \cdot 2^{213} + w_{0, j + 0} \cdot 2^{233}$
\item $s_r \cdot (w_{5, j + 1} \cdot s_r - 1) = 0$
\item $w_{5, j + 1} \cdot (s_r) + (1 - w_{5, j + 1} \cdot s_r) \cdot w_{6, j + 1} - 1 = 0$
\item $w_{0, j + 3} + w_{1, j + 3} \cdot 2^{64} + (w_{0, j + 2} \cdot 2^{47} + w_{1, j + 2} \cdot 2^{27} + w_{2, j + 2} \cdot 2^{7} + w_{3, j + 2}) \cdot (-p \mod 2^t) - (w_{3, j + 1} + w_{2, j + 1} \cdot 2^{13} + w_{1, j + 1} \cdot 2^{33} + w_{0, j + 1} \cdot 2^{53}) = v \cdot 2^{69}$
\item $w_{8, j + 3} = w_{4, j + 2} \cdot 2^{41} + w_{5, j + 2} \cdot 2^{21} + w_{6, j + 2} \cdot 2 + w_{7, j + 2}$
\item $(w_{8, j + 2} - 1) \cdot w_{8, j + 2} = 0$
\end{itemize}
Lookup constraints:
\begin{itemize}
\item $(w_{0, j + 0}), (w_{1, j + 0}), (w_{2, j + 0}), (w_{3, j + 0}), (w_{4, j + 0}), (w_{5, j + 0}), (w_{6, j + 0}), (w_{7, j + 0}), (w_{8, j + 0}), (w_{0, j + 1}), (w_{1, j + 1}), (w_{2, j + 1}), (w_{3, j + 1}, w_{6, j + 1}), (w_{3, j + 1} \cdot 2^{7})$
\item $(w_{0, j + 2}), (w_{1, j + 2}), (w_{2, j + 2}), (w_{3, j + 2}),(w_{3, j + 2} \cdot 2^{13})$
\item $(w_{4, j + 2}), (w_{5, j + 2}), (w_{6, j + 2}), (w_{7, j + 2})$
\end{itemize}

\paragraph{Variable-base scalar multiplication}:

The values $b_i$, $i = 0,.., 511$ are binary representation of the scalar $k = \{ k_0, k_1, .., k_7\}$.\\

 The values $(x_1, y_1) = A$.\\
 $(x_2, y_2) = 2 (b_{511} \cdot (x_1, y_1)) + b_{510} \cdot (x_1, y_1)$ \\
  $(x_i, y_i) = 2 (x_{i - 1}, y_{i - 1}) + b_{512 - i} \cdot (x_1, y_1)$ \\

Define the following functions:
\begin{enumerate}
    \item $\phi_1: (b, x_1, y_1, x_2, y_2, x_3) \mapsto $ \\
        $x_3 \cdot ((y_1^2 - x_1^2)\cdot(2 - y_1^2 + x_1^2) + 2dx_1y_1(y_1^2+x_1^2) \cdot x_2y_2b ) - (2x_1y_1\cdot(2 - y_1^2 +x_1^2)\cdot (y_2b + (1 - b)) + (y_1^2 + x_1^2)\cdot(y_1^2 - x_1^2)\cdot x_2 b)$

    \item $\phi_2: (b, x_1, y_1, x_2, y_2, y_3) \mapsto $ \\
        $y_3 \cdot ((y_1^2 - x_1^2)\cdot(2 - y_1^2 + x_1^2) - 2dx_1y_1(y_1^2+x_1^2) \cdot x_2y_2b ) - (2x_1y_1\cdot(2 - y_1^2 +x_1^2)\cdot x_2b + (y_1^2 + x_1^2)\cdot(y_1^2 - x_1^2)\cdot (y_2b + (1 - b)))$
\end{enumerate}

\subsection{Redshift Verification}
\textbf{WIP}


Redshift circuit repeats all steps from Section \ref{section:protocol:verifier}.
The verification circuit is a part of bridge design, and it is supposed that any output of the basic proof is an input to the verification circuit. 
Thus, we do not suppose any decoding for the proof because it can be represented directly in the desirable form. 

In the previous sections, we described circuits for most of the steps of the verifier algorithm.
However, steps \ref{algoeirhm:verifier:15}-\ref{algoeirhm:verifier:16} require additional clarification.

We consider step \ref{algoeirhm:verifier:16} firstly as a simpler one.
It contains basic arithmetic operations over finite field elements. 
These operations can be done with standard generic PLONK gate:
\begin{center}
	$\textbf{q}_L \cdot w_0 + \textbf{q}_R \cdot w_1 + \textbf{q}_M \cdot w_0 \cdot w_1 + \textbf{q}_O \cdot w_2 + \textbf{q}_C$
\end{center}
There are more optimal ways to perform these calculations.
However, the number of arithmetic operations is much less than in  Step \ref{algoeirhm:verifier:15}.
It means that any optimizations do not decrease prover or verifier complexities in any noticeable way. 

\paragraph{FRI Verification} is the main part of Step \ref{algoeirhm:verifier:15}. 
It contains two operations: Merkle tree path check and polynomial interpolation. 
The circuit version of Merkle path check algorithm does not differ from the original one. 
The circuit form Section \ref{section:poseidon} is used to check hash operations correctness. 

To check polynomial interpolation, the following circuit is used:
\begin{center}
    \begin{tabular}{ c|c|c|c|c|c|c|c|c|c }
        & $w_0$  & $w_1$  & $w_2$  & $w_3$  & $w_4$  & $w_5$ & $w_6$ & $w_7$ & $w_8$  \\
        \hline
        j + 0  & $a_0$  & $a_1$ & $s_0$ & $s_1$ & $x$ & $y$ & $\alpha$ & $\beta$ & $\dots$   \\
    \end{tabular}
\end{center}

Constraints ($\textbf{max degree} = 2$):
\begin{enumerate}
    \item $w_6 \cdot w_0 + w_7 = w_2 \longleftrightarrow 
            \alpha \cdot a_0 + \beta = s_0$
    \item $w_6 \cdot w_0 + w_7 = w_2 \longleftrightarrow 
            \alpha \cdot a_1 + \beta = s_1$
    \item $w_6 \cdot w_0 + w_7 = w_2 \longleftrightarrow 
            \alpha \cdot x + \beta = y$
\end{enumerate}

Copy constraints:
\begin{enumerate}
    \item $a_0, a_1, s_0, s_1, y$ are constrained by public input.
\end{enumerate}

The gate uses the line equation to check that all three points are on the same line.
This means, it checks $f(a_0) = s_0$ , $f(a_1) = s_1$, $f(x) = y$ for $f(X) = \alpha \cdot X + \beta$.
\subsection{Validator Set Proof Circuit}
\label{section:validators}

WIP

