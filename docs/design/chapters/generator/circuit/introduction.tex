\section{Circuit Definition}
\label{section:circuit}
This section contains a description of PLONK-style circuits for In-EVM Solana's
"Light Client" state verification\footnote{\url{https://blog.nil.foundation/2021/10/14/solana-ethereum-bridge.html}}.

This section provides a high-level overview of the circuit used for proof generation
and verification.

Following sections provide sub-circuits details.

\subsection{Verification Circuit Overview}

Let bank-hashes of proving block set be $\{H_{B_{n_1}}, ..., H_{B_{n_2}}\}$.
The last confirmed block is $H_{B_{L}}$.
Each positively confirmed block is signed by $M$ validators.

Denote by $\texttt{block\_data}$ the data that is included in the bank hash other than the bank hash of the parent block.

\begin{enumerate}
    \item $H_{B_{n_1}} = H_{B_{L}}$ // $ H_{B_{L}}$ is a public input
    \item Validator set constraints. // see Section \ref{section:validators}
    \item for $i$ from $n_1 + 1$ to $n_2 + 32$:
    \begin{enumerate}
        \item $H_{B_i} = \texttt{sha256}(\texttt{block\_data} || H_{B_{i - 1}})$ // see Section \ref{section:sha256}
    \end{enumerate}
    \item for $j$ from $0$ to $M$:
    \begin{enumerate}
        \item Ed25519 constraints for $H_{B_{n_2 + 32}}$ // see Section \ref{section:eddsa}
    \end{enumerate}
    \item Merkle tree constraints for the set $\{H_{B_{n_1}}, ..., H_{B_{n_2}}\}$ // see Section \ref{section:merkle}
\end{enumerate}
