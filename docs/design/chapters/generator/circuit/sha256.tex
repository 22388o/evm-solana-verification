\subsection{SHA-256 Circuit}
\label{section:sha256}
Suppose that input data is in the 32-bits form, which is already padded to the required size.
We suppose that the checking that chunked input data corresponds to the original data out of the circuit.
However, we do not need to range constrain these chunks as we get them for free from the SHA-256 circuit.


Thus, the preprocessing constraints for the SHA-256 circuit is a decomposition of $k$ message blocks to $32$ bits chunks without range proofs. For `Solana-EVM` circuit, $k = 3$.

\paragraph{Lookup tables}
We use the following lookup tables:
\begin{enumerate}
\item \textbf{SHA-256 NORMALIZE4} with 2 columns and $2^{14}$ rows. 
The first column contains all possible 14-bits words.
The second column contains corresponding sparse representations with base $4$.
The constraints can be used for the range check and sparse representation simultaneously.
\item \textbf{SHA-256 NORMALIZE7} with 2 columns and $2^{14}$ rows. 
The first column contains all possible 14-bits words.
The second column contains corresponding sparse representations with base $7$.
The constraints can be used for the range check and sparse representation simultaneously.
\item \textbf{SHA-256 NORMALIZE MAJ} with 2 columns and $2^{8}$ rows. 
The first column contains all possible 8-bits words.
The second column contains corresponding sparse representations with base $4$.
\item \textbf{SHA-256 NORMALIZE CH} with 2 columns and $2^{8}$ rows. 
The first column contains all possible 8-bits words.
The second column contains corresponding sparse representations with base $7$.
\end{enumerate}

\paragraph{Message scheduling}
For each block of $512$ bits of the padded message the 64 words are constructed in the following way:
\begin{itemize}
\item The first $16$ words are obtained by splitting the message.
\item The last $48$ words are obtained by using the functions $\sigma_0, \sigma_1$:
\begin{equation}\label{word}
W_i = \sigma_1(W_{i-2}) \oplus W_{i - 7} \oplus \sigma_0(W_{i-15}) \oplus W_{i-16}
\end{equation}
\end{itemize}
Each round of the message scheduling has the following table:
\begin{center}
    \begin{tabular}{ c|c|c|c|c|c|c|c|c|c }
        & $w_0$  & $w_1$  & $w_2$  & $w_3$  & $w_4$  & $w_5$ & $w_6$ & $w_7$ & $w_8$  \\
        \hline
        j + 0 & $a$  & $a_0$  & $a_1$ & $a_2$ & $a_3$ & $ \hat{a}_1$ & $\hat{a}_2$ & $a'_0$  & $$      \\
        j + 1 & $W_i$ & $W_j$ & $a'_1$ & $a'_2$ & $a'_3$ & $s'_0$ & $s'_1$ & $s'_2$ & $s'_3$\\
        j + 2 & $w$ & $s_0$ & $s_1$ & $s_2$ & $s_3$ & $s_0$ & $s_1$ & $s_2$ & $s_3$\\
        j + 3 & $$ & $b'_0$ & $b'_1$ & $b'_2$ & $b'_3$ & $s'_0$ & $s'_1$ & $s'_2$ & $s'_3$\\
        j + 4 & $b$  & $b_0$  & $b_1$ & $b_2$ & $b_3$ & $ \hat{b}_0$ & $\hat{b}_1$ & $\hat{b}_3$  & $$ 
    \end{tabular}
\end{center}
Evaluations:
\begin{center}
Let $b$ be $W_{i-2}$ and $a$ be $W_{i-15}$ from \ref{word}.
The values $W_i$ and $W_j$ in the table corresponds to $W_{i - 7}$ and $W_{i - 16}$ respectively from \ref{word}.
From the round $r = 2$ the copy constraints are used for values $b$ and $w$ from round $r - 2$.
The copy constraints for $W_{i-7}, W_{i-15}$ and $W_{i -16}$ are used in a similar way.
The output of round $W_i$ from \ref{word} is $w$.
\end{center}
The first $16$ words require a range check. 
We get it fo free from range-constraining chunks inside functions $\sigma_0$ and $\sigma_1$.
Thus, for $i$ from $16$ to $63$:
\begin{enumerate}
\item Apply $\sigma_0$ to $W_{i-15}$.
\item Add the following constraint for $W_i$:
\begin{center}
	$w_{0,j + 2} = w_{0, j + 1} + w_{1, j + 1} + w_{1, j + 2} + w_{2,j + 2} \cdot 2^3 + w_{3,j + 2} \cdot 2^{7} + w_{4, j + 2} \cdot 2^{18} + w_{5, j + 2} + w_{6, j + 2} \cdot 2^{10} + w_{7,j + 2} \cdot 2^{17} + w_{8, j + 2} \cdot 2^{19}$,
\end{center}
\item Apply $\sigma_1$ to $W_{i-2}$.
\end{enumerate}
Thus, the message schedule takes $5 \cdot 48 = 240$ rows.

\paragraph{The function $\sigma_0$} contains sparse mapping with base $4$.
Let $a$ be divided to chunks $a_0, a_1, a_2, a_3$ which equals to $3, 4, 11, 14$ bits respectively.
The values $a'_0, a'_1, a'_2, a'_3$ are in sparse form, and $a'$ is a sparse $a$.
\textbf{SHA-256 NORMALIZE4} lookup table is used for mapping to sparse representation and range-constraining for each chunk $a_i$, where bit-length of $a_i > 3$.
If a chunk is $14$ bits long, then it is constrained for free.
Else the prover has to calculate the sparse representation $\hat{a}_i$ for $2^j \cdot a_i$, where $j + \texttt{len}(a_i) = 14$ and $\texttt{len}(a_i)$ is bit-length of $a_i$.
The tuple $\{ s'_0, s'_1, s'_2, s'_3\}$ is a sparse representation of the result of $\sigma_0$ and the tuple $\{ s_0, s_1, s_2, s_3\}$  is a normal representation.
The size of elements of these tuples equals to $\{14, 14, 2, 2\}$ bits respectively.

Constraints:
\begin{center}
    $w_{0, j + 0} = w_{1, j + 0} + w_{2,j + 0} \cdot 2^3 + w_{3,j + 0} \cdot 2^{7} + w_{4, j + 0} \cdot 2^{18}$ \\
    $(w_{1, j + 0} - 7) \cdot (w_{1, j + 0} - 6) \cdot ... \cdot w_{1, j + 0} = 0$ \\
    $w_{5, j + 1} + w_{6, j + 1} \cdot 4^{14} + w_{7, j + 1} \cdot 4^{28} + w_{8, j + 1} \cdot 2^{30} = w_{2, j + 1} + w_{3, j + 1} \cdot 4^{4} + w_{4, j + 1} \cdot 4^{15} + w_{3, j + 1} + w_{4, j + 1} \cdot 4^{11} + w_{7, j + 0} \cdot 4^{25} + w_{2, j + 1} \cdot 4^{28} + w_{4, j + 1} + w_{7, j + 0} \cdot 4^{14}+ w_{2, j + 1} \cdot 4^{17} + w_{3, j + 1} \cdot 4^{21}$ \\
    $(w_{7, j + 1} - 3) \cdot (w_{7, j + 1} - 2) \cdot (w_{7, j + 1} - 1) \cdot w_{7, j + 1} = 0$
    $(w_{8, j + 1} - 3) \cdot (w_{8, j + 1} - 2) \cdot (w_{8, j + 1} - 1) \cdot w_{8, j + 1} = 0$
    10 plookup constraints: $(w_{1, j + 0}, w_{7, j + 0}), ( 2^{10} \cdot w_{2, j + 0}, w_{5, j + 0}), (w_{2, j + 0}, w_{2, j + 1}), (2^3 \cdot w_{3, j + 0}, w_{6, j + 0}), (w_{3, j + 0}, w_{3, j + 1}), (w_{4, j + 0}, w_{4, j + 1}), (w_{1, j + 2}, w_{5, j + 1}), (w_{2, j + 2}, w_{6, j + 1}), (w_{3, j + 2}, w_{7, j + 2}), (w_{4, j + 2}, w_{8, j + 2})$ \\
\end{center}

\paragraph{The function $\sigma_1$} contains sparse mapping subcircuit with base $4$.
Let $a$ be divided to chunks $a_0, a_1, a_2, a_3$ which equals to $10, 7, 2, 13$ bits respectively.
The values $a'_0, a'_1, a'_2, a'_3$ are in sparse form and $a'$ is a sparse $a$.
\textbf{SHA-256 NORMALIZE4} lookup table is used for mapping to sparse representation and range-constraining in the same way as for $\sigma_0$.
The tuple $\{ s'_0, s'_1, s'_2, s'_3\}$ is a sparse representation of the result of $\sigma_1$ and the tuple $\{ s_0, s_1, s_2, s_3\}$  is a normal representation.
The size of elements of these tuples equals to $\{14, 14, 2, 2\}$ bits respectively.

Constraints:
\begin{center}
    $w_{0, j + 3} = w_{1, j + 3} + w_{2, j + 3} \cdot 2^{10} + w_{3,j + 3} \cdot 2^{17} + w_{4, j + 3} \cdot 2^{19}$ \\
    $(w_{3,j + 3} - 3) \cdot (w_{3,j + 3} - 2) \cdot (w_{3,j + 3} - 1) \cdot w_{3,j + 3} = 0$ \\
    $w_{5, j + 3} + w_{6, j + 3} \cdot 4^{14} + w_{7, j + 3} \cdot 4^{28} + w_{8, j + 3} \cdot 2^{30} = w_{2, j + 3} + w_{3, j + 3} \cdot 4^{7} + w_{4, j + 3} \cdot 4^{9} + w_{3, j + 3} + w_{4, j + 3} \cdot 4^{2} + w_{1, j + 3} \cdot 4^{15} + w_{2, j + 3} \cdot 4^{25} + w_{4, j + 3} + w_{1, j + 3} \cdot 4^{13}+ w_{2, j + 3} \cdot 4^{23} + w_{3, j + 3} \cdot 4^{30}$ \\
    $(w_{7, j + 3} - 3) \cdot (w_{7, j + 3} - 2) \cdot (w_{7, j + 3} - 1) \cdot w_{7, j + 3} = 0$
    $(w_{8, j + 3} - 3) \cdot (w_{8, j + 3} - 2) \cdot (w_{8, j + 3} - 1) \cdot w_{8, j + 3} = 0$
    11 plookup constraints: $(2^4 \cdot (w_{1, j + 3}, w_{5, j + 3}), (2^7 \cdot w_{2, j + 3}, w_{6,j+3}), (2 \cdot w_{4, j+3}, w_{7, j + 3}), (w_{1,j + 3}, w_{1,j+2}), (w_{2, j + 3}, w_{2, j + 2}), (w_{3,j + 3}, w_{3,j+2}), (w_{4, j + 3}, w_{4, j+2}), (w_{5, j + 2}, w_{5, j + 3}), (w_{6, j + 2}, w_{6, j + 3}), (w_{7, j + 2}, w_{7, j + 3}), (w_{7, j + 2}, w_{8, j + 3})$ \\
\end{center}

\paragraph{Compression}
There are $64$ rounds of compression. 
Each round of compression has the following table:
\begin{center}
    \begin{tabular}{ c|c|c|c|c|c|c|c|c|c }
        & $w_0$  & $w_1$  & $w_2$  & $w_3$  & $w_4$  & $w_5$ & $w_6$ & $w_7$ & $w_8$  \\
        \hline
        j + 0 & $e$ & $e'_0$ & $e_0$ & $e_1 $  & $e_2$  & $e_3$   & $ \hat{e}_1$ & $\hat{e}_2$  & $\hat{e}_3$   \\
        j + 1 & $e'$  & $f'$ & $e'_1$ & $e'_2$ & $e'_3$ & $s'_0$ & $s'_1$ & $s'_2$ & $s'_3$ \\
        j + 2 & $ch_{0, sparse}$  & $ch_{1, sparse}$ & $ch_{2, sparse}$ & $ch_{3, sparse}$ & $-$ & $s_0$ & $s_1$ & $s_2$ & $s_3$ \\
        j + 3 & $g'$ & $d$ & $h$ & $W_r$ & $e_{new}$ & $ch_0$ & $ch_1$ & $ch_2$ & $ch_3$\\
        j + 4 & $maj_{0,sparse}$  & $maj_{1,sparse}$ & $maj_{2,sparse}$  & $maj_{3,sparse}$  & $a_{new}$ & $maj_3$ & $maj_0$ & $maj_1$ & $maj_2$ \\
        j + 5 & $a'$  & $b'$ & $$  & $$  & $c'$ & $s_0$ & $s_1$ & $s_2$ & $s_3$\\
        j + 6 & $s'_1$  & $s'_2$ & $a'_0$  & $a'_1$  & $a'_2$ & $a'_3$ & $s'_3$ & $s'_4$ & $$\\
        j + 7 & $a$ & $$ & $a_0$ & $a_1$ & $a_2$ & $a_3$ & $\hat{a}_0$  & $ \hat{a}_1$ & $\hat{a}_3$ \\
    \end{tabular}
\end{center}

\paragraph{The working variables}
$ a,  b , c , d, e, f, g, h$ equals to the fixed initial $SHA-256$ values for the first chunk and to the sum of previous output and initial values for the rest of chunks. 
The values for chunk $c, c \neg 1$ are copy-constrained with output from previous round.
The variables with quotes are corresponded sparse representation.
For each chunk, the following rows are used:
\begin{center}
    \begin{tabular}{ c|c|c|c|c|c|c|c|c|c }
        & $w_0$  & $w_1$  & $w_2$  & $w_3$  & $w_4$  & $w_5$ & $w_6$ & $w_7$ & $w_8$  \\
        \hline
        j + 0 & $a$ & $a'$ & $b$ & $b' $  & $d$  & $-$   & $-$ & $-$  & $-$   \\
        j + 1 & $c$  & $c'$ & $e$ & $e'$ & $h$ & $-$ & $-$ & $-$ & $-$ \\
        j + 2 & $f$ & $f'$ & $g$ & $g'$ & $-$ & $-$ & $-$ & $-$ & $-$\\
    \end{tabular}
\end{center}
For the first round, $a, a', b', c' , d, e, e', f', g', h$ are copy constrained with corresponded values from the table above.

For the second round, $b', c' , d, f', g', h$ are copy constrained with $a', b', c, e', f', g$ from the table.
The values $a, e$ are copy constrained with $a_{new}, e_{new}$ from the previous round.

For the third round, $c' , d, g', h$ are copy constrained with $a', b, e', f$.
The values $a, e$ are copy constrained with $a_{new}, e_{new}$ from the previous round.
The values $b', f'$ are copy constrained with $a', e'$ from the previous round.

In the rest of the rounds the following `non-special` copy constraints are used:
\begin{enumerate}
\item The values $a, e$ are copy constrained with $a_{new}, e_{new}$ from the previous round.
\item The values $b', f'$ are copy constrained with $a', e'$ from the previous round.
\item The values $c', g'$ are copy constrained with $b', c'$ from the previous round.
\item The values $d, h$ are copy constrained with $a', e'$ from the round $r - 3$, where $r$ is current round.
\end{enumerate}

\paragraph{The $\Sigma_0$ function}
contains subcircuit with base $4$.
Let $a$ be divided to chunks $a_0, a_1, a_2, a_3$ which equals to $2, 11, 9, 10$ bits respectively.
The values $a'_0, a'_1, a'_2, a'_3$ are in sparse form and $a'$ is a sparse $a$.
The tuple $\{ s'_0, s'_1, s'_2, s'_3\}$ is a sparse representation of the result of $\Sigma_0$ and the tuple $\{ s_0, s_1, s_2, s_3\}$  is a normal representation.
The size of elements of these tuples equals to $\{14, 14, 2, 2\}$ bits respectively.
\textbf{SHA-256 NORMALIZE4} lookup table is used for mapping to sparse representation and range-constraining in the same way as for $\sigma_0$.

Constraints:
\begin{center}
    $w_{0, j + 7} = w_{2, j + 7} + w_{3, j + 7} \cdot 2^{2} + w_{4, j + 7} \cdot 2^{13} + w_{5, j + 7} \cdot 2^{22}$ \\
    $w_{0, j + 5} = w_{2, j + 6} + w_{3, j + 6} \cdot 4^{2} + w_{4, j + 6} \cdot 4^{13} + w_{5, j + 6} \cdot 4^{22}$ \\
    $(w_{2,j + 6} - 3) \cdot (w_{2,j + 6} - 2) \cdot (w_{2,j + 6} - 1) \cdot w_{2,j + 6} = 0$ \\
    $w_{0, j + 6} + w_{1, j + 6} \cdot 4^{14} + w_{6, j + 6} \cdot 4^{28} + w_{7, j + 6} \cdot 2^{30} = w_{3, j + 6} + w_{4, j + 6} \cdot 4^{11} + w_{5, j + 6} \cdot 4^{20} + w_{1, j + 6} \cdot 2^{30} + w_{4, j + 6} + w_{5, j + 6} \cdot 4^{9} + w_{2, j + 6} \cdot 4^{19} + w_{3, j + 6} \cdot 4^{21} + w_{5, j + 6} + w_{2, j + 6} \cdot 4^{10}+ w_{3, j + 6} \cdot 4^{12} + w_{4, j + 6} \cdot 4^{23}$ \\
    $(w_{6, j + 6} - 3) \cdot (w_{6, j + 6} - 2) \cdot (w_{6, j + 6} - 1) \cdot w_{6, j + 6} = 0$
    $(w_{7, j + 6} - 3) \cdot (w_{7, j + 6} - 2) \cdot (w_{7, j + 6} - 1) \cdot w_{7, j + 6} = 0$
    11 plookup constraints: $(2^3 \cdot (w_{3, j + 6}, w_{6, j + 6}), (2^5 \cdot w_{4, j + 6}, w_{7, j + 6}), (2^4 \cdot w_{5, j+6}, w_{8, j + 6}), (w_{2,j + 6}, w_{2,j+5}), (w_{3, j + 6}, w_{3, j + 5}), (w_{4, j + 6}, w_{4, j + 5}), (w_{5, j + 6}, w_{5, j + 5}), (w_{5, j + 5}, w_{0, j + 6}), (w_{6, j + 5}, w_{1, j + 6}), (w_{7, j + 5}, w_{6, j + 6}), (w_{8, j + 5}, w_{7, j + 6})$ \\
\end{center}

\paragraph{The $\Sigma_1$ function}
contains subcircuit with base $7$.
Let $a$ be divided to chunks $a_0, a_1, a_2, a_3$ which equals to $6, 5, 14, 7$ bits respectively.
The values $a'_0, a'_1, a'_2, a'_3$ are in sparse form, and $a'$ is a sparse $a$.
The tuple $\{ s'_0, s'_1, s'_2, s'_3\}$ is a sparse representation of the result of $\Sigma_1$ and the tuple $\{ s_0, s_1, s_2, s_3\}$  is a normal representation.
The size of elements of these tuples equals to $\{14, 14, 2, 2\}$ bits respectively.
\textbf{SHA-256 NORMALIZE7} lookup table is used for mapping to sparse representation and range-constraining in the same way as for $\sigma_0$.

Constraints:
\begin{center}
    $w_{0, j + 0} = w_{2, j + 0} + w_{3, j + 0} \cdot 2^{6} + w_{4, j + 0} \cdot 2^{11} + w_{5, j + 0} \cdot 2^{25}$ \\
    $w_{0, j + 1} = w_{1, j + 0} + w_{2, j + 1} \cdot 7^{6} + w_{3, j + 1} \cdot 7^{11} + w_{4, j + 1} \cdot 7^{25}$ \\
    $w_{5, j + 1} + w_{6, j + 1} \cdot 4^{14} + w_{7, j + 1} \cdot 4^{28} + w_{8, j + 1} \cdot 2^{30} = w_{2, j + 1} + w_{3, j + 1} \cdot 4^{5} + w_{4, j + 1} \cdot 4^{19} + w_{1, j + 0} \cdot 2^{26} + w_{3, j + 1} + w_{4, j + 1} \cdot 4^{14} + w_{1, j + 0} \cdot 4^{21} + w_{2, j + 1} \cdot 4^{27} + w_{4, j + 1} + w_{1, j + 0} \cdot 4^{7}+ w_{2, j + 1} \cdot 4^{13} + w_{3, j + 1} \cdot 4^{27}$ \\
    $(w_{3, j + 1} - 3) \cdot (w_{3, j + 1} - 2) \cdot (w_{3, j + 1} - 1) \cdot w_{3, j + 1} = 0$
    $(w_{4, j + 1} - 3) \cdot (w_{4, j + 1} - 2) \cdot (w_{4, j + 1} - 1) \cdot w_{4, j + 1} = 0$
    11 plookup constraints: $(2^8 \cdot (w_{2, j + 0}, w_{1, j + 0}), (2^9 \cdot w_{3, j + 0}, w_{2, j + 1}), (2^7 \cdot w_{5, j + 0}, w_{4, j + 1}), (w_{2,j + 0}, w_{1,j+0}), (w_{3, j + 0}, w_{2, j + 1}), (w_{4, j + 0}, w_{3, j + 1}), (w_{5, j + 0}, w_{4, j + 1}), (w_{5, j + 2}, w_{5, j + 1}), (w_{6, j + 2}, w_{6, j + 1}), (w_{7, j + 2}, w_{6, j + 1}), (w_{8, j + 2}, w_{7, j + 1})$ \\
\end{center}

\paragraph{The Maj function}
contains subcircuit with base $4$ for $a, b ,c$.
\textbf{SHA-256 NORMALIZE MAJ} lookup table is used for mapping to sparse representation in the same way as for $\sigma_0$.
The value of the $ maj $ function is stored in chunks of $8$ bits $\{maj_0, maj_1, maj_2, maj_3 \}$ and the corresponded sparse value is $\{maj_{0, sparse}, maj_{1, sparse}, maj_{2, sparse}, maj_{3, sparse} \}$
Constraints:
\begin{center}
    $w_{0, j + 4} + w_{1, j + 4} \cdot 4^8 + w_{2, j + 4} \cdot 4^{8 \cdot 2} + w_{3, j + 4} \cdot 4^{8 \cdot 3} = w_{0, j + 5} + w_{1, j + 5} + w_{4, j + 5} $ \\
    4 plookup constraints: $( w_{5, j + 4}, w_{0, j + 4}), ( w_{6, j + 4}, w_{1, j + 4}), (w_{7, j + 4}, w_{2, j + 4}), (w_{8, j + 4}, w_{3, j + 4})$ \\
\end{center}

\paragraph{The Ch function}
contain sparse mapping subcircuit with base $7$ for $e, f ,g$.
\textbf{SHA-256 NORMALIZE CH} lookup table is used for mapping to sparse representation in the same way as for $\sigma_0$.
The value of the $ ch $ function is stored in chunks of $8$ bits $\{ch_0, ch_1, ch_2, ch_3 \}$ and the corresponded sparse value is $\{ch_{0, sparse}, ch_{1, sparse}, ch_{2, sparse}, ch_{3, sparse} \}$
Constraints:
\begin{center}
    $w_{0, j + 2} + w_{1, j + 2} \cdot 7^8 + w_{2, j + 2} \cdot 7^{8 \cdot 2} + w_{3, j + 2} \cdot 7^{8 \cdot 3} = w_{0, j + 1} + 2 \cdot w_{1, j + 1} + 3 \cdot w_{0, j + 3}$ \\
    4 plookup constraints: $( w_{5, j + 3}, w_{0, j + 2}), ( w_{6, j + 3}, w_{1, j + 2}), (w_{7, j + 3}, w_{2, j + 2}), (w_{8, j + 3}, w_{3, j + 2})$ \\
\end{center}

\paragraph{Update the values $a$ and $e$}
The value $W_r$ is a word, where $r$ is a number of round.
It has to be copy-constrained with the word $W_r$ in the message scheduling.
Constraints:
\begin{center}
    $w_{4, j + 3} = w_{1, j + 3} + w_{2, j + 3} + w_{5, j + 2} + w_{6, j + 2} \cdot 2^{14} + w_{7, j + 2} \cdot  2^{28} + w_{8, j + 2} \cdot 2^{30} + w_{5, j + 3} + w_{6, j + 3} \cdot 2^8 + w_{7, j + 3} \cdot 2^{8 \cdot 2} + w_{8, j + 3} \cdot 2^{8 \cdot 3} + k[r] + w_{3, j + 3}$, where $r$ is a number of round. \\
    $w_{4, j + 4} = w_{4, j + 3} - w_{1, j + 3} + w_{5, j + 5} + w_{6, j + 5} \cdot 2^{14} + w_{7, j + 5} \cdot  2^{28} + w_{8, j + 5} \cdot 2^{30} + w_{5, j + 4} + w_{6, j + 4} \cdot 2^8 + w_{7, j + 4} \cdot 2^{8 \cdot 2} + w_{8, j + 4} \cdot 2^{8 \cdot 3}$
\end{center}

\paragraph{Output of the round}\label{sha-out}
\begin{center}
    \begin{tabular}{ c|c|c|c|c|c|c|c|c|c }
        & $w_0$  & $w_1$  & $w_2$  & $w_3$  & $w_4$  & $w_5$ & $w_6$ & $w_7$ & $w_8$  \\
        \hline
        j + 0 & $\overline{a}$  & $\overline{b}$ & $\overline{c}$ & $\overline{d}$ & $\overline{e}$ & $\overline{f}$ & $-$ & $-$ & $-$ \\
        j + 1 & $h_0$ & $h_1$ & $h_2$ & $h_3$  & $h_4$  & $h_5$   & $-$ & $-$  & $-$   \\
        j + 2 & $a$ & $b$ & $c$ & $d$ & $e$ & $f$ & $-$ & $-$ & $-$\\
        j + 3 & $h_6$ & $h_7$ & $\overline{g}$ & $\overline{h}$ & $g$ & $h$ & $-$ & $-$ & $-$\\
    \end{tabular}
\end{center}
Evaluations:
\begin{center}
The values $\overline{\xi}$ copy constrained with initial working variables of this round.
The values  $a, b, c, d, e, f, g, h$ copy constrained with variables from the compression.
The output of the round is $h_0, h_1, .., h_7$
\end{center}
Constraints:
\begin{center}
    $w_{0, j + 1} = w_{0, j + 0} + w_{0, j + 2}$ \\
    $w_{1, j + 1} = w_{1, j + 0} + w_{1, j + 2}$ \\
    $w_{2, j + 1} = w_{2, j + 0} + w_{2, j + 2}$ \\
    $w_{3, j + 1} = w_{3, j + 0} + w_{3, j + 2}$ \\
    $w_{4, j + 1} = w_{4, j + 0} + w_{4, j + 2}$ \\
    $w_{5, j + 1} = w_{5, j + 0} + w_{5, j + 2}$ \\
    $w_{0, j + 3} = w_{2, j + 3} + w_{4, j + 3}$ \\
    $w_{1, j + 3} = w_{3, j + 3} + w_{5, j + 3}$ \\
\end{center}



\paragraph{Cost}
The total value of rows is $48 \cdot 5 + 8 \cdot 64 + 3 = 755$ per chunk.

\subsection{SHA-512 Circuit}
\label{section:sha512}
SHA-512 uses the similar logical functions as in \ref{section:sha256} which operates on $64$-bits words.
Thus, the preprocessing constraints for the SHA-512` circuit is a decomposition of $k$ message blocks to $64$ bits chunks without range proofs. For `eddsa` circuit, $k = 2$.
All evaluations are similar to SHA-256 circuit.

\paragraph{Lookup tables} 
We use the following lookup tables:
\begin{enumerate}
\item \textbf{SHA-256 NORMALIZE4} with 2 columns and $2^{14}$ rows. 
The first column contains all possible 14-bits words.
The second column contains corresponding sparse representations with base $4$.
The constraints can be used for the range check and sparse representation simultaneously.
\item \textbf{SHA-256 NORMALIZE7} with 2 columns and $2^{14}$ rows. 
The first column contains all possible 14-bits words.
The second column contains corresponding sparse representations with base $7$.
The constraints can be used for the range check and sparse representation simultaneously.
\item \textbf{SHA-512 NORMALIZE MAJ} with 2 columns and $2^{16}$ rows. 
The first column contains all possible 16-bits words.
The second column contains corresponding sparse representations with base $4$.
\item \textbf{SHA-512 NORMALIZE CH} with 2 columns and $2^{16}$ rows. 
The first column contains all possible 16-bits words.
The second column contains corresponding sparse representations with base $7$.
\end{enumerate}

\paragraph{Message scheduling}
For each block of $1024$ bits of the padded message the $80$ words are constructed in the following way:
\begin{itemize}
\item The first $16$ words are obtained by splitting the message.
\item The last $64$ words are obtained by using the functions $\sigma_0, \sigma_1$:
\begin{center}
$W_i = \sigma_1(W_{i-2}) \oplus W_{i - 7} \oplus \sigma_0(W_{i-15}) \oplus W_{i-16}$
\end{center}
\end{itemize}
Each round of the message scheduling has the following table:
\begin{center}
    \begin{tabular}{ c|c|c|c|c|c|c|c|c|c }
        & $w_0$  & $w_1$  & $w_2$  & $w_3$  & $w_4$  & $w_5$ & $w_6$ & $w_7$ & $w_8$  \\
        \hline
        j + 0 & $a$  & $a_0$  & $a_1$ & $a_2$ & $a_3$ & $ a_4$ & $a_5$ & $a_6$  & $\hat{a}_1$ \\
        j + 1 & $$  & $a'_0$  & $a'_1$ & $a'_2$ & $a'_3$ & $ a'_4$ & $a'_5$ & $a'_6$  & $$      \\
        j + 2 & $W_i$  & $s'_0$  & $s'_1$ & $s'_2$ & $s'_3$ & $ s'_4$ & $\hat{s}'_4$ & $--$  & $W_j$      \\
        j + 3 & $w$  & $s_0$  & $s_1$ & $s_2$ & $s_3$ & $ s_4$ & $$ & $$  & $$      \\
        j + 4 & $s_0$  & $s_1$  & $s_2$ & $s_3$  & $s_4$ & $s'_2$ & $s'_3$ & $ s'_4$ & $\hat{s}'_4$      \\
        j + 5 & $s'_0$  & $b'_0$  & $b'_1$ & $b'_2$ & $b'_3$ & $ b'_4$ & $b'_5$ & $s'_1$  & $-$  \\
        j + 6 & $b$  & $b_0$  & $b_1$ & $b_2$ & $b_3$ & $ b_4$ & $b_5$ & $\hat{b}_0$  & $\hat{b}_5$ \\
    \end{tabular}
\end{center}

The first $16$ words require a range check. 
We get it fo free from range-constraining chunks inside functions $\sigma_0$ and $\sigma_1$.
Thus, for $i$ from $16$ to $80$:
\begin{enumerate}
\item Apply $\sigma_0$ to $W_{i-15}$.
\item Add the following constraint for $W_i$:
\begin{center}
	$w_{0, j + 3} = w_{0, j + 2} + w_{8, j + 2} + w_{1, j + 3} + w_{2, j + 3} \cdot 2^{14} + w_{3, j + 3} \cdot 2^{28} + w_{4, j + 3} \cdot 2^{42} + w_{5, j + 3} \cdot 2^{56} + w_{0, j + 4} + w_{1, j + 4} \cdot 2^{14} + w_{2, j + 4} \cdot 2^{28} + w_{3, j + 4} \cdot 2^{42} + w_{4, j + 4} \cdot 2^{56}$,
\end{center}
\item Apply $\sigma_1$ to $W_{i-2}$.
\end{enumerate}
Thus, the message schedule takes $7 \cdot 64 = 448$ rows.

\paragraph{The function $\sigma_0$} contains sparse mapping with base $4$.
Let $a$ be divided to chunks $a_0, a_1, a_2, a_3, a_4, a_5, a_6$ which equals to $1, 6, 1, 14, 14, 14, 14$ bits respectively.
The values $a'_0, a'_1, a'_2, a'_3, a'_4, a'_5, a'_6$ are in sparse form, and $a'$ is a sparse $a$.
\textbf{SHA-256 NORMALIZE4} lookup table is used for mapping to sparse representation and range-constraining for each chunk $a_i$, where bit-length of $a_i > 3$.
If a chunk is $14$ bits long, then it is constrained for free.
Else the prover has to calculate the sparse representation $\hat{a}_i$ for $2^j \cdot a_i$, where $j + \texttt{len}(a_i) = 14$ and $\texttt{len}(a_i)$ is bit-length of $a_i$.

Constraints:
\begin{center}
    $w_{0, j + 0} = w_{1, j + 0} + w_{2,j + 0} \cdot 2 + w_{3,j + 0} \cdot 2^{7} + w_{4, j + 0} \cdot 2^{8} + w_{5, j + 0} \cdot 2^{22} + w_{6,  j + 0}\cdot 2^{36} + w_{7, j + 0} \cdot 2^{50}$ \\
    $(w_{1, j + 0} - 1) \cdot w_{1, j + 0} = 0$ \\
    $(w_{3, j + 0} - 1) \cdot w_{3, j + 0} = 0$ \\
    $w_{1, j + 2} + w_{2, j + 2} \cdot 4^{14} + w_{3, j + 2} \cdot 4^{28} + w_{4, j + 2} \cdot 2^{42} + w_{5, j + 2} \cdot 4^{56} = w_{2, j + 1} + w_{3, j + 1} \cdot 4^6 + w_{4, j + 1} \cdot 4^{7} + w_{5, j + 1} \cdot 2^{21} + w_{6, j + 1} \cdot 4^{35} + w_{7, j + 1} \cdot 4^{49} + w_{1, j + 1} \cdot 4^{63} + 
     w_{3, j + 1} + w_{4, j + 1} \cdot 4 + w_{5, j + 1} \cdot 4^{15} + w_{6, j + 1} \cdot 2^{29} + w_{7, j + 1} \cdot 4^{43} + 
     w_{4, j + 1} + w_{5, j + 1} \cdot 4^{14} + w_{6, j + 1} \cdot 4^{28} + w_{7, j + 1} \cdot 2^{42} + w_{1, j + 1} \cdot 4^{56} + w_{2, j + 1} \cdot 4^{57} + w_{3, j + 1} \cdot 4^{63}$ \\
    15 plookup constraints: $(w_{1, j + 0}, w_{1, j + 1}), ( 2^{8} \cdot w_{2, j + 0}, w_{8, j + 0}), (w_{2, j + 0}, w_{2, j + 1}), (w_{3, j + 0}, w_{3, j + 1}), (w_{4, j + 0}, w_{4, j + 1}), (w_{5, j + 0}, w_{5, j + 1}),  (w_{6, j + 0}, w_{6, j + 1}),  (w_{7, j + 0}, w_{7, j + 1}), (w_{1, j + 3}, w_{1, j + 2}), (w_{2, j + 3}, w_{2, j + 2}), (w_{3, j + 3}, w_{3, j + 2}), (w_{4, j + 3}, w_{4, j + 2}), (w_{5, j + 3}, w_{5, j + 2}), (2^6 \cdot w_{5, j + 3}, w_{6, j + 2})$
\end{center}

\paragraph{The function $\sigma_1$} contains sparse mapping subcircuit with base $4$.
Let $a$ be divided to chunks $a_0, a_1, a_2, a_3, a_4, a_5$ which equals to $6, 13, 14, 14, 14, 3$ bits respectively.
The values $a'_0, a'_1, a'_2, a'_3, a'_4, a'_5$ are in sparse form, and $a'$ is a sparse $a$.
\textbf{SHA-256 NORMALIZE4} lookup table is used for mapping to sparse representation and range-constraining in the same way as for $\sigma_0$.

Constraints:
\begin{center}
    $w_{0, j + 6} = w_{1, j + 6} + w_{2, j + 6} \cdot 2^{6} + w_{3, j + 6} \cdot 2^{19} + w_{4, j + 6} \cdot 2^{33} + w_{5, j + 6} \cdot 2^{47} + w_{6,  j + 6}\cdot 2^{61}$ \\
    $(w_{6, j + 6} - 7) \cdot (w_{6, j + 6} - 6) \cdot ... \cdot w_{6, j + 6} = 0$ \\
    $w_{0, j + 5} + w_{7, j + 5} \cdot 4^{14} + w_{5, j + 4} \cdot 4^{28} + w_{6, j + 4} \cdot 2^{42} + w_{7, j + 4} \cdot 4^{56} = w_{2, j + 5} + w_{3, j + 5} \cdot 4^{13} + w_{4, j + 5} \cdot 4^{27} + w_{5, j + 5} \cdot 2^{41} + w_{6, j + 5} \cdot 4^{55} +
     w_{3, j + 5} + w_{4, j + 5} \cdot 4^{14} + w_{5, j + 5} \cdot 4^{28} + w_{6, j + 5} \cdot 2^{42} + w_{1, j + 5} \cdot 4^{45} + w_{2, j + 5} \cdot 4^{51} + 
     w_{6, j + 5} + w_{1, j + 5} \cdot 4^{3} + w_{2, j + 5} \cdot 4^{9} + w_{3, j + 5} \cdot 2^{22} + w_{4, j + 5} \cdot 4^{36} + w_{5, j + 5} \cdot 4^{50}$ \\
    15 plookup constraints: $(w_{1, j + 6}, w_{1, j + 5}), ( 2^{8} \cdot w_{1, j + 6}, w_{7, j + 6}), (w_{2, j + 6}, w_{2, j + 5}), (w_{3, j + 6}, w_{3, j + 5}), (w_{4, j + 6}, w_{4, j + 5}), (w_{5, j + 6}, w_{5, j + 5}),  (w_{6, j + 6}, w_{6, j + 5}),  (2 \cdot w_{6, j + 6}, w_{8, j + 6}), (w_{0, j + 4}, w_{0, j + 5}), (w_{1, j + 4}, w_{7, j + 5}), (w_{2, j + 4}, w_{5, j + 4}), (w_{3, j + 4}, w_{6, j + 4}), (w_{4, j + 4}, w_{7, j + 4}), (2^6 \cdot w_{4, j + 4}, w_{8, j + 4})$
\end{center}

\paragraph{Compression}
There are $80$ rounds of compression. 
Each round of compression has the following table:
\begin{center}
    \begin{tabular}{ c|c|c|c|c|c|c|c|c|c }
        & $w_0$  & $w_1$  & $w_2$  & $w_3$  & $w_4$  & $w_5$ & $w_6$ & $w_7$ & $w_8$  \\
        \hline
        j + 0 & $e$ & $e_0$ & $e_1$ & $e_2 $  & $e_3$  & $e_4$ & $e_5$ & $\hat{e}_1$  & $\hat{e}_3$   \\
        j + 1 & $--$ & $e'_0$ & $e'_1$ & $e'_2 $  & $e'_3$  & $e'_4$ & $e'_5$ & $\hat{e}_5$  & $--$   \\
        j + 2 & $e'$  & $f'$ & $--$ & $\hat{s}'_4$ & $s'_0$ & $s'_1$ & $s'_2$ & $s'_3$ & $s'_4$ \\
        j + 3 & $ch_{0, sparse}$  & $ch_{1, sparse}$ & $ch_{2, sparse}$ & $ch_{3, sparse}$ & $s_0$ & $s_1$ & $s_2$ & $s_3$ & $s_4$ \\
        j + 4 & $g'$ & $--$ & $--$ & $--$ & $e_{new}$ & $ch_0$ & $ch_1$ & $ch_2$ & $ch_3$\\
        j + 5 & $c'$  & $d$ & $h$  & $W_r$  & $a_{new}$ & $maj_3$ & $maj_0$ & $maj_1$ & $maj_2$ \\
        j + 6 & $maj_{0,sparse}$  & $maj_{1,sparse}$ & $maj_{2,sparse}$  & $maj_{3,sparse}$  & $s_0$ & $s_1$ & $s_2$ & $s_3$ & $s_4$\\
        j + 7 & $a'$  & $b'$ & $--$ & $\hat{s}'_4$ & $s'_0$  & $s'_1$  & $s'_2$ & $s'_3$ & $s'_4$\\
        j + 8 & $$ & $a'_0$ & $a'_1$ & $a'_2$ & $a'_3$ & $a'_4$ & $a'_5$  & $ \hat{a}_5$ & $--$ \\
        j + 9 & $a$ & $a_0$ & $a_1$ & $a_2$ & $a_3$ & $a_4$ & $a_5$  & $ \hat{a}_2$ & $\hat{a}_3$ \\
    \end{tabular}
\end{center}

\paragraph{The working variables}
$ a,  b , c , d, e, f, g, h$ equals to the fixed initial $SHA-512$ values for the first chunk and to the sum of previous output and initial values for the rest of chunks. 
The variables with quotes are corresponded sparse representation.
For each chunk, the following rows are used:
\begin{center}
    \begin{tabular}{ c|c|c|c|c|c|c|c|c|c }
        & $w_0$  & $w_1$  & $w_2$  & $w_3$  & $w_4$  & $w_5$ & $w_6$ & $w_7$ & $w_8$  \\
        \hline
        j + 0 & $a$ & $a'$ & $b$ & $b' $  & $d$  & $-$   & $-$ & $-$  & $-$   \\
        j + 1 & $c$  & $c'$ & $e$ & $e'$ & $h$ & $-$ & $-$ & $-$ & $-$ \\
        j + 2 & $f$ & $f'$ & $g$ & $g'$ & $-$ & $-$ & $-$ & $-$ & $-$\\
    \end{tabular}
\end{center}
For the first round, $a, a', b', c' , d, e, e', f', g', h$ are copy constrained with corresponded values from the table above.

For the second round, $b', c' , d, f', g', h$ are copy constrained with $a', b', c, e', f', g$ from the table.
The values $a, e$ are copy constrained with $a_{new}, e_{new}$ from the previous round.

For the third round, $c' , d, g', h$ are copy constrained with $a', b, e', f$.
The values $a, e$ are copy constrained with $a_{new}, e_{new}$ from the previous round.
The values $b', f'$ are copy constrained with $a', e'$ from the previous round.

In the rest of the rounds the following `non-special` copy constraints are used:
\begin{enumerate}
\item The values $a, e$ are copy constrained with $a_{new}, e_{new}$ from the previous round.
\item The values $b', f'$ are copy constrained with $a', e'$ from the previous round.
\item The values $c', g'$ are copy constrained with $b', c'$ from the previous round.
\item The values $d, h$ are copy constrained with $a', e'$ from the round $r - 3$, where $r$ is current round.
\end{enumerate}

\paragraph{The $\Sigma_0$ function}
contains subcircuit with base $4$.
Let $a$ be divided to chunks $a_0, a_1, a_2, a_3, a_4, a_5$ which equals to $14, 14, 6, 5, 14, 11$ bits respectively.
The values $a'_0, a'_1, a'_2, a'_3, a'_4, a'_5$ are in sparse form, and $a'$ is a sparse $a$.
\textbf{SHA-256 NORMALIZE4} lookup table is used for mapping to sparse representation and range-constraining in the same way as for $\sigma_0$.

Constraints:
\begin{center}
    $w_{0, j + 9} = w_{1, j + 9} + w_{2, j + 9} \cdot 2^{14} + w_{3, j + 9} \cdot 2^{28} + w_{4, j + 9} \cdot 2^{34} + w_{5, j + 9} \cdot 2^{39} + w_{6,  j + 9}\cdot 2^{53}$ \\
    $w_{4, j + 7} + w_{5, j + 7} \cdot 4^{14} + w_{6, j + 7} \cdot 4^{28} + w_{7, j + 7} \cdot 2^{42} + w_{8, j + 7} \cdot 4^{56} = w_{3, j + 8} + w_{4, j + 8} \cdot 4^6 + w_{5, j + 8} \cdot 4^{11} + w_{6, j + 8} \cdot 2^{25} + w_{1, j + 8} \cdot 4^{36} + w_{2, j + 8} \cdot 4^{50} + 
     w_{4, j + 8} + w_{5, j + 8} \cdot 4^{5} + w_{6, j + 8} \cdot 4^{19} + w_{1, j + 8} \cdot 2^{30} + w_{2, j + 8} \cdot 4^{44} + w_{3, j + 8} \cdot 4^{58} +
     w_{5, j + 8} + w_{6, j + 8} \cdot 4^{14} + w_{1, j + 8} \cdot 4^{25} + w_{2, j + 8} \cdot 2^{39} + w_{3, j + 8} \cdot 4^{53} + w_{4, j + 8} \cdot 4^{59}$ \\
    15 plookup constraints: $(w_{1, j + 9}, w_{1, j + 8}), (w_{2, j + 9}, w_{2, j + 8}), ( 2^{8} \cdot w_{3, j + 9}, w_{7, j + 9}), (w_{3, j + 9}, w_{3, j + 8}), (2^9 \cdot w_{4, j + 9}, w_{8, j + 9}), (w_{4, j + 9}, w_{4, j + 8}), (w_{5, j + 9}, w_{5, j + 8}), (2^3 \cdot w_{6, j + 9}, w_{7, j + 8}), (w_{6, j + 9}, w_{6, j + 8}), (w_{4, j + 6}, w_{4, j + 7}), (w_{5, j + 6}, w_{5, j + 7}), (w_{6, j + 6}, w_{6, j + 7}), (w_{7, j + 6}, w_{7, j + 7}), (w_{8, j + 6}, w_{8, j + 7}), (2^6 \cdot w_{8, j + 7}, w_{3, j + 7})$
\end{center}
\paragraph{The $\Sigma_1$ function}
contains subcircuit with base $7$.
Let $a$ be divided to chunks $a_0, a_1, a_2, a_3, a_4, a_5$ which equals to $14, 4, 14, 9, 14, 9$ bits respectively.
The values $a'_0, a'_1, a'_2, a'_3, a'_4, a'_5$ are in sparse form, and $a'$ is a sparse $a$.
\textbf{SHA-256 NORMALIZE7} lookup table is used for mapping to sparse representation and range-constraining in the same way as for $\sigma_0$.

Constraints:
\begin{center}
    $w_{0, j + 0} = w_{1, j + 0} + w_{2, j + 0} \cdot 2^{14} + w_{3, j + 0} \cdot 2^{18} + w_{4, j + 0} \cdot 2^{32} + w_{5, j + 0} \cdot 2^{41} + w_{6,  j + 0}\cdot 2^{55}$ \\
    $w_{4, j + 2} + w_{5, j + 2} \cdot 4^{14} + w_{6, j + 2} \cdot 4^{28} + w_{7, j + 2} \cdot 2^{42} + w_{8, j + 2} \cdot 4^{56} = w_{2, j + 1} + w_{3, j + 1} \cdot 4^4 + w_{4, j + 1} \cdot 4^{18} + w_{5, j + 1} \cdot 2^{27} + w_{6, j + 1} \cdot 4^{41} + w_{1, j + 1} \cdot 4^{50} + 
     w_{3, j + 1} + w_{4, j + 1} \cdot 4^{14} + w_{5, j + 1} \cdot 4^{23} + w_{6, j + 1} \cdot 2^{37} + w_{1, j + 1} \cdot 4^{46} + w_{3, j + 1} \cdot 4^{60} +
     w_{5, j + 1} + w_{6, j + 1} \cdot 4^{14} + w_{1, j + 1} \cdot 4^{23} + w_{2, j + 1} \cdot 2^{37} + w_{3, j + 1} \cdot 4^{41} + w_{4, j + 1} \cdot 4^{55}$ \\
    15 plookup constraints: $(w_{1, j + 0}, w_{1, j + 1}), (w_{2, j + 0}, w_{2, j + 1}), ( 2^{10} \cdot w_{2, j + 0}, w_{7, j + 0}), (w_{3, j + 0}, w_{3, j + 1}), (2^5 \cdot w_{4, j + 0}, w_{8, j + 0}), (w_{4, j + 0}, w_{4, j + 1}), (w_{5, j + 0}, w_{5, j + 1}), (2^3 \cdot w_{6, j + 0}, w_{7, j + 1}), (w_{6, j + 0}, w_{6, j + 1}), (w_{4, j + 3}, w_{4, j + 2}), (w_{5, j + 3}, w_{5, j + 2}), (w_{6, j + 3}, w_{6, j + 2}), (w_{7, j + 3}, w_{7, j + 2}), (w_{8, j + 3}, w_{8, j + 2}), (2^6 \cdot w_{8, j + 3}, w_{3, j + 2})$
\end{center}

\paragraph{The Maj function}
contains subcircuit with base $4$ for $a, b ,c$.
\textbf{SHA-512 NORMALIZE MAJ} lookup table is used for mapping to sparse representation in the same way as for $\sigma_0$.
The value of the $ maj $ function is stored in chunks of $16$ bits.
Constraints:
\begin{center}
    $w_{0, j + 6} + w_{1, j + 6} \cdot 4^{16} + w_{2, j + 6} \cdot 4^{16 \cdot 2} + w_{3, j + 6} \cdot 4^{16 \cdot 3} = w_{0, j + 7} + w_{1, j + 7} + w_{0, j + 5} $ \\
    4 plookup constraints: $( w_{5, j + 5}, w_{0, j + 6}), ( w_{6, j + 5}, w_{1, j + 6}), (w_{7, j + 5}, w_{2, j + 6}), (w_{8, j + 5}, w_{3, j + 6})$ \\
\end{center}

\paragraph{The Ch function}
contain sparse mapping subcircuit with base $7$ for $e, f ,g$.
\textbf{SHA-512 NORMALIZE CH} lookup table is used for mapping to sparse representation in the same way as for $\sigma_0$.
The value of the $ch$ function is stored in chunks of $16$ bits.
Constraints:
\begin{center}
    $w_{0, j + 3} + w_{1, j + 3} \cdot 7^{16} + w_{2, j + 3} \cdot 7^{16 \cdot 2} + w_{3, j + 3} \cdot 7^{16 \cdot 3} = w_{0, j + 2} + 2 \cdot w_{1, j + 2} + 3 \cdot w_{0, j + 4}$ \\
    4 plookup constraints: $( w_{5, j + 4}, w_{0, j + 3}), ( w_{6, j + 4}, w_{1, j + 3}), (w_{7, j + 4}, w_{2, j + 3}), (w_{8, j + 3}, w_{3, j + 2})$ \\
\end{center}

\paragraph{Update the values $a$ and $e$}
Constraints:
\begin{center}
    $w_{4, j + 4} = w_{1, j + 5} + w_{2, j + 5} + w_{5, j + 3} \cdot 2^{14} + w_{6, j + 3} \cdot  2^{28} + w_{7, j + 3} \cdot 2^{42} + w_{8, j + 3} \cdot 2^{56} + w_{5, j + 4} + w_{6, j + 4} \cdot 2^{16} + w_{7, j + 4} \cdot 2^{16 \cdot 2} + w_{8, j + 4} \cdot 2^{16 \cdot 3} + k[r] + w_{3, j + 5}$, where $r$ is a number of round. \\
    $w_{4, j + 5} = w_{4, j + 4} - w_{1, j + 5} + w_{4, j + 6} + w_{5, j + 6} \cdot 2^{14} + w_{6, j + 6} \cdot  2^{28} + w_{7, j + 6} \cdot 2^{42} + w_{8, j + 6} \cdot 2^{56} + w_{5, j + 5} + w_{6, j + 5} \cdot 2^{16} + w_{7, j + 5} \cdot 2^{16 \cdot 2} + w_{8, j + 5} \cdot 2^{16 \cdot 3}$
\end{center}

\paragraph{Output of the round}
The final calculations uses the same table and constraints as in \ref{sha-out}.

\paragraph{Cost}
The total value of rows is $64 \cdot 7 + 10 \cdot 80 + 3 = 1248$ per chunk.
