\subsection{SHA-256 Circuit}
\label{section:sha256}
Suppose that input data is in the 32-bits form, which is already padded to the required size.
We suppose that the checking that chunked input data corresponds to the original data out of the circuit.
However, we do not need to range constrain these chunks as we get them for free from the SHA-256 circuit.


Thus, the preprocessing constraints for the SHA-256 circuit is a decomposition of $k$ message blocks to $32$ bits chunks without range proofs. For `Solana-EVM` circuit, $k = 3$.

\paragraph{Lookup tables}
We use the following lookup tables:
\begin{enumerate}
\item \textbf{SHA-256 NORMALIZE4} with 2 columns and $2^{14}$ rows. 
The first column contains all possible 14-bits words.
The second column contains corresponding sparse representations with base $4$.
The constraints can be used for the range check and sparse representation simultaneously.
\item \textbf{SHA-256 NORMALIZE7} with 2 columns and $2^{14}$ rows. 
The first column contains all possible 14-bits words.
The second column contains corresponding sparse representations with base $7$.
The constraints can be used for the range check and sparse representation simultaneously.
\item \textbf{SHA-256 NORMALIZE MAJ} with 2 columns and $2^{8}$ rows. 
The first column contains all possible 8-bits words.
The second column contains corresponding sparse representations with base $4$.
\item \textbf{SHA-256 NORMALIZE CH} with 2 columns and $2^{8}$ rows. 
The first column contains all possible 8-bits words.
The second column contains corresponding sparse representations with base $7$.
\end{enumerate}

\paragraph{Message scheduling}
For each block of $512$ bits of the padded message the 64 words are constructed in the following way:
\begin{itemize}
\item The first $16$ words are obtained by splitting the message.
\item The last $48$ words are obtained by using the functions $\sigma_0, \sigma_1$:
\begin{center}
$W_i = \sigma_1(W_{i-2}) \oplus W_{i - 7} \oplus \sigma_0(W_{i-15}) \oplus W_{i-16}$
\end{center}
\end{itemize}
Each round of the message scheduling has the following table:
\begin{center}
    \begin{tabular}{ c|c|c|c|c|c|c|c|c|c }
        & $w_1$  & $w_2$  & $w_3$  & $w_4$  & $w_5$  & $w_6$ & $w_7$ & $w_8$ & $w_9$  \\
        \hline
        j + 0 & $a$  & $a_0$  & $a_1$ & $a_2$ & $a_3$ & $ \hat{a}_1$ & $\hat{a}_2$ & $a'_0$  & $$      \\
        j + 1 & $W_i$ & $W_j$ & $a'_1$ & $a'_2$ & $a'_3$ & $s_0$ & $s_1$ & $s_2$ & $s_3$\\
        j + 2 & $w$ & $b'_0$ & $b'_1$ & $b'_2$ & $b'_3$ & $s_0$ & $s_1$ & $s_2$ & $s_3$\\
        j + 3 & $b$  & $b_0$  & $b_1$ & $b_2$ & $b_3$ & $ \hat{b}_0$ & $\hat{b}_1$ & $\hat{b}_3$  & $$ 
    \end{tabular}
\end{center}

The first $16$ words require a range check. 
We get it fo free from range-constraining chunks inside functions $\sigma_0$ and $\sigma_1$.
Thus, for $i$ from $16$ to $63$:
\begin{enumerate}
\item Apply $\sigma_0$ to $W_{i-15}$.
\item Add the following constraint for $W_i$:
\begin{center}
	$w_{1,j + 2} = w_{1, j + 1} + w_{2, j + 1} + w_{6, j + 1} + w_{7,j + 1} \cdot 2^3 + w_{8,j + 1} \cdot 2^{7} + w_{9, j + 1} \cdot 2^{18} + w_{6, j + 2} + w_{7, j + 2} \cdot 2^{10} + w_{8,j + 2} \cdot 2^{17} + w_{9, j + 2} \cdot 2^{19}$,
\end{center}
\item Apply $\sigma_1$ to $W_{i-2}$.
\end{enumerate}
Thus, the message schedule takes $4 \cdot 48 = 192$ rows.

\paragraph{The function $\sigma_0$} contains sparse mapping with base $4$.
Let $a$ be divided to chunks $a_0, a_1, a_2, a_3$ which equals to $3, 4, 11, 14$ bits respectively.
The values $a'_0, a'_1, a'_2, a'_3$ are in sparse form, and $a'$ is a sparse $a$.
\textbf{SHA-256 NORMALIZE4} lookup table is used for mapping to sparse representation and range-constraining for each chunk $a_i$, where bit-length of $a_i > 3$.
If a chunk is $14$ bits long, then it is constrained for free.
Else the prover has to calculate the sparse representation $\hat{a}_i$ for $2^j \cdot a_i$, where $j + \texttt{len}(a_i) = 14$ and $\texttt{len}(a_i)$ is bit-length of $a_i$.

Constraints:
\begin{center}
    $w_{1, j + 0} = w_{2, j + 0} + w_{3,j + 0} \cdot 2^3 + w_{4,j + 0} \cdot 2^{7} + w_{5, j + 0} \cdot 2^{18}$ \\
    $(w_{2, j + 0} - 7) \cdot (w_{2, j + 0} - 6) \cdot ... \cdot w_{2, j + 0} = 0$ \\
    10 plookup constraints: $(w_{2, j + 0}, w_{8, j + 0}), ( 2^{10} \cdot w_{3, j + 0}, w_{6, j + 0}), (w_{3, j + 0}, w_{3, j + 1}), (2^3 \cdot w_{4, j + 0}, w_{7, j + 0}), (w_{4, j + 0}, w_{4, j + 1}), (w_{5, j + 0}, w_{5, j + 1}), (w_{6, j + 1}, (w_{4, j + 1} + w_{5,j + 1} + w_{3,j + 1}), (w_{7, j + 1}, (w_{5,j + 1} + w_{8,j + 0} + w_{4, j + 1}, (w_{8, j + 1}, (w_{8,j + 0} + w_{3,j + 1} + w_{5,j + 1}), (w_{9,j + 1}, (w_{3,j + 1} + w_{4, j + 1})$ \\
\end{center}

\paragraph{The function $\sigma_1$} contains sparse mapping subcircuit with base $4$.
Let $a$ be divided to chunks $a_0, a_1, a_2, a_3$ which equals to $10, 7, 2, 13$ bits respectively.
The values $a'_0, a'_1, a'_2, a'_3$ are in sparse form and $a'$ is a sparse $a$.
\textbf{SHA-256 NORMALIZE4} lookup table is used for mapping to sparse representation and range-constraining in the same way as for $\sigma_0$.

Constraints:
\begin{center}
    $w_{1, j + 3} = w_{2, j + 3} + w_{3, j + 3} \cdot 2^{10} + w_{4,j + 3} \cdot 2^{17} + w_{5, j + 3} \cdot 2^{19}$ \\
    $(w_{4,j + 3} - 3) \cdot (w_{4,j + 3} - 2) \cdot (w_{4,j + 3} - 1) \cdot w_{4,j + 3} = 0$ \\
    11 plookup constraints: $(2^4 \cdot (w_{2, j + 3}, w_{6, j + 3}), (2^7 \cdot w_{3, j + 3}, w_{7,j+3}), (2 \cdot w_{5, j+3}, w_{8, j + 3}), (w_{2,j + 3}, w_{2,j+2}), (w_{3, j + 3}, w_{3, j + 2}), (w_{4,j + 3}, w_{4,j+2}), (w_{5, j + 3}, w_{5, j+2}), (w_{6, j + 2}, (w_{4, j + 2} + w_{5,j + 2} + w_{3,j + 2})), (w_{7, j + 2}, (w_{5,j + 2} + w_{2,j + 2} + w_{4,j + 2}, (w_{8,j + 2}, (w_{2,j + 2} + w_{3,j + 2} + w_{5,j + 2}), (w_{9,j + 2}, (w_{3,j + 2} + w_{4,j + 2})$ \\
\end{center}

\paragraph{Compression}
There are $64$ rounds of compression. 
Each round of compression has the following table:
\begin{center}
    \begin{tabular}{ c|c|c|c|c|c|c|c|c|c }
        & $w_1$  & $w_2$  & $w_3$  & $w_4$  & $w_5$  & $w_6$ & $w_7$ & $w_8$ & $w_9$  \\
        \hline
        j + 0 & $e$ & $e'_0$ & $e_0$ & $e_1 $  & $e_2$  & $e_3$   & $ \hat{e}_1$ & $\hat{e}_2$  & $\hat{e}_3$   \\
        j + 1 & $e'$  & $f'$ & $e'_1$ & $e'_2$ & $e'_3$ & $s_0$ & $s_1$ & $s_2$ & $s_3$ \\
        j + 2 & $ch_{0, sparse}$ & $ch_{1, sparse}$ & $ch_{2, sparse}$ & $ch_{3, sparse}$ & $e_{new}$ & $ch_0$ & $ch_1$ & $ch_2$ & $ch_3$\\
        j + 3 & $g'$  & $d$ & $h$  & $W_r$  & $a_{new}$ & $maj_3$ & $maj_0$ & $maj_1$ & $maj_2$ \\
        j + 4 & $maj_{0,sparse}$  & $maj_{1,sparse}$ & $maj_{2,sparse}$  & $maj_{3,sparse}$  & $c'$ & $s_0$ & $s_1$ & $s_2$ & $s_3$\\
        j + 5 & $a'$  & $b'$ & $a'_0$  & $a'_1$  & $a'_2$ & $a'_3$ & $$ & $$ & $$\\
        j + 6 & $a$ & $$ & $a_0$ & $a_1$ & $a_2$ & $a_3$ & $\hat{a}_0$  & $ \hat{a}_1$ & $\hat{a}_3$ \\
    \end{tabular}
\end{center}

\paragraph{The working variables}
$ a,  b , c , d, e, f, g, h$ equals to the fixed initial $SHA-256$ values for the first chunk and to the sum of previous output and initial values for the rest of chunks. 
The variables with quotes are corresponded sparse representation.
For each chunk, the following rows are used:
\begin{center}
    \begin{tabular}{ c|c|c|c|c|c|c|c|c|c }
        & $w_1$  & $w_2$  & $w_3$  & $w_4$  & $w_5$  & $w_6$ & $w_7$ & $w_8$ & $w_9$  \\
        \hline
        j + 0 & $a$ & $a'$ & $b$ & $b' $  & $d$  & $-$   & $-$ & $-$  & $-$   \\
        j + 1 & $c$  & $c'$ & $e$ & $e'$ & $h$ & $-$ & $-$ & $-$ & $-$ \\
        j + 2 & $f$ & $f'$ & $g$ & $g'$ & $-$ & $-$ & $-$ & $-$ & $-$\\
    \end{tabular}
\end{center}
For the first round, $a, a', b', c' , d, e, e', f', g', h$ are copy constrained with corresponded values from the table above.

For the second round, $b', c' , d, f', g', h$ are copy constrained with $a', b', c, e', f', g$ from the table.
The values $a, e$ are copy constrained with $a_{new}, e_{new}$ from the previous round.

For the third round, $c' , d, g', h$ are copy constrained with $a', b, e', f$.
The values $a, e$ are copy constrained with $a_{new}, e_{new}$ from the previous round.
The values $b', f'$ are copy constrained with $a', e'$ from the previous round.

In the rest of the rounds the following `non-special` copy constraints are used:
\begin{enumerate}
\item The values $a, e$ are copy constrained with $a_{new}, e_{new}$ from the previous round.
\item The values $b', f'$ are copy constrained with $a', e'$ from the previous round.
\item The values $c', g'$ are copy constrained with $b', c'$ from the previous round.
\item The values $d, h$ are copy constrained with $a', e'$ from the round $r - 3$, where $r$ is current round.
\end{enumerate}

\paragraph{The $\Sigma_0$ function}
contains subcircuit with base $4$.
Let $a$ be divided to chunks $a_0, a_1, a_2, a_3$ which equals to $2, 11, 9, 10$ bits respectively.
The values $a'_0, a'_1, a'_2, a'_3$ are in sparse form and $a'$ is a sparse $a$.
\textbf{SHA-256 NORMALIZE4} lookup table is used for mapping to sparse representation and range-constraining in the same way as for $\sigma_0$.

Constraints:
\begin{center}
    $w_{1, j + 6}= w_{3, j + 6} + w_{4, j + 6} \cdot 2^{2} + w_{5, j + 6} \cdot 2^{13} + w_{6, j + 6} \cdot 2^{22}$ \\
    $w_{1, j + 5} = w_{3, j + 6} + w_{4, j + 6} \cdot 4^{2} + w_{5, j + 6} \cdot 4^{13} + w_{6, j + 6} \cdot 4^{22}$ \\
    $(w_{3,j + 6} - 3) \cdot (w_{3,j + 6} - 2) \cdot (w_{3,j + 6} - 1) \cdot w_{3,j + 6} = 0$ \\
    11 plookup constraints: $(2^3 \cdot (w_{4, j + 6}, w_{7, j + 6}), (2^5 \cdot w_{5, j + 6}, w_{8, j + 6}), (2^4 \cdot w_{6, j+6}, w_{9, j + 6}), (w_{3,j + 6}, w_{3,j+5}), (w_{4, j + 6}, w_{4, j + 5}), (w_{5, j + 6}, w_{5, j + 5}), (w_{6, j + 6}, w_{6, j + 5}), (w_{6, j + 4}, (w_{4, j + 5} + w_{5,j + 5} + w_{6,j + 5}), (w_{7, j + 4}, (w_{5,j + 5} + w_{6,j + 5} + w_{3,j + 5}, (w_{8, j + 4}, (w_{6,j + 5} + w_{3,j + 5} + w_{4,j + 5}), (w_{9, j + 4}, (w_{3,j + 5} + w_{4,j + 5} + w_{5,j + 5})$ \\
\end{center}

\paragraph{The $\Sigma_1$ function}
contains subcircuit with base $7$.
Let $a$ be divided to chunks $a_0, a_1, a_2, a_3$ which equals to $6, 5, 14, 7$ bits respectively.
The values $a'_0, a'_1, a'_2, a'_3$ are in sparse form, and $a'$ is a sparse $a$.
\textbf{SHA-256 NORMALIZE7} lookup table is used for mapping to sparse representation and range-constraining in the same way as for $\sigma_0$.

Constraints:
\begin{center}
    $w_{1, j + 0} = w_{3, j + 0} + w_{4, j + 0} \cdot 2^{6} + w_{5, j + 0} \cdot 2^{11} + w_{6, j + 0} \cdot 2^{25}$ \\
    $w_{1, j + 1} = w_{2, j + 0} + w_{3, j + 1} \cdot 7^{6} + w_{4, j + 1} \cdot 7^{11} + w_{5, j + 1} \cdot 7^{25}$ \\
    11 plookup constraints: $(2^8 \cdot (w_{3, j + 0}, w_{2, j + 0}), (2^9 \cdot w_{4, j + 0}, w_{3, j + 1}), (2^7 \cdot w_{6, j + 0}, w_{5, j + 1}), (w_{3,j + 0}, w_{2,j+0}), (w_{4, j + 0}, w_{3, j + 1}), (w_{5, j + 0}, w_{4, j + 1}), (w_{6, j + 0}, w_{5, j + 1}), (w_{6, j + 1}, (w_{3, j + 1} + w_{4, j + 1} + w_{5,j + 1}), (w_{7, j + 1}, (w_{4,j + 1} + w_{5,j + 1} + w_{2, j + 0}, (w_{8, j + 1}, (w_{5,j + 1} + w_{2,j + 0} + w_{3,j + 1}), (w_{9, j + 1}, (w_{2,j + 0} + w_{3,j + 1} + w_{4,j + 1})$ \\
\end{center}

\paragraph{The Maj function}
contains subcircuit with base $4$ for $a, b ,c$.
\textbf{SHA-256 NORMALIZE MAJ} lookup table is used for mapping to sparse representation in the same way as for $\sigma_0$.
The value of the $ maj $ function is stored in chunks of $8$ bits.
Constraints:
\begin{center}
    $w_{1, j + 4} + w_{2, j + 4} \cdot 4^8 + w_{3, j + 4} \cdot 4^{8 \cdot 2} + w_{4, j + 4} \cdot 4^{8 \cdot 3} = w_{1, j + 5} + w_{2, j + 5} + w_{5, j + 4} $ \\
    4 plookup constraints: $( w_{6, j + 3}, w_{1, j + 4}), ( w_{7, j + 3}, w_{2, j + 4}), (w_{8, j + 3}, w_{3, j + 4}), (w_{9, j + 3}, w_{4, j + 4})$ \\
\end{center}

\paragraph{The Ch function}
contain sparse mapping subcircuit with base $7$ for $e, f ,g$.
\textbf{SHA-256 NORMALIZE CH} lookup table is used for mapping to sparse representation in the same way as for $\sigma_0$.
The value of the $ ch $ function is stored in chunks of $8$ bits.
Constraints:
\begin{center}
    $w_{1, j + 2} + w_{2, j + 2} \cdot 7^8 + w_{3, j + 2} \cdot 7^{8 \cdot 2} + w_{4, j + 2} \cdot 7^{8 \cdot 3} = w_{1, j + 1} + 2 \cdot w_{2, j + 1} + 3 \cdot w_{1, j + 3}$ \\
    4 plookup constraints: $( w_{6, j + 2}, w_{1, j + 2}), ( w_{7, j + 2}, w_{2, j + 2}), (w_{8, j + 2}, w_{3, j + 2}), (w_{9, j + 2}, w_{4, j + 2})$ \\
\end{center}

\paragraph{Update the values $a$ and $e$}
Constraints:
\begin{center}
    $w_{5, j + 2} = w_{2, j + 3} + w_{3, j + 3} + w_{6, j + 1} + w_{7, j + 1} \cdot 2^6 + w_{8, j + 1} \cdot  2^{11} + w_{9, j + 1} \cdot 2^{25} + w_{6, j + 2} + w_{7, j + 2} \cdot 2^8 + w_{8, j + 2} \cdot 2^{8 \cdot 2} + w_{9, j + 2} \cdot 2^{8 \cdot 3} + k[r] + w_{4, j + 3}$, where $r$ is a number of round. \\
    $w_{5, j + 3} = w_{5, j + 2} - w_{2, j + 3} + w_{6, j + 4} + w_{7, j + 4} \cdot 2^2 + w_{8, j + 4} \cdot  2^{13} + w_{9, j + 4} \cdot 2^{22} + w_{6, j + 3} + w_{7, j + 3} \cdot 2^8 + w_{8, j + 3} \cdot 2^{8 \cdot 2} + w_{9, j + 3} \cdot 2^{8 \cdot 3}$
\end{center}

\paragraph{Cost}
The total value of rows is $48 \cdot 4 + 7 \cdot 64 + 3 = 643$ per chunk.

\subsection{SHA2-512 Circuit}
\label{section:sha512}
SHA-512 uses the similar logical functions as in \ref{section:sha256} which operates on $64$-bits words.
Thus each input uses the same range proof which extended to 64-bits.

\paragraph{Range proof that $a < 2^{64}$}
Let $a = \{ a_0, ..., a_{32} \}$, where $a_i$ is two bits.
\begin{center}
    \begin{tabular}{ c|c|c|c|c|c }
        & $w_1$    & $w_2$     & $w_3$    & $w_4$    & $w_o$ \\
        \hline
        j + 0 & $a_{29}$ & $ a_{30}$ & $a_{31}$ & $a_{32}$ & acc   \\
        j + 1 & $a_{25}$ & $ a_{26}$ & $a_{27}$ & $a_{28}$ & acc   \\
        ...   &          &           &          &          &       \\
        j + 6 & $a_4$    & $a_5$     & $a_6$    & $a_7$    & acc   \\
        j + 7 & $a_0 $   & $a_1$     & $a_2$    & $a_3$    & a     \\
    \end{tabular}
\end{center}
Range gate constraints:
\begin{center}
    $w_{1,i}(w_{1,i}-1)(w_{1,i}-2)(w_{1,i} -3) + w_{2,i}(w_{2,i}-1)(w_{2,i}-2)(w_{2,i} -3)
    + w_{3,i}(w_{3,i}-1)(w_{3,i}-2)(w_{3,i} -3) + w_{4,i}(w_{4,i}-1)(w_{4,i}-2)(w_{4,i} -3)$ \\
    $w_{o,i} = w_{o, i - 1} \cdot 4^4 + w_{4,i} \cdot 4^3 + w_{3,i} \cdot 4^2 + w_{2,i} \cdot 4 + w_{1,i}$
\end{center}
The range proofs are included for each input data block.

\paragraph{The function $\sigma_0$} contain sparse mapping subcircuit with base $4$.
Let $a$ be divided to 8 bits-chunks $a_0, a_1, a_2, ..., a_7$.
The values $a'_0, a'_1, a'_2,...,  a'_7$ are in sparse form, and $a'$ is a sparse $a$.
We need the following lookup tables:
\begin{enumerate}
    \item \textbf{SHA-256 NORMALIZE4}: Read $a_i$ to $a'_i$
    \item \textbf{SHA-512 8ROT1 64}: Read $a'_0$ to $r_1$
    \item \textbf{SHA-512 8SHR7 64}: Read $a'_0$ to $r_3$
\end{enumerate}
\begin{center}
    \begin{tabular}{ c|c|c|c|c|c }
        & $w_1$  & $w_2$   & $w_3$  & $w_4$  & $w_o$      \\
        \hline
        j + 0 & $a_0$  & $ a_1$  & $a_2$  & $a_3$  & $a_4$      \\
        j + 1 & $a'_0$ & $a'_1$  & $a'_2$ & $a'_3$ & a          \\
        j + 2 & $a_5 $ & $a_6$   & $a_7$  & $a_4'$ & $\sigma_0$ \\
        j + 3 & $a'_5$ & $ a'_6$ & $a'_7$ & $r_1$  & $r_2$      \\
    \end{tabular}
\end{center}
Sparse map gate constraints:
\begin{center}
    $w_{o,j+1} = w_{1,j} + w_{2,j} \cdot 2^8 + w_{3,j} \cdot 2^{8 \cdot 2} + w_{4,j} \cdot 2^{8 \cdot 3}
    + w_{o,j} \cdot 2^{8 \cdot 4} + w_{1,j+2} \cdot 2^{8 \cdot 5} + w_{2,j+2} \cdot 2^{8 \cdot 6} + w_{3,j+2} \cdot 2^{8 \cdot 7}$ \\
    $w_{o,j+2} =  w_{2,j+1} \cdot 4^{8-1} + w_{3,j+1} \cdot 4^{8 \cdot 2-1} + w_{4,j+1} \cdot 4^{8 \cdot 3 - 1}
    + w_{4,j+2} \cdot 4^{8 \cdot 4 - 1} + w_{1,j+3} \cdot 4^{8 \cdot 5 - 1} + w_{2,j+3} \cdot 4^{8 \cdot 6 - 1}
    + w_{3,j+3} \cdot 4^{8 \cdot 7 - 1} + w_{1,j+1} \cdot 4^{8 \cdot 7} + w_{2,j+1} + w_{3,j+1} \cdot 4^{8}
    + w_{4,j+1} \cdot 4^{8 \cdot 2} +w_{4,j+2} \cdot 4^{8 \cdot 3} + w_{1,j+3} \cdot 4^{8 \cdot 4}
    + w_{2,j+3} \cdot 4^{8 \cdot 5} + w_{3,j+3} \cdot 4^{8 \cdot 6} + w_{2,j+1} \cdot 4^{8-7}
    + w_{3,j+1} \cdot 4^{8 \cdot 2-7} + w_{4,j+1} \cdot 4^{8 \cdot 3 - 7} + w_{4,j+2} \cdot 4^{8 \cdot 4 - 7}
    + w_{1,j+3} \cdot 4^{8*5 - 7} + w_{2,j+3} \cdot 4^{8 \cdot 6 - 7} + w_{3,j+3} \cdot 4^{8 \cdot 7 - 7}
    + w_{4, j+3} + w_{o, j+3}$ \\
    10 plookup constraints \\
\end{center}

\paragraph{The function $\sigma_1$} contain sparse mapping subcircuit with base $4$.
Let $a$ be divided to 8 bits-chunks $a_0, a_1, a_2, ..., a_7$.
The values $a'_0, a'_1, a'_2,...,  a'_7$ are in sparse form, and $a'$ is a sparse $a$.
We need the following lookup tables:
\begin{enumerate}
    \item \textbf{SHA-256 NORMALIZE4}: Read $a_i$ to $a'_i$
    \item \textbf{SHA-512 8ROT3 64}: Read $a'_2$ to $r_1$
    \item \textbf{SHA-512 8ROT5 SHR6 64}: Read $a'_7 + a'_0$ to $r_2$
\end{enumerate}
\begin{center}
    \begin{tabular}{ c|c|c|c|c|c }
        & $w_1$  & $w_2$   & $w_3$  & $w_4$  & $w_o$      \\
        \hline
        j + 0 & $a_0$  & $ a_1$  & $a_2$  & $a_3$  & $a_4$      \\
        j + 1 & $a'_0$ & $a'_1$  & $a'_2$ & $a'_3$ & a          \\
        j + 2 & $a_5 $ & $a_6$   & $a_7$  & $a_4'$ & $\sigma_1$ \\
        j + 3 & $a'_5$ & $ a'_6$ & $a'_7$ & $r_1$  & $r_2$      \\
    \end{tabular}
\end{center}
Sparse map gate constraints:
\begin{center}
    $w_{o,j+1} = w_{1,j} + w_{2,j} \cdot 2^8 + w_{3,j} \cdot 2^{8 \cdot 2} + w_{4,j} \cdot 2^{8 \cdot 3}
    + w_{o,j} \cdot 2^{8 \cdot 4} + w_{1,j+2} \cdot 2^{8 \cdot 5} + w_{2,j+2} \cdot 2^{8 \cdot 6} + w_{3,j+2} \cdot 2^{8 \cdot 7}$ \\
    $w_{o,j+2} =  w_{1,j+1} \cdot 4^{64 -19} + w_{2,j+1} \cdot 4^{64 + (8-19)}
    + w_{4,j+1} \cdot 4^{8 \cdot 3 - 19} + w_{4,j+2} \cdot 4^{8 \cdot 4 - 19}
    + w_{1,j+3} \cdot 4^{8 \cdot 5 - 19} + w_{2,j+3} \cdot 4^{8 \cdot 6 - 19}
    + w_{3,j+3} \cdot 4^{8 \cdot 7 -19} + w_{1,j+1} \cdot 4^{64  - 61)} + w_{2,j+1} \cdot 4^{64 + (8 - 61)}
    + w_{3,j+1} \cdot 4^{64 + (8 \cdot 2 - 61)} + w_{4,j+1} \cdot 4^{64 + (8 \cdot 3- 61)}
    + w_{4,j+2} \cdot 4^{64 + (8 \cdot 4 - 61)} + w_{1,j+3} \cdot 4^{64 + (8 \cdot 5 - 61)}
    + w_{2,j+3} \cdot 4^{64 +(8 \cdot 6- 61)} + w_{2,j+1} \cdot 4^{8-6} + w_{3,j+1} \cdot 4^{8 \cdot 2-6}
    + w_{4,j+1} \cdot 4^{8 \cdot 3 - 6} + w_{4,j+2} \cdot 4^{8 \cdot 4 - 6} + w_{1,j+3} \cdot 4^{8 \cdot 5 - 6}
    + w_{2,j+3} \cdot 4^{8 \cdot 6 - 6} + w_{3,j+3} \cdot 4^{8 \cdot 7 - 6} + w_{4, j+3} + w_{o, j+3}$ \\
    10 plookup constraints \\
\end{center}

The sparse values $\sigma_0$ and $\sigma_1$ have to be normalized.
The final addition requires one add gate.
Note, that $a'$ already initialized in the row $j - 2$.
We use \textbf{SHA256 NORMALIZE4}
\begin{center}
    \begin{tabular}{ c|c|c|c|c|c }
        & $w_1$  & $w_2$  & $w_3$  & $w_4$  & $w_o$      \\
        \hline
        j + 0 & $a'_0$ & $a'_1$ & $a'_2$ & $a'_3$ & $acc$      \\
        j + 1 & $a_0$  & $ a_1$ & $a_2$  & $a_3$  & 0          \\
        j + 2 & $a'_4$ & $a'_5$ & $a'_6$ & $a'_7$ & $\sigma_i$ \\
        j + 3 & $a_4$  & $ a_5$ & $a_6$  & $a_7$  &            \\
    \end{tabular}
\end{center}

Normalize gate constraints:
\begin{center}
    $w_{o,j+2} = w_{4,j+1} \cdot 256^3 + w_{3,j+1} \cdot 256^2 + w_{2,j+1} \cdot 256 + w_{1,j+1}
    + w_{1,j+3} \cdot 256^4$ \\
    $+ w_{2,j+3} \cdot 256^5+ w_{3,j+3} \cdot 256^6 + w_{4,j+4} \cdot 256^7$ \\
    $w_{o,j} = w_{o, j - 2} - (w_{4,j} \cdot 256^3 + w_{3,j} \cdot 256^2 + w_{2,j} \cdot 256 + w_{1,j})$ \\
    $w_{o,j+1} = w_{o,j} - ( w_{1,j+3} \cdot 256^4 + w_{2,j+3} \cdot 256^5+ w_{3,j+3} \cdot 256^6 + w_{4,j+4} \cdot 256^7)$

    8 plookup constraints \\
\end{center}

\paragraph{The $\Sigma_0$ function}
contain sparse mapping subcircuit with base $4$.
Let $a$ be divided to 7-bits chunks $a_0, a_1, a_2,a_3$ and 9 bits-chunks $ a_4, a_5, a_6, a_7$.
The values $a'_0, a'_1, a'_2,...,  a'_7$ are in sparse form, and $a'$ is a sparse $a$.
We need the following lookup tables:
\begin{enumerate}
    \item \textbf{SHA-512 9NORMALIZE4}: Read $a_i$ to $a'_i$
    \item \textbf{SHA-512 7NORMALIZE4}: Read $a_i$ to $a'_i$
    \item \textbf{SHA-512 9ROT6 64}: Read $a'_4$ to $r_2$
    \item \textbf{SHA-512 9ROT2 64}: Read $a'_5$ to $r_3$
\end{enumerate}
\begin{center}
    \begin{tabular}{ c|c|c|c|c|c }
        & $w_1$  & $w_2$   & $w_3$  & $w_4$  & $w_o$      \\
        \hline
        j + 0 & $a_0$  & $ a_1$  & $a_2$  & $a_3$  & $a_4$      \\
        j + 1 & $a'_0$ & $a'_1$  & $a'_2$ & $a'_3$ & a          \\
        j + 2 & $a_5 $ & $a_6$   & $a_7$  & $a_4'$ & $\Sigma_0$ \\
        j + 3 & $a'_5$ & $ a'_6$ & $a'_7$ & $r_1$  & $r_2$      \\
    \end{tabular}
\end{center}

Sparse map gate constraints:
\begin{center}
    $w_{o,j+1} = w_{1,j} + w_{2,j} \cdot 2^7 + w_{3,j} \cdot 2^{7 \cdot 2} + w_{4,j} \cdot 2^{7 \cdot 3}
    + w_{o,j} \cdot 2^{7 \cdot 4} + w_{1,j+2} \cdot 2^{7 \cdot 4 + 9}
    + w_{2,j+2} \cdot 2^{7 \cdot 4 + 9 \cdot 2} + w_{3,j+2} \cdot 2^{7 \cdot 4 + 9 \cdot 3}$ \\
    $w_{o,j+2} =  w_{4,j+2} + w_{1,j+3} \cdot 4^9 + w_{2,j+3} \cdot 4^{9 \cdot 2}
    + w_{3,j+3} \cdot 4^{9 \cdot 3} + w_{1,j+1} \cdot 4^{9 \cdot 4} + w_{2,j+1} \cdot 4^{9 \cdot 4 +7}$ \\
    $+ w_{3,j+1} \cdot 4^{9 \cdot 4 +7 \cdot 2} + w_{4,j+1} \cdot 4^{9 \cdot 4 +7 \cdot 3}
    + w_{1,j+1} \cdot 4^{64  - 34)} + w_{2,j+1} \cdot 4^{64 + (7 - 34)}
    + w_{3,j+1} \cdot 4^{64 + (7 \cdot 2 - 34)} + w_{4,j+1} \cdot 4^{64 + (7 \cdot 3- 34)}
    + w_{1,j+3} \cdot 4^{7 \cdot 4 + 9 - 34} + w_{2,j+3} \cdot 4^{7 \cdot 4 + 9  \cdot 2  -34}
    + w_{3,j+3} \cdot 4^{7 \cdot 4 + 9 \cdot 3 - 34} + w_{1,j+1} \cdot 4^{64  - 39)}
    + w_{2,j+1} \cdot 4^{64 + (7 - 39)} + w_{3,j+1} \cdot 4^{64 + (7 \cdot 2 - 39)}
    + w_{4,j+1} \cdot 4^{64 + (7 \cdot 3- 39)} +w_{4,j+2} \cdot 4^{64 + (7 \cdot 4 - 39)}
    + w_{2,j+3} \cdot 4^{7 \cdot 4 + 9 \cdot 2 - 39}
    + w_{3,j+3} \cdot 4^{7 \cdot 4 + 9 \cdot 3- 39} + w_{4, j+3} + w_{o, j+3}$ \\
    10 plookup constraints \\
\end{center}

\paragraph{The $\Sigma_1$ function}
contain sparse mapping subcircuit with base $7$.
Let $a$ be divided to 7-bits chunks $a_0, a_1, a_2,a_3$ and 9 bits-chunks $ a_4, a_5, a_6, a_7$.
The values $a'_0, a'_1, a'_2,...,  a'_7$ are in sparse form, and $a'$ is a sparse $a$.
We need the following lookup tables:
\begin{enumerate}
    \item \textbf{SHA-512 9NORMALIZE7}: Read $a_i$ to $a'_i$
    \item \textbf{SHA-512 7NORMALIZE7}: Read $a_i$ to $a'_i$
    \item \textbf{SHA-512 7ROT4 32}: Read $a'_2$ to $r_2$
    \item \textbf{SHA-512 9ROT4 32}: Read $a'_5$ to $r_3$
\end{enumerate}
\begin{center}
    \begin{tabular}{ c|c|c|c|c|c }
        & $w_1$  & $w_2$   & $w_3$  & $w_4$  & $w_o$      \\
        \hline
        j + 0 & $a_0$  & $ a_1$  & $a_2$  & $a_3$  & $a_4$      \\
        j + 1 & $a'_0$ & $a'_1$  & $a'_2$ & $a'_3$ & a          \\
        j + 2 & $a_5 $ & $a_6$   & $a_7$  & $a_4'$ & $\Sigma_1$ \\
        j + 3 & $a'_5$ & $ a'_6$ & $a'_7$ & $r_1$  & $r_2$      \\
    \end{tabular}
\end{center}

Sparse map gate constraints:
\begin{center}
    $w_{o,j+1} = w_{1,j} + w_{2,j} \cdot 2^7 + w_{3,j} \cdot 2^{7 \cdot 2} + w_{4,j} \cdot 2^{7 \cdot 3}
    + w_{o,j} \cdot 2^{7 \cdot 4} + w_{1,j+2} \cdot 2^{7 \cdot 4 + 9}
    + w_{2,j+2} \cdot 2^{7 \cdot 4 + 9 \cdot 2} + w_{3,j+2} \cdot 2^{7 \cdot 4 + 9 \cdot 3}$ \\
    $w_{o,j+2} =  w_{3,j+1} + w_{4,j+1} \cdot 7^7 + w_{4,j+2} \cdot 7^{7 \cdot 2}
    + w_{1,j+3} \cdot 7^{7 \cdot 2+9} + w_{2,j+3} \cdot 7^{7 \cdot 2+9 \cdot 2}
    + w_{3,j+3} \cdot 7^{9 \cdot 3 +7 \cdot 2} + w_{1,j+1} \cdot 7^{9 \cdot 4 +7 \cdot 2}
    + w_{2,j+1} \cdot 7^{9 \cdot 4 +7 \cdot 3} + w_{1,j+1} \cdot 7^{64  - 18)}
    + w_{2,j+1} \cdot 7^{64 + (7 - 18)} + w_{4,j+1} \cdot 7^{7 \cdot 3 - 18}
    + w_{4,j+2} \cdot 7^{7 \cdot 4- 18}
    + w_{1,j+3} \cdot 7^{7 \cdot 4 + 9  - 18}
    + w_{2,j+3} \cdot 7^{7 \cdot 4 + 9 \cdot 2 - 18}
    + w_{3,j+3} \cdot 7^{7 \cdot 4 + 9 \cdot 3 - 18} + w_{1,j+1} \cdot 7^{64  - 41)}
    + w_{2,j+1} \cdot 7^{64 + (7 - 41)} + w_{3,j+1} \cdot 7^{64 + (7 \cdot 2 - 41)}
    + w_{4,j+1} \cdot 7^{64 + (7 \cdot 3- 41)} +w_{4,j+2} \cdot 7^{64 + (7 \cdot 3 + 9 - 41)}
    + w_{2,j+3} \cdot 7^{64 + (7 \cdot 3 + 9 \cdot 2 -41)}
    + w_{3,j+3} \cdot 7^{7 \cdot 3 + 9 \cdot 3- 41} + w_{4, j+3} + w_{o, j+3}$ \\
    10 plookup constraints \\
\end{center}

The sparse values $\Sigma_0$ and $\Sigma_1$ have to be normalized.
We use \textbf{SHA256 NORMALIZE4} and \textbf{SHA256 NORMALIZE7}.
Note, that $a'$ already initialized in the row $j - 2$.
\begin{center}
    \begin{tabular}{ c|c|c|c|c|c }
        & $w_1$  & $w_2$  & $w_3$  & $w_4$  & $w_o$      \\
        \hline
        j + 0 & $a'_0$ & $a'_1$ & $a'_2$ & $a'_3$ & $a''$      \\
        j + 1 & $a_0$  & $ a_1$ & $a_2$  & $a_3$  & $0$        \\
        j + 2 & $a'_4$ & $a'_5$ & $a'_6$ & $a'_7$ & $\Sigma_i$ \\
        j + 3 & $a_4$  & $ a_5$ & $a_6$  & $a_7$  &            \\
    \end{tabular}
\end{center}

Normalize gate constraints:
\begin{center}
    $w_{o,j+2} = w_{4,j+1} \cdot 256^3 + w_{3,j+1} \cdot 256^2 + w_{2,j+1} \cdot 256
    + w_{1,j+1} + w_{1,j+3} \cdot 256^4$ \\
    $+ w_{2,j+3} \cdot 256^5+ w_{3,j+3} \cdot 256^6 + w_{4,j+4} \cdot 256^7$ \\
    $w_{o,j} = w_{1,j-3} + w_{2,j-3} \cdot 4^7 + w_{3,j-3} \cdot 4^{7 \cdot 2}
    + w_{4,j-3} \cdot 4^{7 \cdot 3} + w_{4,j-2} \cdot 4^{7 \cdot 4}
    + w_{1,j-1} \cdot 7^{7 \cdot 4+9}$ \\
    $+ w_{2,j-1} \cdot 7^{7 \cdot 4 + 9 \cdot 2}
    + w_{2,j-1} \cdot 7^{7 \cdot 4 + 9 \cdot 3}$ for maj or ch function. For $\Sigma_1$ replace 4 with 7\\
    $w_{o,j+1} = w_{o, j - 2} - (w_{4,j} \cdot 256^3 + w_{3,j} \cdot 256^2 + w_{2,j} \cdot 256 + w_{1,j} + w_{1,j+3} \cdot 256^4 + w_{2,j+3} \cdot 256^5+ w_{3,j+3} \cdot 256^6 + w_{4,j+4} \cdot 256^7)$

    8 plookup constraints \\
\end{center}

\paragraph{The Maj function}
contain sparse mapping subcircuit with base $4$ for $a, b ,c$.
Note, that the sparse chunks of $a$ we already have in $\Sigma_0$ in the circuit.
The variables $b$ and $c$ were represented in sparse chunks in the previous rounds or it is public inputs.
\begin{center}
    \begin{tabular}{ c|c|c|c|c|c }
        & $w_1$ & $w_2$ & $w_3$ & $w_4$ & $w_o$ \\
        \hline
        j & $a'$  & $b'$  & $c'$  &       & maj   \\
    \end{tabular}
\end{center}
Sparse map gate constraints:
\begin{center}
    $w_{o, j} = w_{1,j} + w_{2, j} + w_{3, j}$ \\
\end{center}

The sparse values $maj$ have to be normalized.
We use \textbf{SHA256 MAJ NORMALIZE4}
Note, that the sparse $maj$ already initialized in the row $j - 1$.
\begin{center}
    \begin{tabular}{ c|c|c|c|c|c}
        & $w_1$  & $w_2$  & $w_3$  & $w_4$  & $w_o$ \\
        \hline
        j + 0 & $a'_0$ & $a'_1$ & $a'_2$ & $a'_3$ & $acc$ \\
        j + 1 & $a_0$  & $ a_1$ & $a_2$  & $a_3$  & 0     \\
        j + 2 & $a'_4$ & $a'_5$ & $a'_6$ & $a'_7$ & $maj$ \\
        j + 3 & $a_4$  & $ a_5$ & $a_6$  & $a_7$  &       \\
    \end{tabular}
\end{center}

Normalize gate constraints:
\begin{center}
    $w_{o,j+2} = w_{4,j+1} \cdot 256^3 + w_{3,j+1} \cdot 256^2 + w_{2,j+1} \cdot 256 + w_{1,j+1}
    + w_{1,j+3} \cdot 256^4$ \\
    $+ w_{2,j+3} \cdot 256^5+ w_{3,j+3} \cdot 256^6 + w_{4,j+4} \cdot 256^7$ \\
    $w_{o,j} = w_{o, j - 1} - (w_{4,j} \cdot 256^3 + w_{3,j} \cdot 256^2 + w_{2,j} \cdot 256 + w_{1,j})$ \\
    $w_{o,j+1} = w_{o,j} - ( w_{1,j+3} \cdot 256^4 + w_{2,j+3} \cdot 256^5+ w_{3,j+3} \cdot 256^6 + w_{4,j+4} \cdot 256^7)$

    8 plookup constraints \\
\end{center}


The final addition requires one add gate.

\paragraph{The Ch function}
contain sparse mapping subcircuit with base $7$ for $e, f ,g$.
Note, that $e$ we already have in the sparse from $\Sigma_1$ in the circuit.
The variables $f$ and $g$ were represented in sparse form in the previous rounds or it is public inputs.
\begin{center}
    \begin{tabular}{ c|c|c|c|c|c }
        & $w_1$ & $w_2$ & $w_3$ & $w_4$ & $w_o$ \\
        \hline
        j + 0 & $e'$  & $f'$  & $g'$  &       & ch    \\
    \end{tabular}
\end{center}
Sparse map gate constraints:
\begin{center}
    $w_{o, j} = w_{1,j} + 2 \cdot w_{2, j} + 3 \cdot w_{3, j}$ \\
\end{center}

The sparse values $ch$ have to be normalized.
Note, that $ch$ already initialized in the row $j - 1$.
We use \textbf{SHA256 CH NORMALIZE7}
\begin{center}
    \begin{tabular}{ c|c|c|c|c|c }
        & $w_1$  & $w_2$  & $w_3$  & $w_4$  & $w_o$ \\
        \hline
        j + 0 & $a'_0$ & $a'_1$ & $a'_2$ & $a'_3$ & $acc$ \\
        j + 1 & $a_0$  & $ a_1$ & $a_2$  & $a_3$  & 0     \\
        j + 2 & $a'_4$ & $a'_5$ & $a'_6$ & $a'_7$ & $ch$  \\
        j + 3 & $a_4$  & $ a_5$ & $a_6$  & $a_7$  &       \\
    \end{tabular}
\end{center}

Normalize gate constraints:
\begin{center}
    $w_{o,j+2} = w_{4,j+1} \cdot 256^3 + w_{3,j+1} \cdot 256^2 + w_{2,j+1} \cdot 256 + w_{1,j+1}
    + w_{1,j+3} \cdot 256^4 + w_{2,j+3} \cdot 256^5$ \\
    $+ w_{3,j+3} \cdot 256^6 + w_{4,j+4} \cdot 256^7$ \\
    $w_{o,j} = w_{o, j - 1} - (w_{4,j} \cdot 256^3 + w_{3,j} \cdot 256^2 + w_{2,j} \cdot 256 + w_{1,j})$ \\
    $w_{o,j+1} = w_{o,j} - ( w_{1,j+3} \cdot 256^4 + w_{2,j+3} \cdot 256^5+ w_{3,j+3} \cdot 256^6 + w_{4,j+4} \cdot 256^7)$ \\
    8 plookup constraints \\
\end{center}

The final addition requires one add gate.

The updating of variables for new rounds costs 10 add gates.

Producing the final hash value costs two add gates.
