\subsection{SHA-256 Circuit}
\label{section:sha256}
Suppose that input data is in the 32-bits form, which is already padded to the required size.
We suppose that the checking that chunked input data corresponds to the original data out of the circuit.
However, we do not need to range constrain these chunks as we get them for free from the SHA-256 circuit.


Thus, the preprocessing constraints for the SHA-256 circuit is a decomposition of $k$ message blocks to $32$ bits chunks without range proofs. For `Solana-EVM` circuit, $k = 3$.

\paragraph{Lookup tables}
We use the following lookup tables:
\begin{enumerate}
\item \textbf{SHA-256 NORMALIZE4} with 2 columns and $2^{14}$ rows. 
The first column contains all possible 14-bits words.
The second column contains corresponding sparse representations.
The constraints can be used for the range check and sparse representation simultaneously.
\end{enumerate}

\paragraph{The function $\sigma_0$} contains sparse mapping with base $4$.
Let $a$ be divided to chunks $a_0, a_1, a_2, a_3$ which equals to $3, 4, 11, 14$ bits respectively.
The values $a'_0, a'_1, a'_2, a'_3$ are in sparse form, and $a'$ is a sparse $a$.
\textbf{SHA-256 NORMALIZE4} lookup table is used for mapping to sparse representation and range-constraining for each chunk $a_i$, where bit-length of $a_i > 3$.
If a chunk is $14$ bits long, then it is constrained for free.
Else the prover has to calculate the sparse representation $\hat{a}_i$ for $2^j \cdot a_i$, where $j + \texttt{len}(a_i) = 14$ and $\texttt{len}(a_i)$ is bit-length of $a_i$.

\begin{center}
    \begin{tabular}{ c|c|c|c|c|c }
        & $w_1$  & $w_2$  & $w_3$  & $w_4$  & $w_o$      \\
        \hline
        j + 0 & $a_0$  & $ \hat{a}_1$ & $\hat{a}_2$  & $a_3$  & a          \\
        j + 1 & $a'_0$ & $a_1$ & $a_2$ & $a'_3$ & $-$\\
        j + 2 & $s_0$ & $a'_1$ & $a'_2$ & $s_3$ & $\sigma_0$ \\
        j + 3 & $b$  & $ s_1$ & $s_2$  & $c$  & $w$ \\
    \end{tabular}
\end{center}
Constraints:
\begin{center}
    $w_{o,j} = w_{1,j} + w_{2,j + 1} \cdot 2^3 + w_{3,j + 1} \cdot 2^{7} + w_{4,j} \cdot 2^{18}$ \\
    $w_{o, j+3} = w_{1,j + 2} + w_{2,j + 3} \cdot 2^3 + w_{3,j + 3} \cdot 2^{7} + w_{4,j +2} \cdot 2^{18}$ \\
    $(w_{1,j} - 7) \cdot (w_{1,j} - 6) \cdot ... \cdot w_{1,j} = 0$ \\
    10 plookup constraints: $(w_{1,j}, w_{1,j+1}), ( 2^10 \cdot w_{2,j+1}, w_{2,j}), (w_{2, j+1}, w_{2, j + 2}), (2^3 \cdot w_{3,j+1}, w_{3,j}), (w_{3, j + 1}, w_{3, j + 2}), (w_{4,j}, w_{4,j+1}), (w_{1, j + 2}, (w_{3, j + 2} + w_{4,j + 1} + w_{2,j + 2}), (w_{2, j + 3}, (w_{4,j + 1} + w_{1,j + 1} + w_{3,j + 2}, (w_{3,j + 3}, (w_{1,j + 1} + w_{2,j + 2} + w_{4,j + 1}), (w_{4,j + 2}, (w_{2,j + 2} + w_{3,j + 2})$ \\
\end{center}

\paragraph{The function $\sigma_1$} contains sparse mapping subcircuit with base $4$.
Let $a$ be divided to chunks $a_0, a_1, a_2, a_3$ which equals to $10, 7, 2, 13$ bits respectively.
The values $a'_0, a'_1, a'_2, a'_3$ are in sparse form and $a'$ is a sparse $a$.
\textbf{SHA-256 NORMALIZE4} lookup table is used for mapping to sparse representation and range-constraining in the same way as for $\sigma_0$.

\begin{center}
    \begin{tabular}{ c|c|c|c|c|c }
        & $w_1$  & $w_2$  & $w_3$  & $w_4$  & $w_o$      \\
        \hline
        j + 0& $s_0$  & $ s_1$ & $-$  & $s_3$  & $\sigma_1$ \\
        j + 1 & $a'_0$ & $a'_1$ & $s_2$ & $a'_3$ & $-$ \\
        j + 2 & $a_0$ & $a_1$ & $a'_2$ & $a_3$ & $-$\\
        j + 3 & $\hat{a}_0$  & $ \hat{a}_1$ & $a_2$  & $\hat{a}_3$  & a\\
    \end{tabular}
\end{center}
Constraints:
\begin{center}
    $w_{o,j} = w_{1,j} + w_{2,j} \cdot 2^{10} + w_{3,j + 1} \cdot 2^{17} + w_{4,j} \cdot 2^{19}$ \\
    $w_{o, j+3} = w_{1,j + 2} + w_{2,j + 2} \cdot 2^{10} + w_{3,j + 3} \cdot 2^{17} + w_{4,j +2} \cdot 2^{19}$ \\
    $(w_{3,j + 3} - 3) \cdot (w_{3,j + 3} - 2) \cdot ... \cdot w_{3,j + 3} = 0$ \\
    11 plookup constraints: $(2^4 \cdot (w_{1,j + 2}, w_{1,j+1}), (2^7 \cdot w_{2,j + 2}, w_{2,j+3}), (2 \cdot w_{3, j+2}, w_{3, j + 3}), (w_{1,j + 2}, w_{1,j+1}), (w_{2, j + 2}, w_{2, j + 1}), (w_{3,j + 3}, w_{3,j+2}), (w_{4, j + 2}, w_{4, j+1}),(w_{1, j}, (w_{3, j + 2} + w_{4,j + 1} + w_{2,j + 1}), (w_{2, j}, (w_{4,j + 1} + w_{1,j + 1} + w_{3,j + 2}, (w_{3,j + 1}, (w_{1,j + 1} + w_{3,j + 2} + w_{4,j + 1}), (w_{4,j }, (w_{2,j + 1} + w_{3,j + 2})$ \\
\end{center}

\paragraph{Message scheduling}
For each block of $512$ bits of the padded message the 64 words are constructed in the following way:
\begin{itemize}
\item The first $16$ words are obtained by splitting the message.
\item The last $48$ words are obtained by using the functions $\sigma_0, \sigma_1$:
\begin{center}
$W_i = \sigma_1(W_{i-2}) \oplus W_{i - 7} \oplus \sigma_0(W_{i-15}) \oplus W_{i-16}$
\end{center}
\end{itemize}
The first $16$ words require a range check. 
We get it fo free from range-constraining chunks inside functions $\sigma_0$ and $\sigma_1$.
Thus, for $i$ from $16$ to $63$:
\begin{enumerate}
\item Apply $\sigma_0$ to $W_{i-15}$. It costs $4$ rows.
\item Add the following constraint for $W_i$:
\begin{center}
	$w_{o,3} = w_{1, 3} + w_{4,3} + w_{o, 2} + w_{o, 4}$,
\end{center}
	where rows $2$ and $3$ belong to the $\sigma_0$ table and row $4$ is the first row of the $\sigma_1$ table. This constraint does not require additional rows. 
\item Apply $\sigma_1$ to $W_{i-2}$. It costs $4$ rows.
\end{enumerate}
Thus, the message schedule takes $8 \cdot 48 = 384$ rows.

\paragraph{The $\Sigma_0$ function}
contain sparse mapping subcircuit with base $2$.
Let $a$ be divided to 8 bits-chunks $a_0, a_1, a_2, a_3$.
The values $a'_0, a'_1, a'_2, a'_3$ are in sparse form, and $a'$ is a sparse $a$.
We need the following lookup tables:
\begin{enumerate}
    \item \textbf{SHA-256 NORMALIZE4}: Read $a_i$ to $a'_i$
    \item \textbf{SHA-256 8ROT2 32}: Read $a'_0$ to $r_1$
    \item \textbf{SHA-256 8ROT5 32}: Read $a'_1$ to $r_2$
    \item \textbf{SHA-256 8ROT6 32}: Read $a'_2$ to $r_3$
\end{enumerate}
\begin{center}
    \begin{tabular}{ c|c|c|c|c|c }
        & $w_1$  & $w_2$  & $w_3$  & $w_4$  & $w_o$      \\
        \hline
        j + 0 & $a_0$  & $ a_1$ & $a_2$  & $a_3$  & a          \\
        j + 1 & $a'_0$ & $a'_1$ & $a'_2$ & $a'_3$ & a'         \\
        j + 2 & $r1 $  & $r_2$  & $r_3$  &        & $\Sigma_0$ \\
    \end{tabular}
\end{center}
Sparse map gate constraints:
\begin{center}
    $w_{o,j} = w_{1,j} + w_{2,j} \cdot 2^8 + w_{3,j} \cdot 2^{8 \cdot 2} + w_{4,j} \cdot 2^{8 \cdot 3}$ \\
    $w_{o,j+1} = w_{1,j+1} + w_{2,j+1} \cdot 4^8 + w_{3,j+1} \cdot 4^{8 \cdot 2} + w_{4,j+1} \cdot 4^{8 \cdot 3}$ \\
    $w_{o,j+2} = w_{2,j+1} \cdot 4^{8-2} + w_{3,j+1} \cdot 4^{8 \cdot 2-2} + w_{4,j+1} \cdot 4^{8 \cdot 3 - 2}
    + w_{1,j+1} \cdot 4^{8 \cdot 3 - 5} + w_{3,j+1} \cdot 4^{8-5} + w_{4,j+1} \cdot 4^{8 \cdot 2 - 5}
    + w_{1,j+1} \cdot 4^{8 \cdot 2-6} + w_{2,j+1} \cdot 4^{8 \cdot 3-6} + w_{4,j+1} \cdot 4^{8 - 6} + w_{1,j+2} + w_{2, j+2} + w_{3, j+2}$ \\
    7 plookup constraints \\
\end{center}

\paragraph{The $\Sigma_1$ function}
contain sparse mapping subcircuit with base $2$.
Let $a$ be divided to 8 bits-chunks $a_0, a_1, a_2, a_3$.
The values $a'_0, a'_1, a'_2, a'_3$ are in sparse form, and $a'$ is a sparse $a$.
We need the following lookup tables:
\begin{enumerate}
    \item \textbf{SHA-256 NORMALIZE7}: Read $a_i$ to $a'_i$
    \item \textbf{SHA-256 8ROT6 32}: Read $a'_0$ to $r_1$
    \item \textbf{SHA-256 8ROT3 32}: Read $a'_1$ to $r_2$
    \item \textbf{SHA-256 8ROT1 32}: Read $a'_3$ to $r_3$
\end{enumerate}
\begin{center}
    \begin{tabular}{ c|c|c|c|c|c }
        & $w_1$  & $w_2$  & $w_3$  & $w_4$  & $w_o$      \\
        \hline
        j + 0 & $a_0$  & $ a_1$ & $a_2$  & $a_3$  & a          \\
        j + 1 & $a'_0$ & $a'_1$ & $a'_2$ & $a'_3$ & a'         \\
        j + 2 & $r1 $  & $r_2$  & $r_3$  &        & $\Sigma_1$ \\
    \end{tabular}
\end{center}
Sparse map gate constraints:
\begin{center}
    $w_{o,j} = w_{1,j} + w_{2,j} \cdot 2^8 + w_{3,j} \cdot 2^{8 \cdot 2} + w_{4,j} \cdot 2^{8 \cdot 3}$ \\
    $w_{o,j+1} = w_{1,j+1} + w_{2,j+1} \cdot 7^8 + w_{3,j+1} \cdot 7^{8 \cdot 2} + w_{4,j+1} \cdot 7^{8 \cdot 3}$ \\
    $w_{o,j+2} =  w_{2,j+1} \cdot 7^{8-6} + w_{3,j+1} \cdot 7^{8 \cdot 2 - 6} + w_{7,j+1} \cdot 4^{8 \cdot 3 - 6}
    + w_{1,j+1} \cdot 7^{8 \cdot 3 - 3} + w_{3,j+1} \cdot 7^{8-3} + w_{4,j+1} \cdot 7^{8 \cdot 2 - 3}+ w_{1,j+1} \cdot 7^{8-1}
    + w_{2,j+1} \cdot 7^{8 \cdot 2-1} + w_{3,j+1} \cdot 7^{8 \cdot 3 - 1}+ w_{1,j+2} + w_{2, j+2} + w_{3, j+2}$ \\
    7 plookup constraints \\
\end{center}

The sparse values $\Sigma_0$ and $\Sigma_1$ have to be normalized.
We use \textbf{SHA256 NORMALIZE4} and \textbf{SHA256 NORMALIZE7}.
\begin{center}
    \begin{tabular}{ c|c|c|c|c|c }
        & $w_1$  & $w_2$  & $w_3$  & $w_4$  & $w_o$      \\
        \hline
        j + 0 & $a'_0$ & $a'_1$ & $a'_2$ & $a'_3$ &            \\
        j + 1 & $a_0$  & $ a_1$ & $a_2$  & $a_3$  & $\Sigma_i$ \\
    \end{tabular}
\end{center}

Normalize gate constraints:
\begin{center}
    $w_{o,j-1} = w_{4,j} \cdot 4^8 \cdot 3 + w_{3,j} \cdot 4^8 \cdot 2 + w_{2,j} \cdot 4^8 + w_{1,j}$ for $\Sigma_1$ replace 4 with 7\\
    $w_{o,i} = w_{4,i} \cdot 256^3 + w_{3,i} \cdot 256^2 + w_{2,i} \cdot 256 + w_{1,i}$ \\
    7 plookup constraints \\
\end{center}

\paragraph{The Maj function}
contain sparse mapping subcircuit with base $2$ for $a, b ,c$.
Let $a; b; c$ be divided to 8 bits-chunks $a_0, a_1, a_2, a_3; b_0, b_1, b_2, b_3; c_0, c_1, c_2, c_3$.
The values $a'_0, a'_1, a'_2, a'_3$ are in sparse form, and $a'$ is a sparse $a$.
Similarly for b and c.
Note, that $a$ we already have in the sparse from $\Sigma_0$ in the circuit.
The variables $b$ and $c$ were represented in sparse form in the previous rounds or it is public inputs.
\begin{center}
    \begin{tabular}{ c|c|c|c|c|c }
        & $w_1$  & $w_2$  & $w_3$  & $w_4$  & $w_o$ \\
        \hline
        j - k & $a'_0$ & $a'_1$ & $a'_2$ & $a'_3$ & a'    \\
        ... & & & & \\
        j - l & $b'_0$ & $b'_1$ & $b'_2$ & $b'_3$ & b'    \\
        ... & & & & \\
        j - t & $c'_0$ & $c'_1$ & $c'_2$ & $c'_3$ & c'    \\
        ... & & & & \\
        j + 0 & a'     & b'     & c'     &        & maj   \\
    \end{tabular}
\end{center}
Sparse map gate constraints:
\begin{center}
    $w_{o, j} = w_{1,j} + w_{2, j} + w_{3, j}$ \\
\end{center}

The sparse values $maj$ have to be normalized.
We use \textbf{SHA256 MAJ NORMALIZE4}
\begin{center}
    \begin{tabular}{ c|c|c|c|c|c }
        & $w_1$  & $w_2$  & $w_3$  & $w_4$  & $w_o$ \\
        \hline
        j + 0 & $a'_0$ & $a'_1$ & $a'_2$ & $a'_3$ &       \\
        j + 1 & $a_0$  & $ a_1$ & $a_2$  & $a_3$  & $maj$ \\
    \end{tabular}
\end{center}

Normalize gate constraints:
\begin{center}
    $w_{o,i} = w_{4,i} \cdot 256^3 + w_{3,i} \cdot 256^2 + w_{2,i} \cdot 256 + w_{1,i}$ \\
\end{center}

The final addition requires one add gate.

\paragraph{The Ch function}
contain sparse mapping subcircuit with base $2$ for $e, f ,g$.
Let $e; f; g$ be divided to 8 bits-chunks $e_0, e_1, e_2, e_3; f_0, f_1, f_2, f_3; g_0, g_1, g_2, g_3$.
The values $e'_0, e'_1, e'_2, e'_3$ are in sparse form, and $e'$ is a sparse $e$.
Similarly for b and c.
Note, that $e$ we already have in the sparse from $\Sigma_1$ in the circuit.
The variables $f$ and $g$ were represented in sparse form in the previous rounds or it is public inputs.
\begin{center}
    \begin{tabular}{ c|c|c|c|c|c }
        & $w_1$  & $w_2$  & $w_3$  & $w_4$  & $w_o$ \\
        \hline
        j - k & $a'_0$ & $a'_1$ & $a'_2$ & $a'_3$ & a'    \\
        ... & & & & \\
        j - l & $b'_0$ & $b'_1$ & $b'_2$ & $b'_3$ & b'    \\
        ... & & & & \\
        j - t & $c'_0$ & $c'_1$ & $c'_2$ & $c'_3$ & c'    \\
        ... & & & & \\
        j + 0 & a'     & b'     & c'     &        & ch    \\
    \end{tabular}
\end{center}
Sparse map gate constraints:
\begin{center}
    $w_{o, j} = w_{1,j} + 2*w_{2, j} + 3*w_{3, j}$ \\
\end{center}

The sparse values $ch$ have to be normalized.
We use \textbf{SHA256 CH NORMALIZE7}
\begin{center}
    \begin{tabular}{ c|c|c|c|c|c }
        & $w_1$  & $w_2$  & $w_3$  & $w_4$  & $w_o$ \\
        \hline
        j + 0 & $a'_0$ & $a'_1$ & $a'_2$ & $a'_3$ &       \\
        j + 1 & $a_0$  & $ a_1$ & $a_2$  & $a_3$  & $ch$  \\
    \end{tabular}
\end{center}

Normalize gate constraints:
\begin{center}
    $w_{o,i} = w_{4,i} \cdot 256^3 + w_{3,i} \cdot 256^2 + w_{2,i} \cdot 256 + w_{1,i}$ \\
\end{center}

The final addition requires one add gate.

The updating of variables for new rounds costs 10 add gates.

Producing the final hash value costs two add gates.

\subsection{SHA2-512 Circuit}
\label{section:sha512}
SHA-512 uses the similar logical functions as in \ref{section:sha256} which operates on $64$-bits words.
Thus each input uses the same range proof which extended to 64-bits.

\paragraph{Range proof that $a < 2^{64}$}
Let $a = \{ a_0, ..., a_{32} \}$, where $a_i$ is two bits.
\begin{center}
    \begin{tabular}{ c|c|c|c|c|c }
        & $w_1$    & $w_2$     & $w_3$    & $w_4$    & $w_o$ \\
        \hline
        j + 0 & $a_{29}$ & $ a_{30}$ & $a_{31}$ & $a_{32}$ & acc   \\
        j + 1 & $a_{25}$ & $ a_{26}$ & $a_{27}$ & $a_{28}$ & acc   \\
        ...   &          &           &          &          &       \\
        j + 6 & $a_4$    & $a_5$     & $a_6$    & $a_7$    & acc   \\
        j + 7 & $a_0 $   & $a_1$     & $a_2$    & $a_3$    & a     \\
    \end{tabular}
\end{center}
Range gate constraints:
\begin{center}
    $w_{1,i}(w_{1,i}-1)(w_{1,i}-2)(w_{1,i} -3) + w_{2,i}(w_{2,i}-1)(w_{2,i}-2)(w_{2,i} -3)
    + w_{3,i}(w_{3,i}-1)(w_{3,i}-2)(w_{3,i} -3) + w_{4,i}(w_{4,i}-1)(w_{4,i}-2)(w_{4,i} -3)$ \\
    $w_{o,i} = w_{o, i - 1} \cdot 4^4 + w_{4,i} \cdot 4^3 + w_{3,i} \cdot 4^2 + w_{2,i} \cdot 4 + w_{1,i}$
\end{center}
The range proofs are included for each input data block.

\paragraph{The function $\sigma_0$} contain sparse mapping subcircuit with base $4$.
Let $a$ be divided to 8 bits-chunks $a_0, a_1, a_2, ..., a_7$.
The values $a'_0, a'_1, a'_2,...,  a'_7$ are in sparse form, and $a'$ is a sparse $a$.
We need the following lookup tables:
\begin{enumerate}
    \item \textbf{SHA-256 NORMALIZE4}: Read $a_i$ to $a'_i$
    \item \textbf{SHA-512 8ROT1 64}: Read $a'_0$ to $r_1$
    \item \textbf{SHA-512 8SHR7 64}: Read $a'_0$ to $r_3$
\end{enumerate}
\begin{center}
    \begin{tabular}{ c|c|c|c|c|c }
        & $w_1$  & $w_2$   & $w_3$  & $w_4$  & $w_o$      \\
        \hline
        j + 0 & $a_0$  & $ a_1$  & $a_2$  & $a_3$  & $a_4$      \\
        j + 1 & $a'_0$ & $a'_1$  & $a'_2$ & $a'_3$ & a          \\
        j + 2 & $a_5 $ & $a_6$   & $a_7$  & $a_4'$ & $\sigma_0$ \\
        j + 3 & $a'_5$ & $ a'_6$ & $a'_7$ & $r_1$  & $r_2$      \\
    \end{tabular}
\end{center}
Sparse map gate constraints:
\begin{center}
    $w_{o,j+1} = w_{1,j} + w_{2,j} \cdot 2^8 + w_{3,j} \cdot 2^{8 \cdot 2} + w_{4,j} \cdot 2^{8 \cdot 3}
    + w_{o,j} \cdot 2^{8 \cdot 4} + w_{1,j+2} \cdot 2^{8 \cdot 5} + w_{2,j+2} \cdot 2^{8 \cdot 6} + w_{3,j+2} \cdot 2^{8 \cdot 7}$ \\
    $w_{o,j+2} =  w_{2,j+1} \cdot 4^{8-1} + w_{3,j+1} \cdot 4^{8 \cdot 2-1} + w_{4,j+1} \cdot 4^{8 \cdot 3 - 1}
    + w_{4,j+2} \cdot 4^{8 \cdot 4 - 1} + w_{1,j+3} \cdot 4^{8 \cdot 5 - 1} + w_{2,j+3} \cdot 4^{8 \cdot 6 - 1}
    + w_{3,j+3} \cdot 4^{8 \cdot 7 - 1} + w_{1,j+1} \cdot 4^{8 \cdot 7} + w_{2,j+1} + w_{3,j+1} \cdot 4^{8}
    + w_{4,j+1} \cdot 4^{8 \cdot 2} +w_{4,j+2} \cdot 4^{8 \cdot 3} + w_{1,j+3} \cdot 4^{8 \cdot 4}
    + w_{2,j+3} \cdot 4^{8 \cdot 5} + w_{3,j+3} \cdot 4^{8 \cdot 6} + w_{2,j+1} \cdot 4^{8-7}
    + w_{3,j+1} \cdot 4^{8 \cdot 2-7} + w_{4,j+1} \cdot 4^{8 \cdot 3 - 7} + w_{4,j+2} \cdot 4^{8 \cdot 4 - 7}
    + w_{1,j+3} \cdot 4^{8*5 - 7} + w_{2,j+3} \cdot 4^{8 \cdot 6 - 7} + w_{3,j+3} \cdot 4^{8 \cdot 7 - 7}
    + w_{4, j+3} + w_{o, j+3}$ \\
    10 plookup constraints \\
\end{center}

\paragraph{The function $\sigma_1$} contain sparse mapping subcircuit with base $4$.
Let $a$ be divided to 8 bits-chunks $a_0, a_1, a_2, ..., a_7$.
The values $a'_0, a'_1, a'_2,...,  a'_7$ are in sparse form, and $a'$ is a sparse $a$.
We need the following lookup tables:
\begin{enumerate}
    \item \textbf{SHA-256 NORMALIZE4}: Read $a_i$ to $a'_i$
    \item \textbf{SHA-512 8ROT3 64}: Read $a'_2$ to $r_1$
    \item \textbf{SHA-512 8ROT5 SHR6 64}: Read $a'_7 + a'_0$ to $r_2$
\end{enumerate}
\begin{center}
    \begin{tabular}{ c|c|c|c|c|c }
        & $w_1$  & $w_2$   & $w_3$  & $w_4$  & $w_o$      \\
        \hline
        j + 0 & $a_0$  & $ a_1$  & $a_2$  & $a_3$  & $a_4$      \\
        j + 1 & $a'_0$ & $a'_1$  & $a'_2$ & $a'_3$ & a          \\
        j + 2 & $a_5 $ & $a_6$   & $a_7$  & $a_4'$ & $\sigma_1$ \\
        j + 3 & $a'_5$ & $ a'_6$ & $a'_7$ & $r_1$  & $r_2$      \\
    \end{tabular}
\end{center}
Sparse map gate constraints:
\begin{center}
    $w_{o,j+1} = w_{1,j} + w_{2,j} \cdot 2^8 + w_{3,j} \cdot 2^{8 \cdot 2} + w_{4,j} \cdot 2^{8 \cdot 3}
    + w_{o,j} \cdot 2^{8 \cdot 4} + w_{1,j+2} \cdot 2^{8 \cdot 5} + w_{2,j+2} \cdot 2^{8 \cdot 6} + w_{3,j+2} \cdot 2^{8 \cdot 7}$ \\
    $w_{o,j+2} =  w_{1,j+1} \cdot 4^{64 -19} + w_{2,j+1} \cdot 4^{64 + (8-19)}
    + w_{4,j+1} \cdot 4^{8 \cdot 3 - 19} + w_{4,j+2} \cdot 4^{8 \cdot 4 - 19}
    + w_{1,j+3} \cdot 4^{8 \cdot 5 - 19} + w_{2,j+3} \cdot 4^{8 \cdot 6 - 19}
    + w_{3,j+3} \cdot 4^{8 \cdot 7 -19} + w_{1,j+1} \cdot 4^{64  - 61)} + w_{2,j+1} \cdot 4^{64 + (8 - 61)}
    + w_{3,j+1} \cdot 4^{64 + (8 \cdot 2 - 61)} + w_{4,j+1} \cdot 4^{64 + (8 \cdot 3- 61)}
    + w_{4,j+2} \cdot 4^{64 + (8 \cdot 4 - 61)} + w_{1,j+3} \cdot 4^{64 + (8 \cdot 5 - 61)}
    + w_{2,j+3} \cdot 4^{64 +(8 \cdot 6- 61)} + w_{2,j+1} \cdot 4^{8-6} + w_{3,j+1} \cdot 4^{8 \cdot 2-6}
    + w_{4,j+1} \cdot 4^{8 \cdot 3 - 6} + w_{4,j+2} \cdot 4^{8 \cdot 4 - 6} + w_{1,j+3} \cdot 4^{8 \cdot 5 - 6}
    + w_{2,j+3} \cdot 4^{8 \cdot 6 - 6} + w_{3,j+3} \cdot 4^{8 \cdot 7 - 6} + w_{4, j+3} + w_{o, j+3}$ \\
    10 plookup constraints \\
\end{center}

The sparse values $\sigma_0$ and $\sigma_1$ have to be normalized.
The final addition requires one add gate.
Note, that $a'$ already initialized in the row $j - 2$.
We use \textbf{SHA256 NORMALIZE4}
\begin{center}
    \begin{tabular}{ c|c|c|c|c|c }
        & $w_1$  & $w_2$  & $w_3$  & $w_4$  & $w_o$      \\
        \hline
        j + 0 & $a'_0$ & $a'_1$ & $a'_2$ & $a'_3$ & $acc$      \\
        j + 1 & $a_0$  & $ a_1$ & $a_2$  & $a_3$  & 0          \\
        j + 2 & $a'_4$ & $a'_5$ & $a'_6$ & $a'_7$ & $\sigma_i$ \\
        j + 3 & $a_4$  & $ a_5$ & $a_6$  & $a_7$  &            \\
    \end{tabular}
\end{center}

Normalize gate constraints:
\begin{center}
    $w_{o,j+2} = w_{4,j+1} \cdot 256^3 + w_{3,j+1} \cdot 256^2 + w_{2,j+1} \cdot 256 + w_{1,j+1}
    + w_{1,j+3} \cdot 256^4$ \\
    $+ w_{2,j+3} \cdot 256^5+ w_{3,j+3} \cdot 256^6 + w_{4,j+4} \cdot 256^7$ \\
    $w_{o,j} = w_{o, j - 2} - (w_{4,j} \cdot 256^3 + w_{3,j} \cdot 256^2 + w_{2,j} \cdot 256 + w_{1,j})$ \\
    $w_{o,j+1} = w_{o,j} - ( w_{1,j+3} \cdot 256^4 + w_{2,j+3} \cdot 256^5+ w_{3,j+3} \cdot 256^6 + w_{4,j+4} \cdot 256^7)$

    8 plookup constraints \\
\end{center}

\paragraph{The $\Sigma_0$ function}
contain sparse mapping subcircuit with base $4$.
Let $a$ be divided to 7-bits chunks $a_0, a_1, a_2,a_3$ and 9 bits-chunks $ a_4, a_5, a_6, a_7$.
The values $a'_0, a'_1, a'_2,...,  a'_7$ are in sparse form, and $a'$ is a sparse $a$.
We need the following lookup tables:
\begin{enumerate}
    \item \textbf{SHA-512 9NORMALIZE4}: Read $a_i$ to $a'_i$
    \item \textbf{SHA-512 7NORMALIZE4}: Read $a_i$ to $a'_i$
    \item \textbf{SHA-512 9ROT6 64}: Read $a'_4$ to $r_2$
    \item \textbf{SHA-512 9ROT2 64}: Read $a'_5$ to $r_3$
\end{enumerate}
\begin{center}
    \begin{tabular}{ c|c|c|c|c|c }
        & $w_1$  & $w_2$   & $w_3$  & $w_4$  & $w_o$      \\
        \hline
        j + 0 & $a_0$  & $ a_1$  & $a_2$  & $a_3$  & $a_4$      \\
        j + 1 & $a'_0$ & $a'_1$  & $a'_2$ & $a'_3$ & a          \\
        j + 2 & $a_5 $ & $a_6$   & $a_7$  & $a_4'$ & $\Sigma_0$ \\
        j + 3 & $a'_5$ & $ a'_6$ & $a'_7$ & $r_1$  & $r_2$      \\
    \end{tabular}
\end{center}

Sparse map gate constraints:
\begin{center}
    $w_{o,j+1} = w_{1,j} + w_{2,j} \cdot 2^7 + w_{3,j} \cdot 2^{7 \cdot 2} + w_{4,j} \cdot 2^{7 \cdot 3}
    + w_{o,j} \cdot 2^{7 \cdot 4} + w_{1,j+2} \cdot 2^{7 \cdot 4 + 9}
    + w_{2,j+2} \cdot 2^{7 \cdot 4 + 9 \cdot 2} + w_{3,j+2} \cdot 2^{7 \cdot 4 + 9 \cdot 3}$ \\
    $w_{o,j+2} =  w_{4,j+2} + w_{1,j+3} \cdot 4^9 + w_{2,j+3} \cdot 4^{9 \cdot 2}
    + w_{3,j+3} \cdot 4^{9 \cdot 3} + w_{1,j+1} \cdot 4^{9 \cdot 4} + w_{2,j+1} \cdot 4^{9 \cdot 4 +7}$ \\
    $+ w_{3,j+1} \cdot 4^{9 \cdot 4 +7 \cdot 2} + w_{4,j+1} \cdot 4^{9 \cdot 4 +7 \cdot 3}
    + w_{1,j+1} \cdot 4^{64  - 34)} + w_{2,j+1} \cdot 4^{64 + (7 - 34)}
    + w_{3,j+1} \cdot 4^{64 + (7 \cdot 2 - 34)} + w_{4,j+1} \cdot 4^{64 + (7 \cdot 3- 34)}
    + w_{1,j+3} \cdot 4^{7 \cdot 4 + 9 - 34} + w_{2,j+3} \cdot 4^{7 \cdot 4 + 9  \cdot 2  -34}
    + w_{3,j+3} \cdot 4^{7 \cdot 4 + 9 \cdot 3 - 34} + w_{1,j+1} \cdot 4^{64  - 39)}
    + w_{2,j+1} \cdot 4^{64 + (7 - 39)} + w_{3,j+1} \cdot 4^{64 + (7 \cdot 2 - 39)}
    + w_{4,j+1} \cdot 4^{64 + (7 \cdot 3- 39)} +w_{4,j+2} \cdot 4^{64 + (7 \cdot 4 - 39)}
    + w_{2,j+3} \cdot 4^{7 \cdot 4 + 9 \cdot 2 - 39}
    + w_{3,j+3} \cdot 4^{7 \cdot 4 + 9 \cdot 3- 39} + w_{4, j+3} + w_{o, j+3}$ \\
    10 plookup constraints \\
\end{center}

\paragraph{The $\Sigma_1$ function}
contain sparse mapping subcircuit with base $7$.
Let $a$ be divided to 7-bits chunks $a_0, a_1, a_2,a_3$ and 9 bits-chunks $ a_4, a_5, a_6, a_7$.
The values $a'_0, a'_1, a'_2,...,  a'_7$ are in sparse form, and $a'$ is a sparse $a$.
We need the following lookup tables:
\begin{enumerate}
    \item \textbf{SHA-512 9NORMALIZE7}: Read $a_i$ to $a'_i$
    \item \textbf{SHA-512 7NORMALIZE7}: Read $a_i$ to $a'_i$
    \item \textbf{SHA-512 7ROT4 32}: Read $a'_2$ to $r_2$
    \item \textbf{SHA-512 9ROT4 32}: Read $a'_5$ to $r_3$
\end{enumerate}
\begin{center}
    \begin{tabular}{ c|c|c|c|c|c }
        & $w_1$  & $w_2$   & $w_3$  & $w_4$  & $w_o$      \\
        \hline
        j + 0 & $a_0$  & $ a_1$  & $a_2$  & $a_3$  & $a_4$      \\
        j + 1 & $a'_0$ & $a'_1$  & $a'_2$ & $a'_3$ & a          \\
        j + 2 & $a_5 $ & $a_6$   & $a_7$  & $a_4'$ & $\Sigma_1$ \\
        j + 3 & $a'_5$ & $ a'_6$ & $a'_7$ & $r_1$  & $r_2$      \\
    \end{tabular}
\end{center}

Sparse map gate constraints:
\begin{center}
    $w_{o,j+1} = w_{1,j} + w_{2,j} \cdot 2^7 + w_{3,j} \cdot 2^{7 \cdot 2} + w_{4,j} \cdot 2^{7 \cdot 3}
    + w_{o,j} \cdot 2^{7 \cdot 4} + w_{1,j+2} \cdot 2^{7 \cdot 4 + 9}
    + w_{2,j+2} \cdot 2^{7 \cdot 4 + 9 \cdot 2} + w_{3,j+2} \cdot 2^{7 \cdot 4 + 9 \cdot 3}$ \\
    $w_{o,j+2} =  w_{3,j+1} + w_{4,j+1} \cdot 7^7 + w_{4,j+2} \cdot 7^{7 \cdot 2}
    + w_{1,j+3} \cdot 7^{7 \cdot 2+9} + w_{2,j+3} \cdot 7^{7 \cdot 2+9 \cdot 2}
    + w_{3,j+3} \cdot 7^{9 \cdot 3 +7 \cdot 2} + w_{1,j+1} \cdot 7^{9 \cdot 4 +7 \cdot 2}
    + w_{2,j+1} \cdot 7^{9 \cdot 4 +7 \cdot 3} + w_{1,j+1} \cdot 7^{64  - 18)}
    + w_{2,j+1} \cdot 7^{64 + (7 - 18)} + w_{4,j+1} \cdot 7^{7 \cdot 3 - 18}
    + w_{4,j+2} \cdot 7^{7 \cdot 4- 18}
    + w_{1,j+3} \cdot 7^{7 \cdot 4 + 9  - 18}
    + w_{2,j+3} \cdot 7^{7 \cdot 4 + 9 \cdot 2 - 18}
    + w_{3,j+3} \cdot 7^{7 \cdot 4 + 9 \cdot 3 - 18} + w_{1,j+1} \cdot 7^{64  - 41)}
    + w_{2,j+1} \cdot 7^{64 + (7 - 41)} + w_{3,j+1} \cdot 7^{64 + (7 \cdot 2 - 41)}
    + w_{4,j+1} \cdot 7^{64 + (7 \cdot 3- 41)} +w_{4,j+2} \cdot 7^{64 + (7 \cdot 3 + 9 - 41)}
    + w_{2,j+3} \cdot 7^{64 + (7 \cdot 3 + 9 \cdot 2 -41)}
    + w_{3,j+3} \cdot 7^{7 \cdot 3 + 9 \cdot 3- 41} + w_{4, j+3} + w_{o, j+3}$ \\
    10 plookup constraints \\
\end{center}

The sparse values $\Sigma_0$ and $\Sigma_1$ have to be normalized.
We use \textbf{SHA256 NORMALIZE4} and \textbf{SHA256 NORMALIZE7}.
Note, that $a'$ already initialized in the row $j - 2$.
\begin{center}
    \begin{tabular}{ c|c|c|c|c|c }
        & $w_1$  & $w_2$  & $w_3$  & $w_4$  & $w_o$      \\
        \hline
        j + 0 & $a'_0$ & $a'_1$ & $a'_2$ & $a'_3$ & $a''$      \\
        j + 1 & $a_0$  & $ a_1$ & $a_2$  & $a_3$  & $0$        \\
        j + 2 & $a'_4$ & $a'_5$ & $a'_6$ & $a'_7$ & $\Sigma_i$ \\
        j + 3 & $a_4$  & $ a_5$ & $a_6$  & $a_7$  &            \\
    \end{tabular}
\end{center}

Normalize gate constraints:
\begin{center}
    $w_{o,j+2} = w_{4,j+1} \cdot 256^3 + w_{3,j+1} \cdot 256^2 + w_{2,j+1} \cdot 256
    + w_{1,j+1} + w_{1,j+3} \cdot 256^4$ \\
    $+ w_{2,j+3} \cdot 256^5+ w_{3,j+3} \cdot 256^6 + w_{4,j+4} \cdot 256^7$ \\
    $w_{o,j} = w_{1,j-3} + w_{2,j-3} \cdot 4^7 + w_{3,j-3} \cdot 4^{7 \cdot 2}
    + w_{4,j-3} \cdot 4^{7 \cdot 3} + w_{4,j-2} \cdot 4^{7 \cdot 4}
    + w_{1,j-1} \cdot 7^{7 \cdot 4+9}$ \\
    $+ w_{2,j-1} \cdot 7^{7 \cdot 4 + 9 \cdot 2}
    + w_{2,j-1} \cdot 7^{7 \cdot 4 + 9 \cdot 3}$ for maj or ch function. For $\Sigma_1$ replace 4 with 7\\
    $w_{o,j+1} = w_{o, j - 2} - (w_{4,j} \cdot 256^3 + w_{3,j} \cdot 256^2 + w_{2,j} \cdot 256 + w_{1,j} + w_{1,j+3} \cdot 256^4 + w_{2,j+3} \cdot 256^5+ w_{3,j+3} \cdot 256^6 + w_{4,j+4} \cdot 256^7)$

    8 plookup constraints \\
\end{center}

\paragraph{The Maj function}
contain sparse mapping subcircuit with base $4$ for $a, b ,c$.
Note, that the sparse chunks of $a$ we already have in $\Sigma_0$ in the circuit.
The variables $b$ and $c$ were represented in sparse chunks in the previous rounds or it is public inputs.
\begin{center}
    \begin{tabular}{ c|c|c|c|c|c }
        & $w_1$ & $w_2$ & $w_3$ & $w_4$ & $w_o$ \\
        \hline
        j & $a'$  & $b'$  & $c'$  &       & maj   \\
    \end{tabular}
\end{center}
Sparse map gate constraints:
\begin{center}
    $w_{o, j} = w_{1,j} + w_{2, j} + w_{3, j}$ \\
\end{center}

The sparse values $maj$ have to be normalized.
We use \textbf{SHA256 MAJ NORMALIZE4}
Note, that the sparse $maj$ already initialized in the row $j - 1$.
\begin{center}
    \begin{tabular}{ c|c|c|c|c|c}
        & $w_1$  & $w_2$  & $w_3$  & $w_4$  & $w_o$ \\
        \hline
        j + 0 & $a'_0$ & $a'_1$ & $a'_2$ & $a'_3$ & $acc$ \\
        j + 1 & $a_0$  & $ a_1$ & $a_2$  & $a_3$  & 0     \\
        j + 2 & $a'_4$ & $a'_5$ & $a'_6$ & $a'_7$ & $maj$ \\
        j + 3 & $a_4$  & $ a_5$ & $a_6$  & $a_7$  &       \\
    \end{tabular}
\end{center}

Normalize gate constraints:
\begin{center}
    $w_{o,j+2} = w_{4,j+1} \cdot 256^3 + w_{3,j+1} \cdot 256^2 + w_{2,j+1} \cdot 256 + w_{1,j+1}
    + w_{1,j+3} \cdot 256^4$ \\
    $+ w_{2,j+3} \cdot 256^5+ w_{3,j+3} \cdot 256^6 + w_{4,j+4} \cdot 256^7$ \\
    $w_{o,j} = w_{o, j - 1} - (w_{4,j} \cdot 256^3 + w_{3,j} \cdot 256^2 + w_{2,j} \cdot 256 + w_{1,j})$ \\
    $w_{o,j+1} = w_{o,j} - ( w_{1,j+3} \cdot 256^4 + w_{2,j+3} \cdot 256^5+ w_{3,j+3} \cdot 256^6 + w_{4,j+4} \cdot 256^7)$

    8 plookup constraints \\
\end{center}


The final addition requires one add gate.

\paragraph{The Ch function}
contain sparse mapping subcircuit with base $7$ for $e, f ,g$.
Note, that $e$ we already have in the sparse from $\Sigma_1$ in the circuit.
The variables $f$ and $g$ were represented in sparse form in the previous rounds or it is public inputs.
\begin{center}
    \begin{tabular}{ c|c|c|c|c|c }
        & $w_1$ & $w_2$ & $w_3$ & $w_4$ & $w_o$ \\
        \hline
        j + 0 & $e'$  & $f'$  & $g'$  &       & ch    \\
    \end{tabular}
\end{center}
Sparse map gate constraints:
\begin{center}
    $w_{o, j} = w_{1,j} + 2 \cdot w_{2, j} + 3 \cdot w_{3, j}$ \\
\end{center}

The sparse values $ch$ have to be normalized.
Note, that $ch$ already initialized in the row $j - 1$.
We use \textbf{SHA256 CH NORMALIZE7}
\begin{center}
    \begin{tabular}{ c|c|c|c|c|c }
        & $w_1$  & $w_2$  & $w_3$  & $w_4$  & $w_o$ \\
        \hline
        j + 0 & $a'_0$ & $a'_1$ & $a'_2$ & $a'_3$ & $acc$ \\
        j + 1 & $a_0$  & $ a_1$ & $a_2$  & $a_3$  & 0     \\
        j + 2 & $a'_4$ & $a'_5$ & $a'_6$ & $a'_7$ & $ch$  \\
        j + 3 & $a_4$  & $ a_5$ & $a_6$  & $a_7$  &       \\
    \end{tabular}
\end{center}

Normalize gate constraints:
\begin{center}
    $w_{o,j+2} = w_{4,j+1} \cdot 256^3 + w_{3,j+1} \cdot 256^2 + w_{2,j+1} \cdot 256 + w_{1,j+1}
    + w_{1,j+3} \cdot 256^4 + w_{2,j+3} \cdot 256^5$ \\
    $+ w_{3,j+3} \cdot 256^6 + w_{4,j+4} \cdot 256^7$ \\
    $w_{o,j} = w_{o, j - 1} - (w_{4,j} \cdot 256^3 + w_{3,j} \cdot 256^2 + w_{2,j} \cdot 256 + w_{1,j})$ \\
    $w_{o,j+1} = w_{o,j} - ( w_{1,j+3} \cdot 256^4 + w_{2,j+3} \cdot 256^5+ w_{3,j+3} \cdot 256^6 + w_{4,j+4} \cdot 256^7)$ \\
    8 plookup constraints \\
\end{center}

The final addition requires one add gate.

The updating of variables for new rounds costs 10 add gates.

Producing the final hash value costs two add gates.
